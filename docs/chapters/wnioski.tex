\section{Wnioski (Zofia Sosińska)}
Finalnie osiągnięto ustalony cel. Stworzony prototyp zawiera w sobie wszystkie mechaniki gry RTS,
które na początku zostały ustalone, a świat gry oddaje realia wczesnego średniowiecza. 
Wśród funkcjonalności znalazły się takie, jak budowanie budynków, zarządzanie
jednostkami i surowcami oraz walka z przeciwnikami. Przy tym zwrócono uwagę, aby jak najdokładniej oddać 
realia, na które stylizowany jest program. Idąc tokiem myślenia ludzi wczesnego średniowiecza, zaniechano
implementacji mapy, stawiając na użycie kompasu. Klimat epoki jest odczuwalny dzięki stylizacji budowli oraz ubrań i przedmiotów postaci,
jak również przy samym prowadzeniu dialogu z nimi. Wypowiedzi zarówno gracza, jak i rozmówcy są stylizowane na język ludzi tamtych czasów.

Przy dalszym rozwoju projektu warte uwagi będzie dodanie wysokiej jakości efektów audiowizualnych. Po dodaniu do stworzonego prototypu takich
aspektów produkt nabrałby unikalnego charakteru. Skomponowanie muzyki słysznej w tle oraz sygnalizującej wykonanie jakiejś czynności podniosłoby immersyjność i klimatyczność.
Stworzenie modeli i tekstur wyglądających estetycznie sprawiłoby, że rozgrywka stanie się przyjemniejsza.

Program został oceniony przez osobę niezależną jako dobra baza do późniejszego rozwinięcia. Zaimplementowane mechaniki często były intuicyjne i przyjemne
w obsłudze. Fabuła gry pozwoliła na użycie wszystkich zaplanowanych funkcjonalności gry RTS, a otoczenie i dialogi z postaciami oddały klimat
epoki wczesnego średniowiecza.