\section{Nawigacja (Zofia Sosińska)}\label{chap:naw}

Kluczowym dla gry założeniem jest ułatwienie graczowi wczucia się w realia świata, w którym się znajduje. 
Jako jeden z głównych warunków pogłębienia immersji uwypuklono brak implementacji mapy, na której gracz widziałby świat. 
W ten sposób nie upraszczamy mu poruszania się i odnajdywania lokacji tak,
jak i człowiek w realnym świecie w czasach średniowiecznych nie kierował się zapisanymi na kartce kartograficznymi obrazami, 
ale własną i zdobytą od innych wiedzą o otaczającym go terenie. 

Jedyną pomocą, jaką otrzyma gracz, będzie pasek obrazujący pole widzenia granej postaci.
Pierwszą rolą narzędzia będzie pokazanie kierunku świata, który znajduje się w polu widzenia gracza.
Zakładamy, że grana postać potrafi sama taką informację odczytać, chociażby z położenia Słońca.

Kolejną informacją na omawianym elemencie będzie miejsce, w którym znajduje się przeciwnik. 
Dotyczy to antagonistów widocznych w polu widzenia, jak i ukrytych za ścianą. 
Druga część będzie logicznie ponieważ zakładamy, że postać gracza może usłyszeć 
wroga za przeszkodą.

