\section{Analiza i specyfikacja wymagań (Bogna Lew, Zofia Sosińska)}\label{s:wymagania}
W niniejszej sekcji przedstawiono specyfikę wymagań funkcjonalnych, pozafunkcjonalnych oraz tych, wynikających z
głównych założeń projektu. Dodatkowo zawiera ona diagramy przypadków użycia, maszyny stanów oraz klas prototypowej gry.

\subsection{Specyfika wymagań wynikających z założeń projektu}
Z punktu widzenia projektu kluczowe jest jak najdokładniejsze oddanie realiów historycznych przy jednoczesnym
uwzględnieniu jakości rozgrywki gracza oraz cech charakterystycznych dla gier typu RTS. Z założeń wynika, że fabuła
gry powinna zostać osadzona w czasach sprzed wielkich odkryć geograficznych. Na tej podstawie zostały zdefiniowane
dodatkowe wymagania, które powinien spełniać prototyp.

Gra powinna zawierać:
\begin{itemize}
  \item sposób nawigacji jak najdokładniej odpowiadający temu stosowanemu w wybranej epoce,
  \item postacie:
  \begin{itemize}
    \item stylistycznie pasujące do realiów historycznych,
    \item wykorzystujące słownictwo adekwatne do czasów, w których osadzona jest gra,
    \item jak najlepiej oddawające światopogląd w danych czasach,
    \item stosujące oręż typowy dla wybranej epoki.
  \end{itemize}
  \item budowle stylistycznie odpowiadające wybranej epoce,
  \item sposób komunikacji z postaciami imitujący ten stosowany w danych czasach.
\end{itemize}

Dodatkowo po konsultacjach z pomysłodawcą projektu wyniknęło, że prototyp gry nie ma być typową grą z gatunku strategii
czasu rzeczywistego, a jego hybrydą z gatunkiem komputerowych gier fabularnych.

\subsection{Wymagania funkcjonalne}\label{ss:fun}
Niniejsza sekcja skupia się na określeniu wymagań funkcjonalnych, które powinien spełniać prototyp gry.

Gra powinna oferować możliwość:
\begin{itemize}\label{list:fun}
  \item uruchomienia nowej gry,
  \item sterowania postacią gracza,
  \item nawigacji w świecie gry,
  \item wchodzenia w interakcję z postaciami niezależnymi,
  \item przyjmowania zleceń od postaci niezależnych,
  \item najmowania postaci wojowników,
  \item wydawania komend wynajętym postaciom,
  \item zlecania budowy,
  \item zdobywania zasobów,
  \item odczytu wybranego stanu gry z komputera użytkownika,
  \item zapisu aktualnego stanu gry lokalnie na komputerze użytkownika.
\end{itemize}

\subsection{Wymagania pozafunkcjonalne}\label{ss:nonfun}
W tej sekcji zostały przedstawione wymagania pozafunkcjonalne projektu.

Gra powinna umożliwiać:
\begin{itemize}\label{list:nonfun}
  \item rozgrywkę w trybie offline,
  \item działanie na urządzeniach z systemem Windows lub Linux,
  \item dostosowywanie rozmiaru do wielkości ekranu komputera użytkownika,
  \item obsługę klawiatury oraz myszy,
  \item działanie w czasie rzeczywistym.
\end{itemize}

\subsection{Diagram przypadków użycia}\label{ss:usecase}
Niniejsza sekcja przedstawia diagram przypadków użycia dla głównych funkcjonalności, które będzie zawierać prototypowa gra.
Opisuje on przewidywane usługi oferowane przez poszczególne mechaniki programu.

Jedną z głównych akcji, które gra udostępni będzie wydanie rozkazów przyjaznym jednostkom. Polegać będzie ona na poinformowaniu
wojowników przez gracza jaką czynność powinni w danym momencie wykonać. Kolejną możliwością będzie zlecenie budowy, czyli
zlecenie budowniczemu wybudowania wybranego obiektu w określonym przez gracza miejscu. Ponadto użytkownik
będzie mógł przeprowadzać rozmowy z postaciami niezależnymi. Oznacza to, że będzie mógł zainicjować z nimi dialog i
następnie kształtować jego przebieg poprzez wybieranie swojej odpowiedzi z opcji proponowanych przez grę.

\begin{figure}[!htbp]
    \centering
    \includegraphics[width=1.0\textwidth]{images/diagrams/usecase.jpg}
    \caption{Diagram przypadków głównych mechanik gry.}\label{fig:usecases_d}
\end{figure}
\FloatBarrier

\subsection{Diagram stanów}\label{ss:state}
W tym podpunkcie został przedstawiony diagram stanów prototypowej gry, który ukazuje jej przewidywany sposób działania.
Prezentuje on podstawowe stany, w których może się znaleźć system gry.

Do podstawowych stanów należą "Menu główne" oraz "Rozgrywka". Pierwszy z nich oznacza, że program został uruchomiony, a
gracz wyświetla panel główny. Z tego stanu możliwe jest przejście do stanów "Odczytanie stanu gry" bądź "Utworzenie
nowej gry", które to powodują rozpoczęcie gry z zapisu lub od początku.

Stan "Rozgrywka" jest stanem złożonym i określa, że gra została rozpoczęta. W jego skład wchodzą przede wszystkim takie
stany, jak "Interakcja z postacią niezależną", w którym program się znajdzie, gdy gracz rozpocznie dialog z postacią w grze,
czy "Zarządzanie jednostkami", który to oznacza, że użytkownik wydaje rozkazy swoim wojownikom.

Diagram stanów został przedstawiony na rysunku \ref{fig:states_d}.

\begin{figure}[!htbp]
    \centering
    \includegraphics[width=1.0\textwidth]{images/diagrams/state.jpg}
    \caption{Diagram stanów gry.}\label{fig:states_d}
\end{figure}
\FloatBarrier

\subsection{Diagram klas}\label{ss:class}
W tej sekcji został pokazany uproszczony diagram klas (rys. \ref{fig:classes_d}), przedstawiający główne elementy gry.
Obrazuje podstawową strukturę tworzonego systemu oraz zależności pomiędzy poszczególnymi komponentami.

Do najważniejszych klas należą "Interfejs użytkownika", "Mechanizm interakcji z postaciami", "Mechanizm zarządzania
jednostkami" oraz "Mechanizm budowania". Obrazują one podstawowe komponenty gry, których głównym zadaniem jest zarządzanie
poszczególnymi mechanikami. "Interfejs użytkownika" jest odpowiedzialny za interakcję z graczem oraz pomaganie mu w
trakcie rozgrywki. Pozostałe trzy kolejno pozwalają graczowi na prowadzenie dialogów z postaciami
niezależnymi, wydawanie komend jego zaprzyjaźnionym jednostkom oraz budowanie obiektów.
\begin{figure}[!htbp]
    \centering
    \includegraphics[width=1.0\textwidth]{images/diagrams/class.jpg}
    \caption{Diagram klas gry.}\label{fig:classes_d}
\end{figure}
