\subsection{Wymagania dla postaci w grze (Bogna Lew)}
Kluczowym aspektem implementowanej gry jest zachowanie realiów epoki. Jednym z elementów mających na to kluczowy wpływ
są postaci w grze. Ich wygląd i zachowanie wpływa na budowaną fabułę oraz jej odbiór przez gracza. Urozmaicają rozgrywkę
poprzez umożliwienie graczowi wchodzenie z nimi w interakcję, zlecanie mu zadań, czy imitowanie życia codziennego.

Postacie w grze przede wszystkim powinny pasować stylistycznie do realiów fabuły. Oznacza to, że ich ubiór oraz oręż,
którym się posługują, musi jak najbardziej odpowiadać tym obecnym w czasach, w których jest osadzona gra. Jako że
fabuła implementowanej gry jest osadzona we wczesnym średniowieczu, to postaci powinny posługiwać się raczej mieczami i
łukami niż na przykład karabinami. Dodatkowo należy wykorzystać stroje jak najbardziej zbliżone do tych, w które ubierali
się ludzie w tamtych czasach.

Kolejnym istotnym elementem jest oddanie światopoglądu obecnego w tej epoce. Z tego powodu postacie wypowiadając się
powinny stosować słownictwo odpowiednie dla wczesnego średniowiecza oraz przedstawiać świat tak, jak zrobiliby to
ludzie tamtych czasów. Dlatego w swoich wypowiedziach powinni wykorzystywać wierzenia tamtych czasów, na przykład
argumentując wydarzenia, których wtedy nie umieli uzasadnić, jako efekt działań mitycznych stworzeń lub wolę bożą.