\subsection{Wymagania dla mechanizmu budowania (Bogna Lew)}
Jednym z istotnych elementów w grach real-time strategy jest tworzenie baz i budowanie fortyfikacji. Mechanizm ten
stanowi urozmaicenie rozgrywki i wprowadza dodatkowe aspekty możliwe do uwzględnienia w planowaniu strategii. Dla wielu
gier RTS jest wręcz nieodłącznym elementem, który umożliwia graczowi tworzenie i rozwój nowych jednostek, produkcję
zasobów, umacnianie swojej pozycji oraz zwiększanie swojej potęgi.

Rozpatrując ten mechanizm konieczne jest uwzględnienie podstawowych ograniczeń związanych z ukształtowaniem terenu oraz
rozmieszczeniem już istniejących obiektów i elementów scenerii. Czynniki te mają ogromne znaczenie, gdyż zignorowanie
ich może doprowadzić do błędów oraz niepożądanych efektów i w rezultacie do obniżenia jakości rozgrywki. Dodatkowo
kluczową kwestią są zasoby wymagane do wybudowania danego obiektu. Ich istnienie powoduje, że gracz jest w pewien sposób
ograniczony i nie może tworzyć budowli w dowolnej ilości.

Jednym z błędów, którym należy przeciwdziałać jest nakładanie się na siebie obiektów. Taka sytuacja może powstać w
wyniku braku rozpatrywania przez grę położenia elementów terenu i istniejących budowli w trakcie umieszczania na mapie
nowych budynków przez gracza. W efekcie może dojść do sytuacji, że obiekt zostanie wybudowany w miejscu zajmowanym przez
inną część scenerii. Gra powinna przeciwdziałać temu zjawisku, uniemożliwiając graczowi tworzenie nowych budynków w
miejscu, które w nawet najmniejszym stopniu narusza obszar zajęty przez już istniejący element.

Inną sytuacją, której należy unikać jest umożliwienie graczowi wybudowanie budynku w sposób fizycznie niemożliwy.
Przykładem może być ulokowanie obiektu na zbyt stromym zboczu albo na nieodpowiednim gruncie, co w rzeczywistym świecie
byłoby niedopuszczalne. Również w tym przypadku gra nie powinna umożliwić graczowi wykonanie takiej akcji. W przeciwnym
razie obiekt mógłby zostać umieszczony w miejscu do którego postać gracza nie będzie mieć dostępu i nie będzie mógł z
niego korzystać. W efekcie zużyje on bezsensownie swoje zasoby tracąc je bezpowrotnie, co może negatywnie wpłynąć na
jego opinię o grze.