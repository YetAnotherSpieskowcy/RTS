\section{Organizacja (Bartosz Strzelecki)}
\subsection{Główne etapy projektu}
\begin{center}
  \begin{tabular}{| m{30em} | m{12em}|} 
  \hline
  Etap & Termin realizacji \\
  \hline\hline
  Wybór i analiza konkretnego kontekstu historycznego. & Kwiecień 2023 \\
  \hline
  Syntetyczny opis modelu postrzegania przestrzeni na podstawie dzieł pisanych, architektury i sztuki. & Kwiecień — Maj 2023 \\
  \hline
  Przegląd rozwiązań stosowanych w grach strategicznych z wybranego okresu oraz dodatkowo mechanizmów z innych gier, które mogłyby być zaadoptowane na potrzeby projektu. & 2, 3 kwartał 2023 \\
  \hline
  Opracowanie fabuły, selekcja postaci i wydarzeń, a także określenie zakresu autonomii świata gry oraz możliwości modyfikowania go przez gracza. & Czerwiec 2023 \\
  \hline
  Opracowanie szczegółowej koncepcji i projektu gry, w tym projekt mechanizmów zawartych w prototypie. & Lipiec 2023 \\
  \hline
  Implementacja poszczególnych funkcjonalności gry. & 4 kwartał 2023 \\ 
  \hline
  Testowanie, weryfikacja założeń i walidacja. & Listopad — Grudzień 2023 \\
  \hline
  Stworzenie dokumentacji przeprowadzonych prac. & 3, 4 kwartał 2023 \\
  \hline
\end{tabular}
\end{center}
Przewidywany termin zakończenia prac nad projektem to grudzień 2023 roku.
\begin{figure}[htbp]
    \centering
    \includegraphics[width=1\textwidth]{uml/Harmonogram}
    \caption{Harmonogram przedstawiony w postaci diagramy gantt.}
\end{figure}
\section{Skład zespołu projektowego}
\begin{center}
  \begin{tabular}{ c c }
    Bogna Lew & 184757 \\
    Zofia Sosińska & 184896 \\
    Bartosz Strzelecki & 184529
  \end{tabular}
\end{center}
