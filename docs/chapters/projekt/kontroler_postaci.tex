\section{Poruszanie postacią (Bogna Lew)}
Podstawową mechaniką, którą będzie oferować prototyp gry, jest sterowanie postacią przez gracza. Chociaż nie jest
to typowy element gier strategicznych czasu rzeczywistego, to zespół zdecydował się na jego implementację w celu umożliwienia
graczowi na silniejsze zanurzenie się w realia historyczne gry. Użytkownik będzie
mieć do dyspozycji jedną postać, którą będzie bezpośrednio zarządzać. Umożliwi mu ona przede wszystkim eksplorację
świata oraz walkę z przeciwnikami.

Model postaci będzie mieć ubrany pancerz i nie będzie posiadać żadnej broni. Jest to spowodowane faktem, że zgodnie z
zamysłem gry gracz ma być dowódcą, a nie wojownikiem. "Model awatara gracza (czyli postać gracza) może mieć silny wpływ
na jego zachowanie. [...] Jeśli postać gracza ma miecz, gracz oczekuje, że będzie mógł uderzyć nim w coś i stanąć do
walki"\cite{projektowanie_gier}. 

Inspiracją dla mechaniki sterowania postacią jest gra The Elder Scrolls V: Skyrim (por. \ref{s:walka}). Gracz będzie mógł przemieszczać się
do przodu, do tyłu, na boki oraz na ukos. Sterowanie będzie możliwe za pomocą klawiszy  “W”, “A”, “S” oraz “D”.
Dodatkowo użytkownik będzie mieć możliwość obracania postaci, a tym samym zmiany kierunku, w który jest zwrócona poprzez
przemieszczanie myszy. Ponadto gra udostępni możliwość zmiany wysokości i kąta patrzenia kamery.

Kolejnym aspektem jest walka. Użytkownik będzie mógł wykonać atak porzez naciśnięcie prawego przycisku myszy. Postać
gracza będzie mogła wykonywać wyłącznie ataki wręcz. Ma to na celu zmotywowanie go do wynajmowania jednostek posługujących
się bronią, a co za tym idzie - silniejszych od niego.