\section{Sterowanie jednostkami, podążanie za główną postacią (Zofia Sosińska)}\label{chap:sjpzgp}
Na samym początku gry postać gracza pojawia się sama i jest jedynym obiektem, którym gracz może sterować.
Kontroluje, gdzie postać idzie, jak walczy oraz z kim rozmawia. Z biegiem czasu gra będzie jednak naciskać na formowanie drużyny,
ponieważ pokonywanie wielu przeciwników w pojedynkę będzie się stawało zbyt trudne. Pojawia się w takim momencie potrzeba 
zaimplementowania funkcjonalności zarządzania wieloma postaciami.

Stworzenie mechaniki poruszania się i oddziaływania na otaczający świat dla jednej postaci wydaje się proste i intuicyjne, ale kierowanie wieloma osobami już nie. 
Bez wprowadzenia zmian, używając jednego sposobu dyrygowania wszystkimi jednostkami tak samo, gra będzie prędko męczyć gracza.
Dla czynności małoznaczących, takich jak przemieszczenie drużyny w konkretne miejsce, wprowadzi monotonię i czasochłonność.
Każdą postać należy wybrać i przemieścić ją w konkretne miejsce. Kilka kliknięć przy jednej postaci jest akceptowalne, ale przy kilku wprowadzi to ogromne opóźnienia.

Jeszcze gorsze skutki pokazałyby się podczas walki. Szybkie przeskakiwanie pomiędzy postaciami podniosłoby zauważalnie trudność gry.
Poruszanie się jedną postacią i zabijanie przeciwników nie ma sensu, gdy reszta drużyny jest bita i nie może się obronić, ponieważ gracz musi się przełączyć na inną postać,
aby ta wykonała ruch. 

Z tego powodu potrzebny jest algorytm odpowiadający za właściwe poruszanie się pobocznych postaci.
Tworzona gra będzie dopuszczała małe, kilkunastoosobowe drużyny z przywódcą - postacią grywalną przez użytkownika - na czele.
Podczas walki graczowi pokażą się możliwe do wydania polecenia oraz specyfikacja, jakiej grupy mają one dotyczyć.
Za pomocą określonych klawiszy klawiatury będzie on mógł kontrolować zachowanie kompanów.

Po rozwiązaniu problemu mechaniki sterowania jednostkami w walce nie można przeoczyć samego poruszania się oraz interakcji ze światem.
Najprostszym rozwiązaniem będzie implementacja mechanizmu, według którego drużyna, po wykryciu znacznego przemieszczenia się przywódcy, sama będzie za nim podążać.
Kompani nie będą też mieli opcji samodzielnej interakcji ze światem, co sprawi, że poza walką zostaną jedynie biernymi obserwatorami.