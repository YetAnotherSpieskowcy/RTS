\section{System dialogów (Bartosz Strzelecki)}\label{s:dial_proj}

System dialogów jest podstawową metodą, którą gracz będzie wykorzystywał, aby pozyskać informacje  o świecie oraz celach misji.
Gracz może inicjować konwersacje z postaciami niezależnymi, po czym zostaną mu zaproponowane opcje sposobu prowadzenia rozmowy.
W zależności od wybranych opcji dialogowych gracz może się spodziewać różnych konsekwencji.
"Postacie niezależne (NPC) w grach są jednym z najbardziej złożonych i uniwersalnych sposobów pośredniego prowadzenia graczy, który może przybierać różne formy" \cite{projektowanie_gier}.


\href{https://assetstore.unity.com/packages/tools/utilities/dialogue-editor-168329}{Dialogue Editor} autorstwa Grasshop Dev jest prostym narzędziem pozwalającym na szybkie dodawanie i modyfikację dialogów.
Zawiera zestaw elementów ułatwiających wdrożenie systemu do projektu oraz udostępnia struktury danych wykorzystywanych do tworzenia interfejsu użytkownika.
Podczas rozmowy z postaciami niezależnymi gracz będzie mógł pozyskać informację o geografii świata, możliwych zagrożeniach oraz zadaniach do wykonania. 
Podobne systemy występują w grach takich jak Pillars of Eternity oraz w grach z serii \textit{Mass Effect} (por. \ref{chap:dialogi}).

