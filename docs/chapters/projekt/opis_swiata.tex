\section{Opis świata gry i charakteru rozgrywki (Bogna Lew)}\label{s:swiat}
Fabuła wytwarzanej gry ma zostać osadzona w realiach wczesnego średniowiecza i została ona zainspirowana kulturą celtycką.
W tym okresie można było ich spotkać głównie w Irlandii, więc z tego powodu mapa świata gry będzie prezentować górzystą wyspę, na
której gracz będzie mógł znaleźć niewielką wioskę oraz obozowiska.

Graczowi udostępniona zostanie jego własna postać do bezpośredniego sterowania. Elementem gry będą zadania
poboczne, które będą stanowić urozmaicenie rozgryki i dodatkowo zachęcą gracza do
zagłębiania się w nią. Z tego powodu w trakcie gry użytkownik będzie mógł spotkać postaci niezależne, z którymi możliwe
będzie wejście w interakcje, kończące się np. zleceniem wykonania zadania. W ich realizacji będą mu pomagać jednostki,
z którymi się zaprzyjaźni podczas rozgrywki i którym będzie mógł wydawać komendy zgodnie z ich typem. Dodatkowo w
trakcie eksploracji świata natrafi na nieprzyjazne postacie, z którymi będzie toczyć walki. Grę urozmaicą postaci
zwierząt, które mogą być neutralne, bądź agresywne wobec gracza.

Gra będzie udostępniać trzy podstawowe surowce, za które gracz będzie mógł budować budynki. Przewidywane są dwa sposoby
ich pozyskiwania. Pierwszym z nich jest wykonywanie zadań, za które może uzyskać nagrody w postaci pewnej ilości
surowców. Kolejnym sposobem jest zbieranie kłód drewna oraz kamieni leżących na ziemi. Podnoszenie ich dostarczy
jednorazowy przypływ odpowiadającego im zasobu.
