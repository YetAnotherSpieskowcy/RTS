\section{Opis świata gry (Bogna Lew)}
Fabuła wytwarzanej gry ma zostać osadzona w realiach wczesnośredniowiecznych. Została ona zainspirowana Celtami, których
w tym okresie można było spotkać głównie w Irlandii. Z tego powodu mapa świata gry będzie prezentować górzystą wyspę, na
której gracz będzie mógł znaleźć niewielką wioskę oraz obozowiska.

Graczowi udostępniona zostanie jego własna postać do bezpośredniego sterowania. Takie rozwiązanie "czyni gracza
kreatywnym elementem działajacym wewnątrz dyskursu, który posiada przestrzenny charakter"\cite{olbrzymwcieniu}. Typowym
elementem gry są zadania poboczne, czasem pośrednio związane z głównym celem gry. Stanowi to urozmaicenie rozgryki i
dodatkowo zachęca gracza do zagłębiania się w nią. "W odniesieniu do gier komputerowych można więc mówić o podwójnym
motywowaiu ich użytkowników, które dokonuje sie na dwóch narracyjnych poziomach: jedna z motywacji wyznacza cel całej
rozgrywce, druga natomiast jest ulokowana w przestrzeni pojedynczej misji i kończy wraz z jej zakończeniem"\cite{olbrzymwcieniu}.
Z tego powodu trakcie gry użytkownik będzie mógł spotkać postaci niezależne, z którymi możliwe będzie wejście w interakcje, kończące
się np. zleceniem wykonania zadania. W ich realizacji będą mu pomagać jednostki, z którymi się zaprzyjaźni w trakcie
rozgrywki i którym będzie mógł wydawać komendy zgodnie z ich typem. Dodatkowo w trakcie eksploracji świata natrafi na
nieprzyjazne postacie, z którymi będzie toczyć walki. Grę urozmaicą postaci zwierząt, które mogą być neutralne, bądź
agresywne wobec gracza.

Gra będzie udostępniać trzy podstawowe surowce, za które gracz będzie mógł budować budynki. Przewidywane są dwa sposoby
ich pozyskiwania. Pierwszym z nich jest wykonywanie zadań, za które może uzyskać nagrody w postaci pewnej ilości
surowców. Kolejnym sposobem jest zbieranie kłód drewna oraz kamieni leżących na ziemi. Podnoszenie ich dostarczy
jednorazowy przypływ odpowiadającego im zasobu.
