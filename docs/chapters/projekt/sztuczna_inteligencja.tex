\section{Sztuczna inteligencja (Bartosz Strzelecki)}\label{s:ai_proj}
W przypadku implementacji mechanizmów sztucznej inteligencji przeciwników będziemy się inspirować trybami kampanii w grach \textit{Warcraft III} oraz \textit{Starcraft II} (por. \ref{s:ai_wpr}).
Na mapie będą rozsiane punkty, w których będą pojawiać się przeciwnicy. 
Tak długo, jak drużyna gracza będzie się znajdować poza zasięgiem, wrogowie pozostaną nieaktywni.
Aktywni przeciwnicy będą się zachowywać zgodnie z ich archetypem (klasą postaci) oraz pozostaną aktywni tak długo, jak drużyna gracza będzie w zasięgu.
Zdezaktywowani przeciwnicy będą powracać do swojego oryginalnego stanu. Gracz będzie napotykał tego typu obozowiska przede wszystkim w trakcie eksploracji świata.

Będziemy wyróżniać trzy archetypy jednostek w zależności od sposobu walki (bliski zasięg, średni zasięg, daleki zasięg).
Postacie walczące w bliskim zasięgu będą mieć na celu podejście w stronę najbliższego przeciwnika, aby wykonać atak.
Jednostki średnio zasięgowe w momencie, w którym najbliższy przeciwnik jest odpowiednio daleko, będą wykonywać atak dystansowy,
w przeciwnym wypadku będą się zachowywać tak jak jednostki walczące w zwarciu.
Postacie dalekodystansowe będą wykonywać wyłącznie ataki dystansowych w kierunku najbliższego przeciwnika oraz uciekać,
gdy ten podejdzie zbyt blisko.


