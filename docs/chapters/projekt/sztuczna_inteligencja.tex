\section{Sztuczna inteligencja (Bartosz Strzelecki)}
W przypadku implementacji mechanizmów sztucznej inteligencji przeciwników będziemy się inspirować trybami kampanii w grach Warcraft III oraz Starcraft II (por. \ref{s:ai_wpr}). 
Na mapie będą rozsiane punkty, w których będą pojawiać się przeciwnicy. 
Tak długo, jak drużyna gracza jest poza zasięgiem, wrogowie pozostają nieaktywni. 
Aktywni przeciwnicy zachowują się zgodnie z ich archetypem (klasą postaci) oraz pozostają aktywni tak długo, jak drużyna gracza jest w zasięgu. 
Zdezaktywowani przeciwnicy wracają do swojego oryginalnego stanu. Gracz będzie napotykał tego typu obozowiska przede wszystkim w trakcie eksploracji świata. 
Innym planowanym przykładem implementacji sztucznej inteligencji przeciwników jest model, w którym jednostki wroga poruszają się z punktu początkowego w stronę bazy gracza. 
Jeśli podczas swojej podróży napotkają drużynę gracza, wtedy niezwłocznie zmieniają swój cel ataku.
Będziemy wyróżniać trzy archetypy jednostek w zależności od sposobu walki (bliski zasięg, średni zasięg, daleki zasięg). 
Postacie walczące na bliski zasięg mają na celu podejście w stronę najbliższego przeciwnika i wykonać atak. 
Jednostki średnio zasięgowe w momencie, w którym najbliższy przeciwnik jest odpowiednio daleko, wykonują atak dystansowy, 
w przeciwnym wypadku zachowują się tak jak jednostki walczące w zwarciu. 
Postacie  dalekodystansowe dokonują ataków dystansowych w kierunku najbliższego przeciwnika, natomiast uciekają, gdy ten podejdzie zbyt blisko.


