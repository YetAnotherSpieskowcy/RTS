\section{Mechanizm budowania (Bogna Lew)}\label{s:build_proj}
Inspiracją do implementacji tego mechanizmu jest gra \textit{Warhammer 40,000: Dawn of War} (por. \ref{s:budowanie}). Do pożądanych
efektów, które ten tytuł zapewnia, należą walidacja terenu oraz wymuszanie poniesienia kosztów budowy. Dodatkowo
wytwarzana gra będzie implementować proces budowania analogicznie jak w \textit{Warhammer 40,000: Dawn of War}.

Najważniejszym aspektem implementowanego mechanizmu będzie walidacja terenu. Zostanie ona uzyskana poprzez wyświetlenie
podglądu budowli w trakcie umiejscawiania konstrukcji na mapie. Jeśli miejsce, w którym gracz chce postawić budynek jest
poprawne tzn. nie nachodzi na inne obiekty oraz teren jest odpowiedni, to pokazywany widok jest podświetlany na zielono,
w przeciwnym razie - na czerwono. Jest to efekt, który wytwarzana przez zespół gra będzie zawierać.

Kolejnym pożądanym efektem jest konieczność poniesienia przez użytkownika kosztów
wybudowania obiektu. W tym celu gra będzie monitorowała, czy gracz posiada wystarczającą ilość wymaganych zasobów, a w
przypadku niespełnienia tych warunków - blokowała możliwość budowy wybranego obiektu.

Dodatkowo budowa obiektu nie może być natychmiastowa, gdyż nie jest to zgodnie z rzeczywistością. Dlatego podobnie jak w
grze \textit{Warhammer 40,000: Dawn of War} każdy budynek będzie musiał zostać wybudowany przez dedykowaną do tego postać. Będzie
ona imitowała proces budowania i dopiero gdy ukończy to zadanie, dana budowla zacznie przynosić graczowi korzyści.
