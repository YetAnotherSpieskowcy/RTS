\section{Mechanizm budowania (Bogna Lew)}\label{chap:build}
Inspiracją do implementacji tego mechanizmu są gry Orcs must die! oraz Warhammer 40,000: Dawn of War. Do pożądanych
efektów, które oba tytuły zapewniają są walidacja terenu oraz wymuszanie poniesienia kosztów budowy. Dodatkowo
wytwarzana gra będzie implementować proces budowania oraz przynoszenie korzyści przez budynek analogicznie jak w
Warhammer 40,000: Dawn of War.

Najważniejszym aspektem implementowanego mechanizmu będzie walidacja terenu. Zostanie on uzyskany poprzez wyświetlenie
podglądu budowli w trakcie umiejscawiania go na mapie. Jeśli miejsce w którym gracz chce postawić budynek jest poprawne
tzn. nie nachodzi na inne obiekty oraz teren jest odpowiedni to pokazywany widok jest podświetlany na zielono, w
przeciwnym razie - na czerwono. Jest to efekt, który wytwarzana przez zespół gra będzie zawierać.

Kolejnym pożądanym efektem, oferowanym przez oba tytuły, jest konieczność poniesienia przez użytkownika kosztów
wybudowania obiektu. W tym celu gra będzie monitorowała, czy gracz posiada wystarczającą ilość wymaganych zasobów, a w
przypadku niespełnienia tych warunków - blokowała możliwość wybudowania.

Dodatkowo budowa obiektu nie może być natychmiastowe, gdyż nie jest to zgodnie z rzeczywistością. Dlatego podobnie jak w
grze Warhammer 40,000: Dawn of War każdy budynek będzie musiał zostać wybudowany przez dedykowaną do tego postać. Będzie
ona imitowała proces budowania i dopiero gdy ukończy to zadanie dany budynek zacznie przynosić graczowi korzyści.

Każda z budowli będzie przynosić graczowi pewne korzyści w postaci produkcji zasobów. Dzięki temu użytkownik będzie miał
możliwość łatwego pozyskiwania surowców, które następnie będzie mógł wykorzystać na przykład do budowy innych obiektów.
