Do stworzenia interfejsu użytkownika (UI) za wzorce użyliśmy gier “Mount&Blade”, “Mount&Blade2 Bannerlord”, “Warcraft3”, “The Elder Scrolls V: Skyrim”, “Orcs must die”aby umożliwić graczowi jak najprzyjemniejszą rozrywkę oraz wszystkie potrzebne funkcje.

Pierwszy ekran
Wzorowanie: “Mount&Blade”
Jako pierwszy ekran, który widzi gracz przewidujemy grafikę z możliwością wybrania jednej z opcji z menu głównego.

Poruszanie się

Przewidujemy:
górny pasek z najważniejszymi informacjami:surowce, fundusze, czas, kompas
ikony postaci, na które gracz może się przełączyć;
obszar dla komentarzy
pasek życia.

Inspiracja dla górnego paska z najważniejszymi informacjami:surowce, fundusze, czas oraz ikon postaci, na które gracz może się przełączyć:

źródło: Warcraft 3

Inspiracja dla kompasu:

źródło: The Elder Scrolls V: Skyrim
Inspiracja dla obszaru dla komentarzy oraz pasek życia:

źródło: Mount&Blade2 Bannerlord

Rozmowa 

źródło: Mount&Blade

Walka
Przy walce dostępne materiały zamienią się na pasek pokazujący sumaryczne życie naszej drużyny i drużyny przeciwnej oraz możliwe komendy do wydania.


Inspiracja dla możliwych komend:
źródło: Mount&Blade

Budowa budynków
W tym trybie pokażą nam się dostępne do zbudowania budynki, a po wybraniu pojawią się przed nami. Po zatwierdzeniu budynek zostanie wybudowany.


Inspiracja dla trybu budowania:

źródło: Orcs must die
