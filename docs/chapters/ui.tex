\section{Interfejs Użytkownika. Zofia Sosińska}\label{chap:ui}

Projekt Interfejsu Użytkownika przewiduje trzy trybu: zwykły, budowania oraz walki. Zadaniem każdego z nich będzie odzwierciedlenie aktualnej wiedzy granej postaci. Gracz musi proste widzieć podsumowanie wszystkich najważniejszych informacji.
	Tryb zwykły przewiduje podstawowe funkcje, takie jak pokazanie:
surowce i funduszy,
aktualnego czasu w grze, 
kompas pokazujący także wrogów i ważne lokacje,
ikony postaci, na które gracz może się przełączyć;
Inspiracją dla górnego paska z informacjami  oraz dla ikon dostępnych postaci jest ten użyty w grze Warcraft 3.

\begin{figure}[htbp]
    \centering
    \includegraphics[width=0.9\textwidth]{images/ui/warcraft3.png}
    \caption{Pasek z informacjami oraz dla ikony dostępnych postaci w grze Warcraft 3}\label{fig:Warcraft3}
\end{figure}

W naszej grze skupimy się jednak na tym, aby Interfejs Użytkownika zabierał jak najmniej miejsca. Dlatego też projekt zakłada, że poszczególne obiekty nie będą ze sobą połączone, a jedynie “dryfować” w przestrzeni.
Jako ważny element tej części UI zawarty zostanie kompas, wzorowany na tym z gry The Elder Scrolls V: Skyrim.

\begin{figure}[htbp]
    \centering
    \includegraphics[width=0.9\textwidth]{images/ui/compassSkyrim.png}
    \caption{Kompas z gry Skyrim}\label{fig:Fallout}
\end{figure}


\begin{figure}[htbp]
    \centering
    \includegraphics[width=0.9\textwidth]{images/ui/naszpasek.png}
    \caption{Projekt paska z najważniejszymi informacjami o stanie gry: aktualnym czasie, posiadanych surowcach i funduszach oraz o położeniu i otoczeniu gracza.
    }\label{fig:compass}
\end{figure}
 

 W trybie budowania informacje wcześniej przedstawione zostaną na ekranie. Dodatkowo   pokażą nam się dostępne do zbudowania budynki, a po wybraniu pojawią się przed nami. Po zatwierdzeniu budynek zostanie wybudowany.
	Inspiracją do przedstawienia dostępnych budowli jest rozwiązanie gry Orcs must die!

    \begin{figure}[h!tbp]
        \centering
        \includegraphics[width=0.9\textwidth]{images/ui/buoildingsOrcs.png}
        \caption{Wyświetlenie dostępnych pułapek w Orcs must die!}\label{fig:Orcs}
    \end{figure}


Bliźniaczo do trybu budowania, gdy rozpocznie się walka, podstawowe informacje zostają na ekranie, a dodatkowo gracz dostaje informacje o dostępnych rozkazach do wydania. Nasze rozwiązanie będzie podobne do pomysłu z gry Mount\&Blade.

\begin{figure}[h!tbp]
    \centering
    \includegraphics[width=0.9\textwidth]{images/ui/commandsMountBla.png}
    \caption{Wykaz dostępnych rozkazów z gry Mount\&Blade.}\label{fig:MountnBlade}
    \label{fig:mnb}
\end{figure}
