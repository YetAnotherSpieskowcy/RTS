\section{Warhammer 40,000: Down of War}

Warhammer 40,000: Dawn of War jest grą typu real-time strategy osadzoną w uniwersum gry bitewnej Warhammer 40,000. Udostępnia
ona tryb jednoosobowy oraz wieloosobowy dla maksymalnie sześciu graczy. W pierwszym wariancie gracz wciela się w postać dowódcy
armii Space Marines z Blood Ravens i ma za zadanie zapobiec inwazji Orków. Gra Warhammer 40,000: Dawn of War bardzo szybko
zyskała na popularności i oferowała wszystko, co było potrzebne dla tego gatunku. Z tego powodu warto się jej przyjrzeć,
pomimo faktu, że jej realia znacząco odbiegających od tych, w których zostanie osadzona tworzona przez nas gra.

Kluczowym aspektem tej gry jest sposób, w jaki wydarzenia na mapie są prezentowane graczowi. Nie jest on przypisany do
konkretnej postaci, którą steruje, a wciela się we wszystkowiedzącego dowódcę, który rozwój wydarzeń obserwuje z daleka.
Gracz może swobodnie przeglądać odkryte przez siebie fragmenty terenu oraz zarządzać swoimi jednostkami rozmieszczonymi
w nawet najodleglejszych zakątkach mapy. Może on wydawać rozkazy wszystkim podlegającym mu postaciom, niezależnie gdzie
się znajdują, a o każdym wydarzeniu jest informowany przez grę poprzez komunikat wyświetlany na ekranie.

Warhammer 40,000: Dawn of War wyróżnia model pozyskiwania surowców. W grze dostępne są dwa rodzaje: Energia, która jest
generowana przez dedykowane do tego budowle oraz Rekwizycja, której szybkość wytwarzania jest uzależniona od kontrolowanych
przez gracza punktów strategicznych. Taka mechanika znacznie lepiej wpasowuje się w realia gry oraz wymusza na użytkowniku
przyjęcie agresywniejszej strategii.

Dodatkowo Warhammer 40,000: Dawn of War posiada typowy dla gier real-time strategy mechanizm tworzenia budowli. Gracz ma
do dyspozycji jednostki, którym może zlecić budowę wybranego przez siebie obiektu po poniesieniu kosztów jego utworzenia.
Zanim będzie możliwe rozpoczęcie budowania użytkownik musi wybrać miejsce, w którym budynek powstanie. Robi to, przesuwając
jego podgląd po mapie. W tym czasie gra dokonuje walidacji miejsca i informuje gracza czy wybrany obszar jest poprawny,
odpowiednio podświetlając widok budynku. Wybudowanie obiektu nie jest natychmiastowe, co sprawia, że gra lepiej oddaje
realia, w których jest osadzona.

Podstawą rozgrywki są starcia między oddziałami gracza i jego przeciwnikami. Gra nastawiona jest na małe potyczki mające
miejsce w różnych częściach mapy, często równolegle. Sprawia to, że gracz musi wykazać się umiejętnością obsłużenia wielu
wydarzeń naraz i umieć natychmiast podejmować decyzje. Poszczególne bitwy trwają od kilku do kilkudziesięciu minut, a
fabuła jest budowana przez krótkie sceny wyświetlane między kolejnymi starciami. Dzięki takiej mechanice rozgrywka
posiada wciągającą dynamikę.

