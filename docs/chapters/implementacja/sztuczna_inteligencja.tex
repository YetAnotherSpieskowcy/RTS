\section{Sztuczna inteligencja (Bartosz Strzelecki)}\label{s:ai_impl}
Nawigacja przeciwników została zrealizowana poprzez wbudowany w silnik Unity system \texttt{NavMesh}. Pozwala on na łatwe wyznaczenie powierzchni, po której mogą poruszać
się postacie niekontrolowane przez gracza oraz realizuje zadanie wyznaczania ścieżki dla tych postaci.

\begin{figure}[h]
\centering
\includegraphics[width=1\textwidth]{images/ai}
\caption{Obraz przedstawia zasięgi odpowiednich regionów.}
\label{fig:regions}
\end{figure}
\FloatBarrier

Przeciwnicy są kontrolowani poprzez jeden obiekt przydzielający cele każdemu przypisanemu wrogowi. W normalnym trybie wrogowie poruszają się w sposób losowy
w obrębie wyznaczonej przestrzeni (biały okrąg na rys. \ref{fig:regions}). Kiedy przyjazne jednostki znajdą się w wystarczającej odległości (czerwony okrąg na rys. \ref{fig:regions}), przeciwnicy obiorą sobie za cel jedną z nich.
Po opuszczeniu przez drużynę gracza wyznaczonego obszaru (żółty okrąg na rys. \ref{fig:regions}) wrogowie wracają do poruszania się w sposób losowy w obrębie białego okręgu.


\begin{figure}[h]
\centering
\includegraphics[width=1\textwidth]{uml/ai}
\caption{Diagram przedstawiający przepływ stanów dla sztucznej inteligencji przeciwników.}
\end{figure}
