\section{Kontroler postaci [Bogna Lew]}

W trakcie rozgrywki istotnym aspektem wpływającym na jakość jest mechanizm sterowania postacią. Z tą mechaniką gracz ma
bezpośredni kontakt, ponieważ to właśnie za jej pomocą może eksplorować świat.

Do implementacji tego mechanizmu zainspirowaliśmy się grą Skyrim. Postać jest sterowana za pomocą klawiszy “w”, “a”, “s”
oraz “d”, natomiast jej rotacja oraz obrót kamery jest kontrolowany przez mysz. Całość została podzielona na cztery
części.

Pierwsza z nich odpowiada za przemieszczanie się postaci. Do tego wykorzystuje Input Manager, w którym zostały
zmapowane odpowiednie klawisze dla każdej z osi, wzdłuż której gracz może się przemieszczać. W rezultacie powstały dwa
kierunki - przód/tył oraz prawo/lewo. Naciśnięcie odpowiedniego przycisku skutkuje zwróceniem wartości 1 bądź -1, które
symbolizują zwrot wektora przemieszczenia wzdłuż osi. Na tej podstawie wyznaczane jest faktyczne przesunięcie postaci
względem kierunku, w którym jest zwrócona oraz modyfikowane jest jej położenie. Dodatkowo ten komponent odpowiada za
wyznaczenie prędkości, z jaką gracz się porusza. Domyślnie postać przemieszcza się tempem chodu, jednakże jeśli gracz
przytrzyma lewy klawisz Shift, to zacznie się poruszać biegiem.

Druga część kontrolera odpowiada za ustawienie odpowiedniej animacji. Do tego wykorzystuje ona wartości zwrotu wzdłuż
obu osi oraz prędkość, które są wyznaczane w poprzednim komponencie. Na ich podstawie wyznacza kolejne części nazwy
animacji, po czym, jeśli jest inna niż aktualnie wyświetlana, uruchamia ją.

Kolejny komponent nadzoruje rotację postaci oraz co za tym idzie - kamery. Analogicznie wykorzystywany jest Input
Manager, jednak w tym przypadku przechwytywane jest przesunięcie myszy. Komponent udostępnia graczowi możliwość
sterowania kierunkiem, w którym postać patrzy, a co za tym idzie - względem którego się przemieszcza. Jednocześnie
następuje dostosowanie położenia kamery tak, aby zawsze patrzyła w tym samym kierunku co postać gracza. Ponadto
komponent umożliwia przesunięcie kamery po łuku do góry bądź do dołu. Dzięki temu gracz może dokładniej zobaczyć,
co znajduje się nad oraz tuż przed nim.

Ostatni komponent odpowiada za kontrolowanie przybliżania kamery. W tym celu w Input Managerze zostało zmapowane kółko
myszy. Zwrócona przez system wartość jest wykorzystywana do obliczenia odległości kamery od postaci przy zachowaniu
ustalonych przez grę ograniczeń. Dodatkowo komponent dokonuje również automatycznego przybliżania, jeśli wskutek
przemieszczania się postaci, pomiędzy nią a kamerą miałaby się znaleźć przeszkoda. Dzięki temu nie zostanie zasłonięty
widok tego co się dzieje wokół postaci. Odsunięcie się od przeszkody zaskutkuje oddaleniem kamery do poprzedniej
odległości.
