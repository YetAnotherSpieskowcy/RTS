\section{Interfejs Użytkownika (Zofia Sosińska)}\label{chap:ui_imp}
Interfejs użytkownika, jako uporządkowany i przejrzysty obraz wiedzy i możliwych opcji granej postaci odciąża użytkownika
aplikacji, zdejmując z niego przymus pamiętania dokładnie każdej pojawiającej się informacji. Umililiśmy i uprościliśmy 
rozgrywkę, przedstawiając suche dane w postaci przyjemnych dla oka obrazów, wyszczególniając najważniejsze informacje.

Tak jak przewidywano, UI udostępnia interfes podstawowy z zawsze widocznymi elementami oraz dynamicznie pojawiające okna, wywoływane za pomocą konkretnych klawiszy.
Wszystkie łączy surowy i prosty przewodni motyw graficzny, wykorzystujący także różnorodne obrazy dla urozmaicenia. Przy implementacji ważne także było, aby UI zabierał
jak najmniej miejsca, jednocześnie podając jak najwięcej przydatnych informacji.

\subsection{Interfejs podstawowy}
Interfejs podstawowy towarzyszy graczowi podczas całej rozgrywki. Skupia się on w górnej części ekranu. Jego zadaniem jest pomoc 
użytkownikowi w ogólnym odnalezieniu się w świecie. W tym celu są mu ukazane następujące informacje:
\begin{itemize}
    \item stan surowców i funduszy, aby nie był obarczony kalkulacjami przy każdym zakupie lub przypływie zasobów;
    \item godzina w grze, aby stworzyć iluzję upływającego czasu w świecie gry;
    \item położenie gracza względem stron świata oraz wrogów ukazane na kompasie, aby ułatwić nawigację;
    \item etap, na którym są przypisane graczowi zadania, aby przypomnieć mu, że na takowe się zgodził;
    \item stan zdrowia gracza, aby zasygnalizować mu, czy przypadkiem nie rozsądniejsze będzie wycofanie się z potyczki.
\end{itemize}

\begin{figure}[htbp]
    \centering
    \includegraphics[width=0.9\textwidth]{images/ui/naszpasek.png}
    \caption{Implementacja paska z najważniejszymi informacjami o stanie gry: aktualnym czasie, posiadanych surowcach 
    i funduszach, położeniu i otoczeniu gracza oraz o możliwości rozpoczęciu konwersacji z inną postacią.
    }\label{fig:compass}
\end{figure}


\subsection{Menu stawiania budynków}
Typowa mechanika gier typu RTS, czyli budowanie budynków ma specjalnie przygotowany interfejs, wyświetlany po wciśnięciu klawisza "B". 
Ulokowany on został w dolnej części ekranu. Najważniejszym elementem menu stawiania budynków jest lista dostępnych budowli. Widnieją tam 
obrazy ukazujące każdą z nich, a gracz ma wgląd w ich szczegóły poprzez zmianę aktywnej prawą i lewą strzałką. Wtedy wokół niej pokażą się także informacje dotyczące jej kupna,
a po lewej stronie ekranu - informacja o ewentualnym nieprawidłowym umiejscowieniu. W wypadku, gdy zakup jest niemożliwy z powodu niewystarczającej liczby 
surowców, pojawi się stosowny komunikat.
\begin{figure}[htbp]
    \centering
    \includegraphics[width=0.9\textwidth]{images/ui/opis_ekementow_budowanie.png}
    \caption{Implementacja menu stawiania budynków, na którym pokazane są możliwe do zbudowania budowle 
    i szczegółowe informacje o ich dostępności.
    }\label{fig:compass}
\end{figure}

\subsection{Menu wydawania komend}

\begin{figure}[htbp]
    \centering
    \includegraphics[width=0.9\textwidth]{images/ui/opis_ekementow_mwnu_wyboru_komendy.png}
    \caption{Implementacja menu wydawania komend}\label{fig:c}
\end{figure}
\subsection{Menu zapisu}

\subsection{Notatnik z zadaniami}

\subsection{Informacja o możliwej czynności}

\subsection{Menu końca gry}