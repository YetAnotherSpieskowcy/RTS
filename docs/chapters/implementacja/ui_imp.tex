\section{Interfejs Użytkownika (Zofia Sosińska)}\label{chap:ui_imp}
Interfejs użytkownika, jako uporządkowany i przejrzysty obraz wiedzy i możliwych opcji granej postaci odciąża użytkownika
aplikacji, zdejmując z niego przymus pamiętania dokładnie każdej pojawiającej się informacji. Umililiśmy i uprościliśmy 
rozgrywkę, przedstawiając suche dane w postaci przyjemnych dla oka obrazów, wyszczególniając to, co jest najważniejsze.

Tak jak przewidywano, UI udostępnia interfejs podstawowy z zawsze widocznymi elementami oraz dynamicznie pojawiające okna, wywoływane za pomocą konkretnych klawiszy.
Wszystkie łączy surowy i prosty przewodni motyw graficzny, wykorzystujący także różnorodne obrazy dla urozmaicenia. Przy implementacji ważne także było, aby UI zabierał
jak najmniej miejsca, jednocześnie podając jak najwięcej przydatnych informacji.

\subsection{Interfejs podstawowy}
Interfejs podstawowy towarzyszy graczowi podczas całej rozgrywki. Skupia się on w górnej części ekranu. Jego zadaniem jest pomoc 
użytkownikowi w ogólnym odnalezieniu się w świecie. W tym celu są mu ukazane następujące informacje:
\begin{itemize}
    \item aktualny czas w grze;
    \item położenie gracza względem stron świata oraz wrogów ukazane na kompasie;
    \item stan surowców i funduszy;
    \item stan zdrowia gracza;
    \item etap, na którym są przypisane graczowi zadania.
\end{itemize}

\begin{figure}[htbp]
    \centering
    \includegraphics[width=0.9\textwidth]{images/ui/naszpasek.png}
    \caption{Implementacja paska z najważniejszymi informacjami o stanie gry: aktualnym czasie, posiadanych surowcach 
    i funduszach, położeniu i otoczeniu gracza, zdrowiu postaci oraz ikona sygnalizująca, czy użytkownik ma do wykonania jakieś zadanie.
    }\label{fig:naszpasek}
\end{figure}

W tych statycznie umiejscowionych elementach interfejsu dynamicznie zmienia się ich treść. Na zegarze czas zmienia się w ustalony sposób,
dodając minutę w grze co 5 sekund upływające w realnym świecie. Na kompasie umiejscowienie zmieniają symbole stron świata i wskaźniki przeciwników,
zależnie od ich pozycji, położenia gracza i strony, w którą główna posatć patrzy. Fundusze rosną po wykonaniu zadania i otrzymaniu zapłaty oraz maleją, gdy opłacamy
najemników, czy płacimy za budowle. Liczba posiadanych surowców zwiększa się, po zebraniu ich z ziemi, za to zmniejsza, gdy za ich pomocą budujemy budynek.  
Pasek zdrowia zmienia swoją wartość, gdy dostaniemy obrażenia, jednocześnie ukazując początkową, maksymalną wartość. Ikona zadania do wykonania ukazuje symbol 
czerwonego wykrzyknika, gdy gracz ma jakieś aktywne, nieskończone zadanie, co ukazano na rysunku \ref{fig:wyq}.
\begin{figure}[htbp]
    \centering
    \includegraphics[width=0.5\textwidth]{images/ui/wykrzyknik_quest.png}
    \caption{Implementacja ikony zadania do wykonania, gdy użytkownik ma do wykonania jakieś zadanie.
    }\label{fig:wyq}
\end{figure}

\subsection{Menu stawiania budynków}
Typowa mechanika gier typu RTS, czyli budowanie budynków, ma specjalnie przygotowany interfejs, wyświetlany po wciśnięciu klawisza "B". 
Ulokowany on został w dolnej części ekranu. Najważniejszym elementem menu stawiania budynków jest lista dostępnych budowli. Widnieją tam 
obrazy ukazujące każdą z nich, a gracz ma wgląd w ich szczegóły poprzez zmianę aktywnej prawą i lewą strzałką. Wtedy wokół niej pokazują się także informacje dotyczące jej kupna,
a po lewej stronie ekranu - informacja o ewentualnym nieprawidłowym umiejscowieniu. W wypadku, gdy zakup jest niemożliwy z powodu niewystarczającej liczby 
surowców, pojawi się stosowny komunikat. Menu zamyka się za pomocą klawisza "Escape".
\begin{figure}[htbp]
    \centering
    \includegraphics[width=0.9\textwidth]{images/ui/opis_ekementow_budowanie.png}
    \caption{Implementacja menu stawiania budynków, na którym pokazane są możliwe do zbudowania budowle 
    i szczegółowe informacje o ich dostępności.
    }\label{fig:compass}
\end{figure}

\subsection{Menu wydawania komend}
Gra umożliwia graczowi wykupienie usług najemników, jeśli dysponuje odpowiednimi funduszami. Gdy są oni już pod dowództwem użytkownika, może on im rozkazywać za pomocą 
menu wydawania komend. Po naciśnięciu klawisza "Q" pokazuje się lista dostępnych grup podwładnych, podzielonych według ich specjalizacji. Po wybraniu jednej z nich 
wylistowane zostaną możliwe do rozkazania czynności. W obu przypadkach gracz ma także możliwość anulowania. 
\begin{figure}[htbp]
    \centering
    \includegraphics[width=0.9\textwidth]{images/ui/opis_ekementow_mwnu_wyboru_komendy.png}
    \caption{Implementacja menu wydawania komend.}\label{fig:cmd_menu}
\end{figure}

\subsection{Menu zapisu}
Po uznaniu przez gracza, że nie chce stracić aktualnego stanu gry, może go permamentnie zapisać w pamięci urządzenia, na którym włączony jest program. Po naciśnięciu 
klawisza "Z" pojawia sie menu zapisu. Gracz może odczytać z niego nazwę pliku, w którym zapisany zostanie stan gry. W menu znajduje się także informacja o tym, że 
poprzez ponowne naciśnięcie klawisza "Z" potwierdzi on zapis. 
\begin{figure}[htbp]
    \centering
    \includegraphics[width=0.9\textwidth]{images/ui/menu_zapisu.png}
    \caption{Implementacja menu zapisu.}\label{fig:men_zap}
\end{figure}

\subsection{Informacja o możliwej interakcji}
W świecie gry postać gracza nie jest odosobniona, co więcej może spotkać wiele różnorodnych osób, z którymi można porozmaiwać. Interaktywni
jednak nie są tylko ludzie, ale i surowce, które można zbierać. Jeśli możliwa jest interakcja,  wyświetlana jest informacja o takim stanie
 rzeczy. Zaraz pod kompasem pojawia się grafika instruująca, że po naciśnięciu klawisza "E" klawiatury, postać gracza wykona opisaną czynność.
 \begin{figure}[htbp]
    \centering
    \includegraphics[width=0.9\textwidth]{images/ui/interakcja_rozmowa.png}
    \caption{Implementacja grafiki informującej o możliwym zaczęciu rozmowy.}\label{fig:rozmow}
\end{figure}
\begin{figure}[htbp]
    \centering
    \includegraphics[width=0.9\textwidth]{images/ui/interakcja_podnieś.png}
    \caption{Implementacja grafiki informującej o możliwości podniesienia przedmiotu.}\label{fig:przedmio}
\end{figure}

\subsection{Dziennik z zadaniami}
Niektóre postacie interaktywne mogą zlecić głównemu bohaterowi zadanie. Jeśli gracz się zgodzi, notatka o nim zostaje zapisana w dzienniku. Można do niego zajrzeć
wciskając klawisz "J". Graczowi ukaże się tytuł sygnalizujący główny cel zadania, streszczenie najważniejszych informacji oraz porada dotycząca mechanik gry np. 
"Wciśnij jakiś klawisz, żeby stało się coś". To, że zadanie nie zostało jeszcze wykonane, sygnalizuje czerwony wykrzyknik w prawym, górnym rogu notatki. Gracz może 
zamknąć dziennik poprzez ponowne naciśnięcie klawisza "J".
\begin{figure}[htbp]
    \centering
    \includegraphics[width=0.9\textwidth]{images/ui/journal_quest.png}
    \caption{Implementacja dziennika z aktualnie zaczętym i nieukończonym zadaniem.}\label{fig:end_sc}
\end{figure}

\subsection{Ekran końca gry}
Postać gracza ma określoną liczbę punktów życia, które może stracić podczas potyczek z przeciwnikami. Gdy spadną one do zera gra kończy się, a główny bohater umiera.
W takim wypadku graczowi pojawia się ekran końca gry, który informuje go o śmierci oraz o możliwości przejścia do menu głównego. Aby jeszcze bardziej uwypuklić zakończenie 
rozgrywki, cały ekran zostaje przyciemniony.
\begin{figure}[htbp]
    \centering
    \includegraphics[width=0.9\textwidth]{images/ui/endgame_screen.png}
    \caption{Implementacja menu zapisu.}\label{fig:end_sc}
\end{figure}