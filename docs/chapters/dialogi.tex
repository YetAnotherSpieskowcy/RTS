\section{System dialogów. Bartosz Strzelecki}

System dialogów jest podstawową metodą, którą gracz będzie wykorzystywał, aby pozyskać informacje  o świecie oraz celach misji.
Gracz może inicjować konwersacje z postaciami niezależnymi, po czym zostaną mu zaproponowane opcje sposobu prowadzenia rozmowy.
W zależności od wybranych opcji dialogowych gracz może się spodziewać różnych konsekwencji.


\begin{figure}[h]
\centering
\includegraphics[width=0.6\textwidth]{images/fallout3}
\caption{Kadr z gry Fallout 3 przedstawiający przykładowy dialog}
\end{figure}

\href{https://assetstore.unity.com/packages/tools/utilities/dialogue-editor-168329}{Dialogue Editor} autorstwa Grasshop Dev jest prostym narzędziem pozwalającym na szybkie dodawanie i modyfikację dialogów.
Zawiera zestaw elementów ułatwiających wdrożenie systemu do projektu oraz udostępnia struktury danych wykorzystywanych do tworzenia interfejsu użytkownika.
Podczas rozmowy z postaciami niezależnymi gracz będzie mógł pozyskać informację o geografii świata, możliwych zagrożeniach oraz zadaniach do wykonania. 
Podobne systemy występują w grach takich jak Pillars of Eternity oraz w grach z serii Mass Effect.

W powyższej implementacji konwersację składają się z dwóch rodzajów węzłów. Jednego odpowiedzialnego za wypowiedzi postaci niezależnych oraz drugiego
pozwalającego na podjęcie przez gracza decyzji.
Dodanie nowej konwersacji odbywa się poprzez stworzenie obiektu z przypisanym komponentem NPC Conversation.
Wywołanie dialogu można osiągnąć poprzez wykorzystanie metody


\begin{verbatim}
ConversationManager.Instance.StartConversation(/*NPCConversation*/);
\end{verbatim}

\begin{figure}[h]
\centering
\includegraphics[width=0.6\textwidth]{images/dial}
\caption{Przykładowa konwersacja z wykorzystaniem Dialogue Editor}
\end{figure}
System pozwała na łatwe dostosowanie elementów interfejsu użytkownika odpowiedzialne za wyświetlenie dialogów, aby jak najlepiej wpasować się
w styl graficzny gry.

\begin{figure}[h]
\centering
\includegraphics[width=0.6\textwidth]{images/d}
\caption{Przykładowe okno dialogowe widziane z perspektywy gracza.}
\end{figure}
