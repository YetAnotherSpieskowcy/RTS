\section{Dalsze możliwości rozwoju (Bartosz Strzelecki, Zofia Sosińska)}\label{s:dalsze}
Powyższy projekt jest jedynie prototypem zawierającym podstawowe mechaniki.
Dalszy rozwój pracy głównie będzie polegać na konsolidacji zaimplementowanych
systemów w gotowy produkt.

Głównym zadaniem byłoby opracowanie i realizacja rozwiniętej linii fabularnej,
wykorzystując gotowe elementy wykonane w ramach projektu. Istotnym aspektem
byłaby wymiana modeli i tekstur z tymczasowych, prototypowych zasobów, służących
do celów demonstracyjnych zaimplementowanych mechanik na wyższej jakości
i lepiej odzwierciedlające klimat rozgrywki. Gry nie różnią się w tym wypadku od
innych produktów dostępnych na rynku. To, jak dobrze program wygląda tak samo wpłynie na
pozytywny odbiór przez użytkownika, jak chociażby wykonanie i jakość materiałów, z których są wykonane elementy
gier planszowych. "Estetyczne odczucia z grania w [...] gry ma znaczenie. Kiedy podnosisz pieczołowicie wyrzeźbiony pionek, reagujesz na niego
doceniając aspekt estetyczy - jedną z form przyjemności." \cite{theory_of_fun}.

Kolejnym elementem byłoby skomponowanie muzyki dobrze oddającej atmosferę wczesnego
średniowiecza oraz przygotowanie podkładów dźwiękowych, które zostałyby wykorzystane
jako efekty dźwiękowe na przykład podczas wykonywanych animacji ataku, podnoszenia przedmiotów itd.

Dalszy plan rozwoju skupiałby się głównie na aspektach audiowizualnych.
Powyższe aspekty pozytywnie wpłynęłyby na odbiór produktu przez potencjalnych przyszłych
użytkowników. 
