\chapter*{Streszczenie}
\addcontentsline{toc}{chapter}{Streszczenie}
Tematem pracy inżynierskiej jest realizacja gry z gatunku strategii czasu rzeczywistego osadzonej w realiach historycznych
wybranej epoki. Niniejsza praca zawiera opis przykładowych rozwiązań zastosowanych w niektórych grach dostępnych na rynku
oraz projekt i implementację wytwarzanego programu. Praca skupia się na przedstawieniu zaimplementowanych mechanik oraz
sposobie oddania realiów, w których osadzona jest fabuła opracowanej gry.

Do przygotowania prototypu gry wykorzystano silnik Unity oraz udostępniane przez niego narzędzia. Za ich pomocą
przygotowano podstawowe mechanizmy, typowe dla gier strategii czasu rzeczywistego oraz przykładowe zadania tworzące
fabułę gry. Utworzony prototyp pozwala na dalsze rozwijanie świata gry poprzez dodawanie nowych zadań oraz elementów
rozgrywki. Zagadnienia implementacyjne zostały opisane w rozdziale Implementacja, a efekt finalny w Przebiegu rozgrywki.

Podstawowym celem projektu jest przygotowanie gry, która w jak najdokładniejszy sposób oddawałaby realia wybranej epoki.
Zdecydowaliśmy się na osadzenie fabuły we wczesnym średniowieczu. Główną inspiracją stali się Celtowie, którzy w
tamtych czasach zamieszkiwali tereny współczesnej Irlandii. W ramach projektu przede wszystkim skupiono się na
opracowaniu sposobu nawigacji w grze oraz zachowania przeciwników, stosujących broń oraz słownictwo charakterystyczne
dla realiów historycznych. W rozdziale Projekt została przedstawiona wizja autorów na poszczególne elementy gry.

W celu przygotowania się do projektu i implementacji gry, przeprowadzono przegląd wybranych mechanik w grach dostępnych
na rynku. Niniejsza praca zawiera opis ich działania na podstawie przykładów z istniejących gier oraz sposobu, w jaki
wpływają na ich rozgrywkę. Przedstawione rozwiązania stanowią inspirację dla wytwarzanej gry.

Praca została przygotowana przez trzech współautorów. Bartosz Strzelecki był odpowiedzialny za punkty \ref{ss:rts},
\ref{s:ai_wpr}, \ref{chap:dialogi}, \ref{chap:dbd}, \ref{s:cel}, \ref{ss:navmesh}, \ref{ss:protobuf}, \ref{s:org},
\ref{s:dial_proj}, \ref{s:ai_proj}, \ref{s:wid_proj}, \ref{s:dia_impl}, \ref{s:ai_impl}, \ref{s:wid_impl} i
\ref{s:save_impl}. Jest również współautorem podpunktów \ref{ss:comp}, \ref{s:rozgrywka} oraz \ref{s:dalsze}. Bogna Lew
odpowiadała za punkty \ref{ss:rpg}, \ref{s:budowanie}, \ref{s:walka}, \ref{s:fabula}, \ref{s:silniki}, \ref{s:silniki},
\ref{ss:tTool}, \ref{ss:so}, \ref{s:swiat}, \ref{s:por_proj}, \ref{s:build_proj}, \ref{s:por_impl}, \ref{s:bud_impl} i
\ref{s:efekty} oraz współtworzyła \ref{chap:introduction}, \ref{ss:comp}, \ref{s:wymagania} oraz \ref{s:rozgrywka}.
Zofia Sosińska była odpowiedzialna za pracę nad punktami \ref{ss:tbs}, \ref{c:elem_ui}, \ref{chap:mb},
\ref{chap:menu_main}, \ref{chap:ui}, \ref{chap:naw}, \ref{chap:sjpzgp}, \ref{chap:menu_main_impl}, \ref{chap:ui_imp},
\ref{chap:naw_impl} oraz \ref{s:opinia}. Współtworzyła również punkty \ref{chap:introduction}, \ref{s:wymagania} i \ref{s:dalsze}.

\chapter*{Abstract}
\addcontentsline{toc}{chapter}{Abstract}  
The topic of the enginnering thesis is a prototype of real-time strategy game set in historical reality of selected era.
This work contains a description of sample solutions used in selected games available on the market
and the design and a overview of the implementation of the produced program. The work focuses on the presentation 
of the implemented mechanics as well as
the way of rendering the realities in which the plot of the developed game is set.

The Unity engine and the tools it provides were used to prepare a prototype of the game.
Employing these tools, basic mechanics typical of real-time strategy games, were prepared,
as well as sample tasks forming the games's plot. The devised prototype allows for 
further development of the game environment by adding new activities and gameplay elements.
The implementation matters are described in the Implementation chapter, and the results in the Gameplay Course.

The fundamental goal of the project is to develop a game that would reflect the reality of the selected era as accurately as possible.
The main inspiration came from the Celts, who at that time inhabited the territory of modern day Ireland. 
The project primarily focused on
developing the means of navigation in the game and the behavior of opponents, using weaponry and vocabulary characteristic to the
historical realities. The Project chapter presents the authors' vision for the various elements of the game.

In preparation for the design and implementation of the game, a review of selected mechanics in games available
on the market were conducted. This paper describes how they work using examples from existing games and how
they affect their gameplay. The solutions presented provide inspiration for the developed game.

Diploma was prepared by three coauthors. Bartosz Strzelecki was responsible for sections \ref{ss:rts},
\ref{s:ai_wpr}, \ref{chap:dialogi}, \ref{chap:dbd}, \ref{s:cel}, \ref{ss:navmesh}, \ref{ss:protobuf}, \ref{s:org},
\ref{s:dial_proj}, \ref{s:ai_proj}, \ref{s:wid_proj}, \ref{s:dia_impl}, \ref{s:ai_impl}, \ref{s:wid_impl} and
\ref{s:save_impl}. He also coauthored \ref{ss:comp}, \ref{s:rozgrywka} and \ref{s:dalsze}. Bogna Lew
was responsible for \ref{ss:rpg}, \ref{s:budowanie}, \ref{s:walka}, \ref{s:fabula}, \ref{s:silniki}, \ref{s:silniki},
\ref{ss:tTool}, \ref{ss:so}, \ref{s:swiat}, \ref{s:por_proj}, \ref{s:build_proj}, \ref{s:por_impl}, \ref{s:bud_impl} and
\ref{s:efekty} and coauthored \ref{chap:introduction}, \ref{ss:comp}, \ref{s:wymagania} and \ref{s:rozgrywka}.
Zofia Sosińska was responsible for preparing sections \ref{ss:tbs}, \ref{c:elem_ui}, \ref{chap:mb},
\ref{chap:menu_main}, \ref{chap:ui}, \ref{chap:naw}, \ref{chap:sjpzgp}, \ref{chap:menu_main_impl}, \ref{chap:ui_imp},
\ref{chap:naw_impl} and  \ref{s:opinia}. She also coauthored sections \ref{chap:introduction}, \ref{s:wymagania} and \ref{s:dalsze}.
 
