\chapter*{Streszczenie}
\addcontentsline{toc}{chapter}{Streszczenie}
Tematem pracy inżynierskiej jest realizacja gry strategii czasu rzeczywistego osadzonej w realiach historycznych
wybranej epoki. Niniejsza praca zawiera opis przykładowych rozwiązań zastosowanych w niektórych grach dostępnych na rynku
oraz projekt i implementację wytwarzanego programu. Praca skupia się na przedstawieniu zaimplementowanych mechanik oraz
sposobie oddania realiów, w których osadzona jest fabuła opracowanej gry.

Do przygotowania prototypu gry wykorzystano silnik Unity oraz udostępniane przez niego narzędzia. Za ich pomocą
przygotowano podstawowe mechanizmy, typowe dla gier strategii czasu rzeczywistego oraz przykładowe zadania tworzące
fabułę gry. Utworzony prototyp pozwala na dalsze rozwijanie świata gry poprzez dodawanie nowych zadań oraz elementów
rozgrywki. Zagadnienia implementacyjne zostały opisane w rozdziale Implementacja, a efekt finalny w Przebiegu rozgrywki.

Podstawowym celem projektu jest przygotowanie gry, która w jak najdokładniejszy sposób oddawałaby realia wybranej epoki.
Zdecydowaliśmy się na osadzenie fabuły we wczesnym średniowieczu. Główną inspiracją stali się Celtowie, którzy w
tamtych czasach zamieszkiwali tereny współczesnej Irlandii. W ramach projekty przede wszystkim skupiono się na
opracowaniu sposobu nawigacji w grze oraz zachowania przeciwników, stosujących broń oraz słownictwo charakterystyczne
dla realiów historycznych. W rozdziale Projekt została przedstawiona wizja autorów na poszczególne elementy gry.

W celu przygotowania się do projektu i implementacji gry, przeprowadzono przegląd wybranych mechanik w grach dostępnych
na rynku. Niniejsza praca zawiera opis ich działania na podstawie przykładów z istniejących gier oraz sposobu, w jaki
wpływają na ich rozgrywkę. Przedstawione rozwiązania stanowią inspirację dla wytwarzanej gry.


\chapter*{Abstract}
\addcontentsline{toc}{chapter}{Abstract}  
