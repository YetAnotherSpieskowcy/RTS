\chapter*{Streszczenie}
% \addcontentsline{toc}{chapter}{Streszczenie}
Celem pracy inżynierskiej był projekt i realizacja prototypowej gry będącej hybrydą gatunków strategii czasu rzeczywistego oraz
komputerowych gier fabularnych, osadzonej w realiach historycznych
wybranej epoki. Niniejsza praca zawiera opis przykładowych rozwiązań zastosowanych w wybranych grach dostępnych na rynku
oraz projekt i implementację wytwarzanego programu. Praca skupia się na przedstawieniu zaimplementowanych mechanik oraz
sposobie oddania realiów, w których osadzona jest fabuła opracowanej gry.

Do przygotowania prototypu gry wykorzystano silnik Unity oraz udostępniane przez niego narzędzia. Za ich pomocą
przygotowano podstawowe mechanizmy, typowe dla gier strategii czasu rzeczywistego i komputerowych gier fabularnych oraz przykładowe zadania tworzące
fabułę gry. Utworzony prototyp pozwala na dalsze rozwijanie świata gry poprzez dodawanie nowych zadań oraz elementów
rozgrywki.

W celu przygotowania się do pracy nad prototypem gry, przeprowadzono przegląd wybranych gatunków gier komputerowych oraz
mechanik stosowanych w grach. Niniejsza praca zawiera opis działania poszczególnych mechanizmów na podstawie przykładów
z gier dostępnych na rynku oraz przedstawienie ich wpływu na rozgrywkę. Przedstawione rozwiązania stanowią inspirację dla wytwarzanej gry.

W ramach projektu opracowano prototyp gry osadzonej w realiach historycznych wczesnego średniowiecza. W grze zostały
zawarte mechanizmy zarządzania przyjaznymi jednostkami, budowania budowli oraz sztuczna inteligencja przeciwników, czyli
mechaniki typowe dla gier RTS. Dodatkowo gra udostępnia prosty kontroler postaci, system rozwoju postaci oraz mechanizm
interakcji z postaciami niezależnymi, które to są charakterystyczne dla komputerowych gier fabularnych.

Praca została zrealizowana przez trzech współautorów. Bartosz Strzelecki był odpowiedzialny za przygotowanie systemu
dialogów, sztucznej inteligencji postaci niezależnych, mechanizmu widzenia przez przeszkody oraz zapisu i odczytu stanu gry.
Jest autorem podrozdziałów: \ref{ss:rts},
\ref{s:ai_wpr}, \ref{chap:dialogi}, \ref{chap:dbd}, \ref{s:cel}, \ref{ss:navmesh}, \ref{ss:protobuf}, \ref{s:org}, \ref{s:proj_progres}, \ref{s:wid_proj},
\ref{s:dial_proj}, \ref{s:ai_proj}, \ref{s:impl_progres}, \ref{s:wid_impl}, \ref{s:dia_impl}, \ref{s:ai_impl}, i
\ref{s:save_impl}. Jest również współautorem podpunktów \ref{ss:comp}, \ref{s:wpr_progres}, \ref{s:rozgrywka} oraz \ref{s:dalsze}.
Bogna Lew odpowiadała za modelowanie terenu gry oraz implementację mechanizmu budowania i kontrolera postaci. Jest autorką
podrozdziałów \ref{ss:rpg}, \ref{s:budowanie}, \ref{s:fabula}, \ref{s:walka}, \ref{s:silniki},
\ref{ss:tTool}, \ref{ss:so}, \ref{s:wymagania}, \ref{s:swiat}, \ref{s:por_proj}, \ref{s:build_proj}, \ref{s:por_impl}, \ref{s:bud_impl} i
\ref{s:efekty} oraz współtworzyła \ref{chap:introduction}, \ref{ss:comp}, \ref{s:wpr_progres}, oraz \ref{s:rozgrywka}.
Zofia Sosińska była odpowiedzialna za przygotowanie interfejsu użytkownika wraz z mechanizmem przetwarzania interakcji z użytkownikiem. Jest
autorką podrozdziałów \ref{ss:tbs}, \ref{c:elem_ui}, \ref{chap:mb}, \ref{chap:ui},
\ref{chap:menu_main}, \ref{chap:naw}, \ref{chap:sjpzgp}, \ref{chap:ui_imp}, \ref{chap:menu_main_impl},
\ref{chap:naw_impl} oraz \ref{s:opinia}. Współtworzyła również punkty \ref{chap:introduction} i \ref{s:dalsze}.

\chapter*{Abstract}
% \addcontentsline{toc}{chapter}{Abstract}
The purpose of the enginnering thesis was a project and a prototype of real-time strategy game set in historical reality of selected era.
This work contains a description of sample solutions used in selected games available on the market
and the design and a overview of the implementation of the produced program. The work focuses on the presentation 
of the implemented mechanics as well as
the way of rendering the realities in which the plot of the developed game is set.

The Unity engine and the tools it provides were used to prepare a prototype of the game.
Employing these tools, basic mechanics typical of real-time strategy games and computer role-playing games were prepared,
as well as sample tasks forming the games's plot. The devised prototype allows for 
further development of the game environment by adding new activities and gameplay elements.

In preparation for the design and implementation of the game, a review of selected mechanics in games available
on the market were conducted. This paper describes how they work using examples from existing games and how
they affect their gameplay. The solutions presented provide inspiration for the developed game.

Within the project, a prototype game set in the historical realities of the early Middle Ages was developed. The game
included mechanics of managing friendly units, building objects and artificial intelligence of the opponents, which are
typical mechanics of RTS games. Additionally, the game facilitates simple character controller, character development
system and a mechanic of interactiong with non-playable characters, which are representative for computer role-playing games.

Diploma was carried out by three coauthors. Bartosz Strzelecki was responsible for preparing dialog system, artificial
intelligence of the non-playable characters, vision through objects mechanic as well as saving and loading state of a game.
He is the author of sections \ref{ss:rts},
\ref{s:ai_wpr}, \ref{chap:dialogi}, \ref{chap:dbd}, \ref{s:cel}, \ref{ss:navmesh}, \ref{ss:protobuf}, \ref{s:org}, \ref{s:proj_progres}, \ref{s:wid_proj},
\ref{s:dial_proj}, \ref{s:ai_proj}, \ref{s:impl_progres}, \ref{s:wid_impl}, \ref{s:dia_impl}, \ref{s:ai_impl}, and
\ref{s:save_impl}. He also coauthored \ref{ss:comp}, \ref{s:wpr_progres}, \ref{s:rozgrywka} and \ref{s:dalsze}. Bogna Lew
was responsible for modeling of the game's world as well as implementation of build mechanic and character controller. She
is the author of sections \ref{ss:rpg}, \ref{s:budowanie}, \ref{s:fabula}, \ref{s:walka}, \ref{s:silniki},
\ref{ss:tTool}, \ref{ss:so}, \ref{s:wymagania}, \ref{s:swiat}, \ref{s:por_proj}, \ref{s:build_proj}, \ref{s:por_impl}, \ref{s:bud_impl} and
\ref{s:efekty} and coauthored \ref{chap:introduction}, \ref{ss:comp}, \ref{s:wpr_progres} and \ref{s:rozgrywka}.
Zofia Sosińska was responsible for preparing user interface with a system for resolving user input. She is the author of
sections \ref{ss:tbs}, \ref{c:elem_ui}, \ref{chap:mb}, \ref{chap:ui},
\ref{chap:menu_main}, \ref{chap:naw}, \ref{chap:sjpzgp}, \ref{chap:ui_imp}, \ref{chap:menu_main_impl},
\ref{chap:naw_impl} and  \ref{s:opinia}. She also coauthored sections \ref{chap:introduction} and \ref{s:dalsze}.
