\section{Przegląd silników gier (Bogna Lew)}\label{s:silniki}
Silnik gier komputerowych jest oprogramowaniem służącym głównie do wytwarzania gier komputerowych. "Silnik zawiera
narzędzia i komponenty, które obsługują różne podstawowe elementy gry, takie jak rysowanie grafiki, kontrolowanie
dźwięku i przesuwanie obiektów, i pozwalają programiście skupić się na generowaniu danych dołączanych do silnika, takich
jak modele/obrazy, tekst, efekty dźwiękowe i tak dalej" \cite{design_essent}. Wybranie odpowiedniego silnika przed
rozpoczęciem pracy ma kluczowy wpływ na proces wytwórczy i jego efekt.

Obecnie dostępnych jest wiele silników, a każdy z nich ma różne możliwości. W celu uproszczenia wyboru zdecydowaliśmy
się zawęzić listę do trzech najpopularniejszych obecnie darmowych silników, czyli Godot, Unity oraz Unreal Engine.

Pierwszy z nich jest w pełni darmowym silnikiem open source. Posiada prosty i intuicyjny interfejs, a w Internecie
tworzone jest przez społeczność wiele samouczków. Nie posiada on jednak oficjalnej dokumentacji oraz jest zdecydowanie
mniej popularny od pozostałych dwóch.

Kolejny silnik, Unity jest określany jako przyjazny dla początkujących. Posiada bogatą dokumentację oraz jest dostępnych
dużo samouczków stworzonych przez jego społeczność. Unity świetnie się nadaje do tworzenia gier 3D. Silnik ten jest
dostępny w wersji bezpłatnej oraz oferującej więcej możliwości wersji płatnej. Co więcej, ma możliwość rozszerzenia
o dodatkowe narzędzia dostępne w Asset Store.

Ostatni z silników jest najbardziej kojarzony z grami AAA. Cechuje go zaawansowana grafika, która umożliwia wytwarzanie
fotorealistycznych gier. Korzystanie z niego jest darmowe, a opłata w wysokości 5\% jest naliczana jedynie, gdy gra
zarobi ponad milion USD.

Tabela \ref{fig:teng} przedstawia porównanie wymienionych silników w istotnych, z punktu widzenia projektu, aspektach.

\begin{table}[h]
\caption{Porównanie silników.}
\begin{center}
\begin{tabular}{ |c||c|c|c| }
 \hline
 Silnik & Unity & Unreal Engine & Godot \\
 \hline \hline
 Popularność & duża & duża & mała \\
 \hline
 3D & Tak & Tak & Tak \\
 \hline
 Język & C\# & C++ & C\#, C++, GDScript \\
 \hline
 Baza wiedzy & dokumentacja, samouczki & dokumentacja, samouczki & samouczki, fora \\
 \hline
 Open source & Nie & Nie & Tak \\
 \hline
\end{tabular}
\end{center}
\label{fig:teng} 
\end{table}

Finalnie zdecydowaliśmy się na implementację gry w Unity, ponieważ jest to silnik, który najlepiej odpowiada wymaganiom projektu.
