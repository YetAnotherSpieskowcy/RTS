\subsection{Terrain Toolbox (Bogna Lew)}\label{ss:tTool}
\texttt{Terrain Toolbox} jest narzędziem dostępnym dla silnika Unity. Jest to zasób dostępny w paczce \texttt{Terrain Tools}, który
upraszcza pracę nad modelowaniem terenu do gry.

Do podstawowych funkcjonalności udostępnianych przez \texttt{Terrain Toolbox} należy generowanie terenu na podstawie
map wysokości (ang. \textit{heightmap}) oraz podstawowych parametrów takich jak długość, szerokość i wysokość terenu.
Pozwala to na szybkie utworzenie grywalnej mapy. Ponadto narzędzie \texttt{Terrain Toolbox} umożliwia wygładzenie, dodatkowe
wymodelowanie oraz nałożenie tekstur na tak utworzony teren za pomocą udostępnionych przez nie funkcjonalności.

\begin{figure}[h!]
    \centering
    \includegraphics[width=0.8\textwidth]{images/modelowanie_terenu/przykladowe_heightmapy.jpg}
    \caption{Przykładowa mapa wysokości (po lewej) oraz wygenerowany na jej podstawie teren (po prawej).}
\end{figure}

Kolejną istotną funkcjonalnością jest malowanie terenu drzewami. Umożliwia ona automatyczne umiejscowienie obiektów na
mapie w losowy sposób. Pozwala to szybko utworzyć realistyczne skupiska obiektów, takich jak las. Poniżej
został przedstawiony przykładowy rezultat.

\begin{figure}[h!]
    \centering
    \includegraphics[width=0.8\textwidth]{images/modelowanie_terenu/drzewa.jpg}
    \caption{Widok na teren z drzewami.}
\end{figure}
