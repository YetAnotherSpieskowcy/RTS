\subsection{Protobuf (Bartosz Strzelecki)}

Biblioteka Protobuf jest wydajnym i uniwersalnym rozwiązaniem przystosowanym do serializacji
danych. Służy do zdefiniowania formatu przechowywanych informacji, który jest neutralny dla platformy.
W swojej istocie protobuf definiuje niezależny od języka programowania schemat opisu interfejsu,
który pozwala na określenie struktury danych za pomocą prostej i intuicyjnej składni.
Elastyczność komponentu Protobuf wykracza poza proste struktury danych, oferując obsługę złożonych typów danych, pól opcjonalnych i powtarzalnych,
a także struktur zagnieżdżonych. Dodatkowo udostępnia narzędzia do generowania kodu,
które automatycznie tworzą implementację specyficzną dla danego języka, ułatwiając pracę programistom.
