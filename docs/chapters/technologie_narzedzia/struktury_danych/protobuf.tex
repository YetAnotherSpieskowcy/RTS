\subsection{Protobuf (Bartosz Strzelecki)}\label{ss:protobuf}

Biblioteka \texttt{Protobuf} jest wydajnym i uniwersalnym rozwiązaniem przystosowanym do serializacji
danych. Służy do zdefiniowania formatu przechowywanych informacji, który jest neutralny dla platformy.
"\texttt{Protocol Buffers} zapewniają format serializacji dla pakietów o typowanych i ustrukturyzowanych danych o rozmiarze do
kilku megabajtów. Format nadaje się zarówno do efemerycznego ruchu sieciowego, jak i długoterminowego przechowywania
danych. \texttt{Protocol Buffers} można rozszerzyć o nowe informacje bez pozbawienia ważności istniejących danych lub
konieczności aktualizacji kodu." \cite{protobuf}.
W swojej istocie \texttt{Protobuf} definiuje niezależny od języka programowania schemat opisu interfejsu,
który pozwala na określenie struktury danych za pomocą prostej i intuicyjnej składni.
Elastyczność komponentu \texttt{Protobuf} wykracza poza proste struktury danych, oferując obsługę złożonych typów danych,
pól opcjonalnych i powtarzalnych,
a także struktur zagnieżdżonych. Dodatkowo udostępnia narzędzia do generowania kodu,
które automatycznie tworzą implementację specyficzną dla danego języka, ułatwiając pracę programistom.
