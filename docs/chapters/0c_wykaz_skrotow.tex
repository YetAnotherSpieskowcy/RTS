\chapter*{Wykaz ważniejszych oznaczeń i skrótów}
\addcontentsline{toc}{chapter}{Wykaz ważniejszych oznaczeń i skrótów}
\begin{description}[style=multiline,leftmargin=3cm]
\item[RTS] Gra strategii czasu rzeczywistego (ang. \textit{real-time strategy}). Jest to gatunek gier, w którym gracz nie jest
ograniczany przez turowość i kolejność ruchów. Wymusza szybsze podejmowanie decyzji, a ich skutki są natychmiastowe.
\item[cRPG] Komputerowa gra fabularna (ang. \textit{computer role-playing game}). Gatunek gier, w którym gracz wciela
się w postać lub drużynę, przemieszczając się w świecie stworzonym przez autorów gry.
\item[TBS] Strategiczna gra turowa (ang. \textit{turn-based strategy}). Jest to podgatunek gier strategicznych, w którym
gracze wykonują swoje akcje w turach.
\item[AAA (Triple-A)] Termin stosowany w przemyśle gier komputerowych. Służy do określenia wysokobudżetowych gier, od
których oczekuje się wysokiej jakości.
\item[Wielkie odkrycia geograficzne] Termin odnoszący się do odkryć geograficznych, które miały miejsce na przełomie XV
i XVI wieku.
\item[AI] Sztuczna inteligencja (ang. \textit{artificial intelligence}) jest wykorzystywana do imitowania inteligentnego
zachowania postaci niezależnych.
\item[NPC] Termin określa postacie niezależne (ang. \textit{non-playable character}), czyli postacie, które nie są kontrolowane bezpośrednio
przez gracza.
\item[Input Manager] Komponent silnika Unity, który umożliwia definiowanie wirtualnych osi oraz przypisywanie do nich
odpowiednich klawiszy. Poprzez odczyt wartości zwracanej przez oś możliwe jest wyznaczenie odpowiedniej akcji.
\item[UI] Interfejs użytkownika (ang. \textit{user interface}), czyli oprogramowanie umożliwiające użytkownikowi
interakcję z systemem.
\item[mapa wysokościowa] (ang. \textit{heightmap}) Jest to termin określający zdjęcie w odcieniach szarości obrazujące
wysokość terenu. Kolor czarny na zdjęciu reprezentuje najniższe punkty, natomiast kolor biały - najwyższe.
\end{description}
