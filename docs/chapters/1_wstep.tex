\chapter{Wstęp i cel pracy}\label{chap:introduction}

Celem projektu jest zaprojektowanie oraz zaimplementowanie gry real-time strategy osadzonej we wczesnym średniowieczu i przeznaczonej dla jednego gracza. Kluczowe dla niej będzie uwzględnienie mechanik, które umożliwią użytkownikowi podejmowanie decyzji w zbliżony sposób to tego wykorzystywanego w tamtych czasach. Dawniej decyzji strategicznych nie podejmowano na podstawie precyzyjnych map. Ludzie zmuszeni byli  budować  przestrzenny obraz istniejącej sytuacji  w toku dyskusji z naocznymi świadkami, takimi jak dowódcy czy podróżnicy. Z punktu widzenia projektu, uwzględnienie tych różnic jest kluczowe, gdyż umożliwi to graczowi wczucie się w realia epoki. Bliski pożądanemu efektowi jest sposób w jaki zrealizowano dowodzenie drużyną w grze RPG „Pillars of Eternity”. Niestety, te mechaniki w tego typu grach zwykle są bardzo ograniczone i nie pozwalają na zarządzanie większymi grupami wojsk, czy budowę baz, czyli zagadnienia podstawowe z punktu widzenia gier RTS. Z tego powodu celem pracy jest opracowanie projektu gry strategicznej czasu rzeczywistego dla jednego gracza, w której dowodzenie przebiegałoby w sposób możliwie zgodny z realiami historycznymi.