\chapter{Wstęp i cel pracy (Bogna Lew, Zofia Sosińska)}\label{chap:introduction}

Postrzeganie świata przed erą wielkich odkryć geograficznych znacząco różniło się od tego, które dominuje obecnie. Dawniej
decyzji strategicznych nie podejmowano na podstawie precyzyjnych map. Ludzie zmuszeni byli  budować  przestrzenny obraz
istniejącej sytuacji  w toku dyskusji z naocznymi świadkami, takimi jak dowódcy czy podróżnicy.

Współczesne gry komputerowe, których fabuła osadzona jest w tych czasach, stosuje wiele uproszczeń. Ma to na celu poprawienie jakości
rozgrywki gracza, jednakże sprawia, że nie oddaje w pełni realiów. Często obraz dzisiejszych możliwości zasłania faktyczne dowodzenie
w czasach, w których dzieje się gra. Technologia XXI wieku pozwala oficerom na wydawanie rozkazów na podstawie dokładnych map poprzez radio.
Jednak w dawniejszych czasach było to niemożliwe. Grywalna postać momentami ma wręcz boskie umiejętności i wiedzę. Zwykły człowiek nie był w
stanie w jednej chwili zobaczyć całego grywalnego świata i wydać komendę jednostkom, znajdującym się na drugim jego końcu, o przemieszczeniu
się do punktu z dokładnością do jednego metra.

Celem projektu jest zaprojektowanie oraz zaimplementowanie gry real-time strategy osadzonej we wczesnym średniowieczu i
przeznaczonej dla jednego gracza. Udostępniane przez nią mechaniki mają jak najlepiej oddawać realia tamtych czasów, przy
jednoczesnym zachowaniu podstawowych cech tego gatunku. Dodatkowo oferowane przez nią funkcjonalności nie mogą pogarszać
jakości rozgrywki.

W rozdziale \"Wprowadzenie do dziedziny\" zostały przedsawione przykłady podstawowych mechanik w grach strategii czasu
rzeczywistego. Kolejny rozdział skupia się na podstawowych narzędziach oraz technologiach wykorzystywanych do wytwarzania
gier komputerowych. Dodatkowo precyzuje wybrane przez zespół środowiska, które zostaną wykorzystane do implementacji
rozwiązań. W następnym rozdziale został przedstawiony projekt wytwarzanej gry. Skupia się on na wizji zespołu oraz
oczekiwanych efektach. Na koniec, w rozdziale \"Implementacja\" zaprezentowane zostały osiągnięte rezultaty oraz w jaki
sposób je opracowano.