\chapter{Wprowadzenie do dziedziny}\label{chap:field}

\section{Przegląd silników}
Podstawą tworzenia gier jest silnik graficzny. Stanowi serce kodu, odpowiadając za interakcję poszczególnych elementów. Dostarcza podstawowe narzędzia dzięki którym łatwiej można dokonywać zmian w grze. Wybranie odpowiedniego silnika przed rozpoczęciem pracy ma kluczowy wpływ na proces wytwórczy i efekt końcowy.

Obecnie dostępnych jest wiele silników, a każdy z nich posiada różne możliwości. W celu uproszczenia wyboru zdecydowaliśmy się zawęzić listę do trzech pozycji. Są to Godot, Unity oraz Unreal Engine.

Pierwszy z nich jest w pełni darmowym silnikiem open source. Posiada prosty i intuicyjny interfejs, a w internecie tworzonych jest przez społeczność wiele samouczków. Nie posiada on jednak oficjalnej dokumentacji oraz jest zdecydowanie mniej popularny od pozostałych dwóch.

Kolejny silnik, Unity jest określany jako przyjazny dla początkujących. Posiada bogatą dokumentację oraz jest dostępnych dużo samouczków stworzonych przez jego społeczność. Unity świetnie się nadaje do tworzenia gier 3D. Silnik ten jest dostępny w wersji bezpłatnej oraz oferującej więcej możliwości wersji płatnej. Co więcej, posiada możliwość rozszerzenia o dodatkowe narzędzia dostępne w Assets Store.

Ostatni z silników jest najbardziej kojarzony z grami AAA. Cechuje go zaawansowana grafika, która umożliwia wytwarzanie fotorealistycznych gier. Korzystanie z niego jest darmowe, a opłata w wysokości 5% jest naliczana jedynie gdy gra zarobi ponad milion USD.

Poniższa tabela przedstawia porównanie wymienionych silników w istotnych, z punktu widzenia projektu, aspektach.

\begin{table}[h]
\begin{center}
\begin{tabular}{ |c||c|c|c| }
 \hline
 Silnik & Unity & Unreal Engine & Godot \\
 \hline \hline
 Popularność & duża & duża & mała \\
 \hline
 Elastyczność & duża & duża & duża \\
 \hline
 Obsługa & dobra & dobra & dobra \\
 \hline
 Język & C\# & C++ & C\#, C++, GDScript \\
 \hline
 Baza wiedzy & dokumentacja,  samouczki & dokumentacja, samouczki & samouczki, fora \\
 \hline
\end{tabular}
\end{center}
\end{table}

Finalnie zdecydowaliśmy się na implementację gry w Unity, ponieważ jest to silnik, który najlepiej odpowiada wymaganiom projektu.