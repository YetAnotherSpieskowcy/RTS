\section{Opinia osoby z zewnątrz (Zofia Sosińska)}
Osoba niezależna została poproszona o zagranie w wytworzoną, prototypową grę. Następnie opisała swoje odczucia, które zostały przytoczone 
w tej pracy. O opinię poproszono osobę z wieloletnim dośwaidczeniem w graniu w różne rodzaje gier, aby miała ona odniesienie do produktów
dostepnych na rynku.

Pierwszym odruchem do przygotowania się do gry było ustawienie lewej ręki na klawiszach "W", "A", "S" i "D", a w prawą chwyciła myszkę. Spodziewano się 
że sterowanie będzie wymagało tych komponentów, gdyż jest to najczęstsze rozwiązanie. Następnie logiczne wydało mu się użycie przycisku lewego klawisza "Shift", aby przyspieszyć ruch postaci. 
Tak samo naturalne było użycie myszy do obrotu kamery, a całe sterowanie postacią zostało opisane jako przyjemne.

Po kolei odkrywano wszystkie mechaniki programu. Przyznano, że informacja o możliwej interakcji zwróciła uwagę na postaci, z którymi można było porozmawiać. 
Doceniono pieczołowite oddanie średniowiecznego stylu wypowiedzi w dialogach. Zabieg znacznie zwiększył immersję i umilił rozgrywkę. Zwrócono jednak uwagę,
że dialogi są długie i przydatna byłaby mechanika wyświetlenia całego tekstu po naciśnięciu jakiegoś przycisku. Interfejs budowania wzbudził bardzo pozytywną reakcję.
Uznano go za intuicyjny oraz przyjemny do użytku. Zwrócono uwagę na animację budowania budynku. Wykonana czynność podniosła poziom immersji gry. Przy nawigacji po świecie
kompas odegrał kluczową rolę. Dzięki niemu to zadanie było proste. Szybko spostrzeżono, że lewym przyciskiem myszy można wyprowadzić atak, co było bardzo wygodne. 
Przy konfrontacji z niedźwiedziem dłuższą chwilę zajęło wyczucie odległości, z jakiej nalepiej go uderzyć i za lepszy pomysł uznano zatrudnienie najemników, którzy szybko poradzili
sobie z bestią. Wydawanie im komend uznano za bardzo przydatną mechanikę, która wprowadziła element taktyczny. Jako miłe ułatwienie uznano dziennik ze streszczeniami zadań i 
podpowiedziami dotyczącymi mechanik gry.

Prototyp ogólnie uznano za dobrą bazę do późniejszego rozwijania. Projektanci zwracali uwagę na intuicyjność i wygodę sterowania, immersję gry i oddanie realiów wczesnego średniowiecza
oraz zaiplementowali najpotrzebniejsze mechaniki gry RTS. 