\chapter{Podsumowanie}\label{chap:summary}
Niniejszy rozdział zawiera podsumowanie osiągniętych rezultatów, ewentualne dalsze możliwości rozwoju oraz opinię o
utworzonym prototypie. W tym celu zespół poprosił osobę z zewnątrz o zagranie w przygotowaną grę oraz wyrażenie swojego
zdania na jej temat.

\section{Osiągnięte efekty (Bogna Lew)}
W ramach projektu został przygotowany prototyp gry strategii czasu rzeczywistego posiadającej aspekty komputerowej gry
fabularnej. Gra osadzona została w realiach wczesnego średniowiecza i czerpała inspirację z Celtów. Została ona
utworzona za pomocą silnika Unity oraz dostępnych dla niego narzędzi. Dodatkowo wykorzystano w niej modele i animacje,
które są możliwe do pozyskania w Asset Store oraz stylistycznie oddają realia wybranej epoki historycznej.

Utworzony prototyp udostępnia podstawowe mechanizmy typowe dla gatunku gier strategii czasu rzeczywistego, takie jak
zarządzanie jednostkami, budowanie budowli oraz prosty system zarządzania zasobami. Ponadto gra umożliwia użytkownikowi
możliwość sterowania własną postacią oraz wchodzenia w interakcje z postaciami niezależnymi, od których może przyjmować
zlecenia.

W grze występują cztery główne typy jednostek: postacie bez broni, miecznicy, łucznicy oraz budowniczowie. Pierwsze dwa
rodzaje odpowiadają za walkę w zwarciu, wykonując odpowiednie ataki. Łucznicy preferują wykonywać ataki dystansowe,
jednakże w razie konieczności podejmują walkę wręcz. Budowniczowe są jednostkami odpowiedzialnymi za budowanie budynków
wskazanych przez gracza i nie wykonują żadnych ataków. Gracz może wydawać komendy postaciom wojowników takie jak rozkaz
ataku, podążania za jego awatarem bądź udania się we wskazane miejsce.

Mechanizm budowania udostępnia graczowi cztery typy budowli: ławkę, namiot, dom oraz płot, a ich modele stylistycznie
pasują do wypranej epoki historycznej. Gra stosuje typowe dla swojego gatunku ograniczenie jakim jest konieczność
poniesienia kosztów budowy obiektu. Ponadto gra uniemożliwia graczowi umiejscowienie budowli na obszarze zbyt stromym
lub na którym znajdują się inne obiekty bądź postacie.

Gra zawiera również prosty interfejs użytkownika, wspomagający gracza w trakcie rozgrywki. Oferuje między innymi kompas,
na którym są zaznaczane najważniejsze punkty w świecie. Imituje to sposób nawigacji w epoce, w której osadzona jest gra,
kiedy to ludzie nie posiadali pełnej wiedzy o otoczeniu, a jedynie pewne ogólne informacje.

Utworzona gra spełnia główne założenie projektu, jakim jest jak najdokładniejsze oddanie realiów historycznych wybranej
epoki. Wykorzystywane w grze modele, bronie, czy słownictwo zostało starannie dobrane tak, aby dobrze oddawały
czasy wczesnego średniowiecza i panujący wtedy światopogląd.

\section{Opinia osoby z zewnątrz (Zofia Sosińska)}
Osoba niezależna została poproszona 
\section{Dalsze możliwości rozwoju (Bartosz Strzelecki, Zofia Sosińska)}\label{s:dalsze}
Powyższy projekt jest jedynie prototypem zawierającym podstawowe mechaniki.
Dalszy rozwój pracy głównie będzie polegać na konsolidacji zaimplementowanych
systemów w gotowy produkt.

Głównym zadaniem byłoby opracowanie i realizacja rozwiniętej linii fabularnej,
wykorzystując gotowe elementy wykonane w ramach projektu. Istotnym aspektem
byłaby wymiana modeli i tekstur z tymczasowych, prototypowych zasobów, służących
do celów demonstracyjnych zaimplementowanych mechanik na wyższej jakości
i lepiej odzwierciedlające klimat rozgrywki. Gry nie różnią się w tym wypadku od
innych produktów dostępnych na rynku. To, jak dobrze program wygląda tak samo wpłynie na
pozytywny odbiór przez użytkownika, jak chociażby wykonanie i jakość materiałów, z których są wykonane elementy
gier planszowych. "Estetyczne odczucia z grania w [...] gry ma znaczenie. Kiedy podnosisz pieczołowicie wyrzeźbiony pionek, reagujesz na niego
doceniając aspekt estetyczy - jedną z form przyjemności." \cite{theory_of_fun}.

Kolejnym elementem byłoby skomponowanie muzyki dobrze oddającej atmosferę wczesnego
średniowiecza oraz przygotowanie podkładów dźwiękowych, które zostałyby wykorzystane
jako efekty dźwiękowe na przykład podczas wykonywanych animacji ataku, podnoszenia przedmiotów itd.

Dalszy plan rozwoju skupiałby się głównie na aspektach audiowizualnych.
Powyższe aspekty pozytywnie wpłynęłyby na odbiór produktu przez potencjalnych przyszłych
użytkowników. 

\section{Opinia osoby z zewnątrz (Zofia Sosińska)}
Osoba niezależna została poproszona 
\section{Wnioski (Zofia Sosińska)}
Finalnie osiągnięto ustalony cel. Stworzony prototyp zawiera w sobie wszystkie mechaniki gry RTS,
które na początku zostały ustalone, a świat gry oddaje realia wczesnego Średniowiecza. 
Wśród funkcjonalności znalazły się takie, jak budowanie budynków, zarządzanie
jednostkami i surowcami oraz walka z przeciwnikami. Przy tym zwrócono uwagę, aby jak najdokładniej oddać 
realia, na które stylizowany jest program. Idąc tokiem myślenia ludzi wczesnego Średniowiecza, zaniechano
implementacji mapy, stawiając na użycie kompasu. Klimat epoki jest odczuwalny dzięki stylizacji budowli oraz ubrań i przedmiotów postaci,
jak również przy samym prowadzeniu dialogu z nimi. Wypowiedzi zarówno gracza, jak i rozmówcy są stylizowane na język ludzi tamtych czasów.

Przy dalszym rozwoju projektu warte uwagi będzie dodanie wysokiej jakości efektów audiowizualnych. Po dodaniu do stworzonego prototypu takich
aspektów produkt nabrałby unikalnego charakteru. Skomponowanie muzyki słysznej w tle oraz sygnalizującej wykonanie jakiejś czynności podniosłoby immersyjność i klimatyczność.
Stworzenie modeli i tekstur wyglądających estetycznie sprawiłoby, że rozgrywka stanie się przyjemniejsza.

Program został oceniony przez osobę niezależną jako dobra baza do późniejszego rozwinięcia. Zaimplementowane mechaniki często były intuicyjne i przyjemne
w obsłudze. Fabuła gry pozwoliła na użycie wszystkich zaplanowanych funkcjonalności gry RTS, a otoczenie i dialogi z postaciami oddały klimat
epoki wczesnego Średniowiecza.
