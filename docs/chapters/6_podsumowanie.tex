\chapter{Podsumowanie}\label{chap:summary}
Niniejszy rozdział zawiera podsumowanie osiągniętych rezultatów, ewentualne dalsze możliwości rozwoju oraz opinię o
utworzonym prototypie. W tym celu zespół poprosił osobę z zewnątrz o zagranie w przygotowaną grę oraz wyrażenie swojego
zdania na jej temat.

\section{Osiągnięte efekty (Bogna Lew)}
W ramach projektu został przygotoway prototyp gry strategii czasu rzeczywistego posiadającej aspekty komputerowej gry
fabularnej. Gra została utworzona za pomocą silnika Unity oraz dostępnych dla niego narzędzi. Dodatkowo wykorzystano w
niej zasoby takie jak modele i animacje, które są możliwe do pozyskania w Asset Store.

Utworzony prototyp udostępnia podstawowe mechanizmy typowe dla gatunku gier strategii czasu rzeczywistego, takie jak
zarządzanie jednostkami, budowanie budowli oraz prosty system zarządzania zasobami. Ponadto gra umożliwia użytkownikowi
możliwość sterowania własną postacią oraz wchodzenia w interakcje z postaciami niezależnymi, od których może przyjmować
zlecenia.

W grze występują cztery główne typy jednostek: postacie bez broni, miecznicy, łucznicy oraz budowniczowie. Pierwsze dwa
rodzaje odpowiadają za walkę w zwarciu, wykonując odpowiednie ataki. Łucznicy preferują wykonywać ataki dystansowe,
jednakże w razie konieczności podejmują walkę wręcz. Budowniczowe są jednostkami odpowiedzialnymi za budowanie budynków
wskazanych przez gracza i nie wykonują żadnych ataków. Gracz może wydawać komendy postaciom wojowników takie jak rozkaz
ataku, podążania za jego awatarem bądź udania się we wskazane miejsce.

Mechanizm budowania udostępnia graczowi cztery typy budowli: ławkę, namiot, dom oraz płot. Za wybudowanie obiektu
użytkownik musi ponieść koszty budowy. Ponadto gra uniemożliwia graczowy umiejscowienie budowli na obszarze zbyt stromym,
lub na którym znajdują się inne obiekty bądź postacie.

\section{Dalsze możliwości rozwoju (Bartosz Strzelecki, Zofia Sosińska)}\label{s:dalsze}
Powyższy projekt jest jedynie prototypem zawierającym podstawowe mechaniki.
Dalszy rozwój pracy głównie będzie polegać na konsolidacji zaimplementowanych
systemów w gotowy produkt.

Głównym zadaniem byłoby opracowanie i realizacja rozwiniętej linii fabularnej,
wykorzystując gotowe elementy wykonane w ramach projektu. Istotnym aspektem
byłaby wymiana modeli i tekstur z tymczasowych, prototypowych zasobów, służących
do celów demonstracyjnych zaimplementowanych mechanik na wyższej jakości
i lepiej odzwierciedlające klimat rozgrywki. Gry nie różnią się w tym wypadku od
innych produktów dostępnych na rynku. To, jak dobrze program wygląda tak samo wpłynie na
pozytywny odbiór przez użytkownika, jak chociażby wykonanie i jakość materiałów, z których są wykonane elementy
gier planszowych. "Estetyczne odczucia z grania w [...] gry ma znaczenie. Kiedy podnosisz pieczołowicie wyrzeźbiony pionek, reagujesz na niego
doceniając aspekt estetyczy - jedną z form przyjemności." \cite{theory_of_fun}.

Kolejnym elementem byłoby skomponowanie muzyki dobrze oddającej atmosferę wczesnego
średniowiecza oraz przygotowanie podkładów dźwiękowych, które zostałyby wykorzystane
jako efekty dźwiękowe na przykład podczas wykonywanych animacji ataku, podnoszenia przedmiotów itd.

Dalszy plan rozwoju skupiałby się głównie na aspektach audiowizualnych.
Powyższe aspekty pozytywnie wpłynęłyby na odbiór produktu przez potencjalnych przyszłych
użytkowników. 

