\chapter{Podsumowanie}\label{chap:summary}
Niniejszy rozdział zawiera podsumowanie osiągniętych rezultatów, ewentualne dalsze możliwości rozwoju oraz opinię o
utworzonym prototypie. W tym celu zespół poprosił osobę z zewnątrz o zagranie w przygotowaną grę oraz wyrażenie swojego
zdania na jej temat.

\section{Osiągnięte efekty (Bogna Lew)}
W ramach projektu został przygotowany prototyp gry strategii czasu rzeczywistego posiadającej aspekty komputerowej gry
fabularnej. Gra osadzona została w realiach wczesnego średniowiecza i czerpała inspirację z Celtów. Została ona
utworzona za pomocą silnika Unity oraz dostępnych dla niego narzędzi. Dodatkowo wykorzystano w niej modele i animacje,
które są możliwe do pozyskania w Asset Store oraz stylistycznie oddają realia wybranej epoki historycznej.

Utworzony prototyp udostępnia podstawowe mechanizmy typowe dla gatunku gier strategii czasu rzeczywistego, takie jak
zarządzanie jednostkami, budowanie budowli oraz prosty system zarządzania zasobami. Ponadto gra umożliwia użytkownikowi
możliwość sterowania własną postacią oraz wchodzenia w interakcje z postaciami niezależnymi, od których może przyjmować
zlecenia.

W grze występują cztery główne typy jednostek: postacie bez broni, miecznicy, łucznicy oraz budowniczowie. Pierwsze dwa
rodzaje odpowiadają za walkę w zwarciu, wykonując odpowiednie ataki. Łucznicy preferują wykonywać ataki dystansowe,
jednakże w razie konieczności podejmują walkę wręcz. Budowniczowe są jednostkami odpowiedzialnymi za budowanie budynków
wskazanych przez gracza i nie wykonują żadnych ataków. Gracz może wydawać komendy postaciom wojowników takie jak rozkaz
ataku, podążania za jego awatarem bądź udania się we wskazane miejsce.

Mechanizm budowania udostępnia graczowi cztery typy budowli: ławkę, namiot, dom oraz płot, a ich modele stylistycznie
pasują do wypranej epoki historycznej. Gra stosuje typowe dla swojego gatunku ograniczenie jakim jest konieczność
poniesienia kosztów budowy obiektu. Ponadto gra uniemożliwia graczowi umiejscowienie budowli na obszarze zbyt stromym
lub na którym znajdują się inne obiekty bądź postacie.

Gra zawiera również prosty interfejs użytkownika, wspomagający gracza w trakcie rozgrywki. Oferuje między innymi kompas,
na którym są zaznaczane najważniejsze punkty w świecie. Imituje to sposób nawigacji w epoce, w której osadzona jest gra,
kiedy to ludzie nie posiadali pełnej wiedzy o otoczeniu, a jedynie pewne ogólne informacje.

Utworzona gra spełnia główne założenie projektu, jakim jest jak najdokładniejsze oddanie realiów historycznych wybranej
epoki. Wykorzystywane w grze modele, bronie, czy słownictwo zostało starannie dobrane tak, aby dobrze oddawały
czasy wczesnego średniowiecza i panujący wtedy światopogląd.

\section{Opinia osoby z zewnątrz (Zofia Sosińska)}
Osoba niezależna została poproszona o zagranie w wytworzoną, prototypową grę. Następnie opisała swoje odczucia, które zostały przytoczone 
w tej pracy. O opinię poproszono osobę z wieloletnim dośwaidczeniem w graniu w różne rodzaje gier, aby miała ona odniesienie do produktów
dostepnych na rynku.

Pierwszym odruchem do przygotowania się do gry było ustawienie lewej ręki na klawiszach "W", "A", "S" i "D", a w prawą chwyciła myszkę. Spodziewano się 
że sterowanie będzie wymagało tych komponentów, gdyż jest to najczęstsze rozwiązanie. Następnie logiczne wydało mu się użycie przycisku lewego klawisza "Shift", aby przyspieszyć ruch postaci. 
Tak samo naturalne było użycie myszy do obrotu kamery, a całe sterowanie postacią zostało opisane jako przyjemne.

Po kolei odkrywano wszystkie mechaniki programu. Przyznano, że informacja o możliwej interakcji zwróciła uwagę na postaci, z którymi można było porozmawiać. 
Doceniono pieczołowite oddanie średniowiecznego stylu wypowiedzi w dialogach. Zabieg znacznie zwiększył immersję i umilił rozgrywkę. Zwrócono jednak uwagę,
że dialogi są długie i przydatna byłaby mechanika wyświetlenia całego tekstu po naciśnięciu jakiegoś przycisku. Interfejs budowania wzbudził bardzo pozytywną reakcję.
Uznano go za intuicyjny oraz przyjemny do użytku. Zwrócono uwagę na animację budowania budynku. Wykonana czynność podniosła poziom immersji gry. Przy nawigacji po świecie
kompas odegrał kluczową rolę. Dzięki niemu to zadanie było proste. Szybko spostrzeżono, że lewym przyciskiem myszy można wyprowadzić atak, co było bardzo wygodne. 
Przy konfrontacji z niedźwiedziem dłuższą chwilę zajęło wyczucie odległości, z jakiej nalepiej go uderzyć i za lepszy pomysł uznano zatrudnienie najemników, którzy szybko poradzili
sobie z bestią. Wydawanie im komend uznano za bardzo przydatną mechanikę, która wprowadziła element taktyczny. Jako miłe ułatwienie uznano dziennik ze streszczeniami zadań i 
podpowiedziami dotyczącymi mechanik gry.

Prototyp ogólnie uznano za dobrą bazę do późniejszego rozwijania. Projektanci zwracali uwagę na intuicyjność i wygodę sterowania, immersję gry i oddanie realiów wczesnego średniowiecza
oraz zaiplementowali najpotrzebniejsze mechaniki gry RTS. 
\section{Dalsze możliwości rozwoju (Bartosz Strzelecki)}
Powyższy projekt jest jedynie prototypem zawierającym podstawowe mechaniki.
Dalszy rozwój pracy głównie będzie polegać na konsolidacji zaimplementowanych
systemów w gotowy produkt.

Głównym zadaniem byłoby opracowanie i realizacja rozwiniętej linii fabularnej,
wykorzystując gotowe elementy wykonane w ramach projektu. Istotnym aspektem
byłaby wymiana  modeli i tekstur z tymczasowych prototypowych zasobów, służących
do celów demonstracyjnych zaimplementowanych mechanik na w wyższej jakości
i lepiej odzwierciedlające klimat rozgrywki.

Kolejnym elementem byłoby skomponowania muzyki dobrze oddającej atmosferę wczesnego
średniowiecza oraz przygotowanie podkładów dźwiękowych, które zostałyby wykorzystane
jako efekty dźwiękowe na przykład podczas wykonywanych animacji ataku, podnoszenia przedmiotów itd.

Dalszy plan rozwoju skupiałby się głównie na aspektach audiowizualnych.
Powyższe aspekty pozytywnie wpłynęłyby na odbiór produktu przez potencjalnych przyszłych
użytkowników. 

\section{Opinia osoby z zewnątrz (Zofia Sosińska)}
Osoba niezależna została poproszona o zagranie w wytworzoną, prototypową grę. Następnie opisała swoje odczucia, które zostały przytoczone 
w tej pracy. O opinię poproszono osobę z wieloletnim dośwaidczeniem w graniu w różne rodzaje gier, aby miała ona odniesienie do produktów
dostepnych na rynku.

Pierwszym odruchem do przygotowania się do gry było ustawienie lewej ręki na klawiszach "W", "A", "S" i "D", a w prawą chwyciła myszkę. Spodziewano się 
że sterowanie będzie wymagało tych komponentów, gdyż jest to najczęstsze rozwiązanie. Następnie logiczne wydało mu się użycie przycisku lewego klawisza "Shift", aby przyspieszyć ruch postaci. 
Tak samo naturalne było użycie myszy do obrotu kamery, a całe sterowanie postacią zostało opisane jako przyjemne.

Po kolei odkrywano wszystkie mechaniki programu. Przyznano, że informacja o możliwej interakcji zwróciła uwagę na postaci, z którymi można było porozmawiać. 
Doceniono pieczołowite oddanie średniowiecznego stylu wypowiedzi w dialogach. Zabieg znacznie zwiększył immersję i umilił rozgrywkę. Zwrócono jednak uwagę,
że dialogi są długie i przydatna byłaby mechanika wyświetlenia całego tekstu po naciśnięciu jakiegoś przycisku. Interfejs budowania wzbudził bardzo pozytywną reakcję.
Uznano go za intuicyjny oraz przyjemny do użytku. Zwrócono uwagę na animację budowania budynku. Wykonana czynność podniosła poziom immersji gry. Przy nawigacji po świecie
kompas odegrał kluczową rolę. Dzięki niemu to zadanie było proste. Szybko spostrzeżono, że lewym przyciskiem myszy można wyprowadzić atak, co było bardzo wygodne. 
Przy konfrontacji z niedźwiedziem dłuższą chwilę zajęło wyczucie odległości, z jakiej nalepiej go uderzyć i za lepszy pomysł uznano zatrudnienie najemników, którzy szybko poradzili
sobie z bestią. Wydawanie im komend uznano za bardzo przydatną mechanikę, która wprowadziła element taktyczny. Jako miłe ułatwienie uznano dziennik ze streszczeniami zadań i 
podpowiedziami dotyczącymi mechanik gry.

Prototyp ogólnie uznano za dobrą bazę do późniejszego rozwijania. Projektanci zwracali uwagę na intuicyjność i wygodę sterowania, immersję gry i oddanie realiów wczesnego średniowiecza
oraz zaiplementowali najpotrzebniejsze mechaniki gry RTS. 
\section{Wnioski (Zofia Sosińska)}
Finalnie osiągnięto ustalony cel. Stworzony program zawiera w sobie wszystkie mechaniki gry RTS,
które na początku zostały ustalone, a świat gry oddaje realia wczesnego średniowiecza. 
Efekt prac został oceniony pozytywnie przez osobę niezależną, jednak trzeba mieć na uwadze, 
że nie należy go oceniać, jak pełnowymiary produkt dostępny na rynku, a jedynie prototyp.
Tworzenie gry nie obeszło się bez wymagających zadań ze względu na ograniczony czas oraz wielkość 
projektu. Udostępnione zostały działjące mechniki, jednak często w okrojonej bądź uproszczonej formie.
Należy też dodać, że tak samo jak w przypadku dużych stódiów przeznaczjących na projekty lata, także 
w tej grze funkcjonalności nie pokrywają prawidłowym działaniem 100\% przypadków.
