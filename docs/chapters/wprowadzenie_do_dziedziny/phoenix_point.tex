\section{Modele celowania w grach (Bartosz Strzelecki)}
Celowanie w grach taktycznych jest podstawowym elementem rozgrywki, który ma
olbrzymi wpływ na przebieg rozgrywki. Ten mechanizm jest przede wszystkim wykorzystywany
do wzbogacenia gry o kolejną warstwę decyzji taktycznych i zażądania podejmowanym ryzykiem.
Gracz, przed oddaniem strzału, podejmuje decyzję czy ryzykuje podjęcie strzału o niskiej szansy
na trafienie, czy zdecyduję się na niezawodny atak, lecz wtedy może być wystawiony na ataki przeciwników.

W Phoenix Point modelowanie celności to wieloaspektowy system, który w zawiły sposób definiuje wynik interakcji bojowych. 
Gra wykorzystuje dynamiczny system celowania, który uwzględnia różne elementy, takie jak postawa żołnierza, biegłość w posługiwaniu się bronią, zasięg, 
osłona i warunki środowiskowe, aby określić precyzję strzału. Każdy z tych elementów odgrywa znaczącą rolę w ogólnym obliczeniu trafienia w cel.

W przeciwieństwie do podobnej gry XCOM, gdzie celność jest zamodelowana za pomocą prostej szansy na trafienie, w grze Phoenix Point
trajektoria każdego pocisku obliczana jest osobno. Podczas celowania widoczne są dwa okręgi: wewnętrzny, który reprezentuje miejsce,
w którym znajdzie się 50\% pocisków oraz zewnętrzny, który reprezentuje maksymalny rozrzut broni. W tym przypadku im celniejsza broń tym
okręgi będą mniejsze.

\begin{figure}[h]
\centering
\includegraphics[width=0.6\textwidth]{images/point}
\caption{System celowania występujący w grze Phoenix Point.}
\label{fig:acc}
\end{figure}

Ostatecznie system zaimplementowany w grze Phoenix Point okazuje się dużo bardziej realistyczny i pozwala na utrzymywanie ciągłego napięcia
pomiędzy trafieniem i chybieniem celu. Umożliwiając w ten sposób na strategiczne wyzwania, leżących u podstaw gier tego typu. 
