\section{Model sztucznej inteligencji przeciwników w grach RTS (Bartosz Strzelecki)}
W sferze gier RTS sztuczna inteligencja (ang. artificial intelligence, AI) odgrywa kluczową rolę, wspierając doświadczenia z rozgrywki.
W tym przypadku termin odnosi się do zbioru algorytmów i systemów zaprojektowanych w celu symulacji
zachowań podobnych do tych gracza. Między innymi umiejętność podejmowania decyzji i rozwiązywania problemów.

Sztuczna inteligencja przeciwników w grach takich jak Warcraft III lub StarCraft II, przede wszystkim w trybie kampanii,
jest odpowiedzialna za kontrolowanie wrogich jednostek w celu zaoferowania graczowi wyzwania. Głównym zadaniem AI jest zasymulowanie
strategicznych decyzji i wydajne zarządzanie zasobami.
AI podejmuje decyzję na podstawie predefiniowanych zasad i algorytmów. Analizuje sytuację, w której się znajduje, biorąc pod uwagę
siłę swojej własnej armii, siłę armii gracza oraz specjalne zdolności jednostek i środowisko, w którym toczy się gra.
Ta analiza pozwala komputerowi na podejmowanie strategicznych decyzji jak na przykład, kiedy atakować, bronić się, eksplorować oraz rozszerzać swoje terytorium.
W tych grach sztuczna inteligencja może przybrać jeden z kilku wariantów wynikających z poziomu trudności. Wyższe poziomy
dają przeciwnikowi przewagę taką, jak wydajniejsze zbieranie zasobów lub szybsza produkcja jednostek.

W grach strategicznych czasu rzeczywistego w trybie kampanii zachowanie przeciwników jest zaprojektowane z myślą o zanurzeniu gracza w fabularnej opowieści, jednocześnie
prezentując wyzwania związane z rozgrywką. Akcje wykonywane przez sztuczną inteligencję są dostosowane do celów danej misji, co pozwala
na dopasowanie do obowiązującej narracji.
Początkowo przeciwnik konstruuje i rozbudowuje swoją bazę, w celu zgromadzenia odpowiedniej liczby zasobów, szkolenia jednostek i prowadzenia badań.
AI strategicznie rozmieszcza budynki i struktury obronne, aby ochronić swoją fortecę przed najazdami gracza. 
Misje kampanii często też zawierają oskryptowane wydarzenia lub walki, które dodają głębi rozgrywce. Podczas tych starć wroga sztuczna inteligencja
może zachowywać się w specjalny sposób, kontrolując potężne jednostki, do których gracz normalnie nie ma dostępu lub inicjując działania, które popychają
narrację do przodu.
Zachowanie wroga w kampanii jest zróżnicowane i obejmuje różnorodne cele misji i scenariusze. Gracze mogą napotkać wrogów, którzy preferują agresywne ataki,
inni skupiają się na strategiach obronnych lub specjalizują się w taktyce hit and run. Sztuczna inteligencja dostosowuje proces podejmowania decyzji do
konkretnych wymagań misji, często wykorzystując ukształtowanie terenu, synergię jednostek i scenariusze wydarzeń, aby rzucić wyzwanie umiejętnościom gracza.
Motywuje to do wykorzystywania myślenia strategicznego, zarządzania zasobami i efektywnego składu jednostek, aby przezwyciężyć różnorodne strategie stosowane przez wrogą sztuczną inteligencję.
