\section{Model sztucznej inteligencji przeciwników w grach RTS (Bartosz Strzelecki)}\label{s:ai_wpr}
W większości gier, w tym omówionych w poprzednim podrozdziale, kluczową rolę odgrywa sztuczna inteligencja AI (ang.  \textit{artificial intelligence}), wspierając doświadczenia z rozgrywki.
W tym przypadku termin odnosi się do zbioru algorytmów i systemów zaprojektowanych w celu symulacji
zachowań podobnych do tych gracza, czyli między innymi umiejętności podejmowania decyzji i rozwiązywania problemów.

Sztuczna inteligencja przeciwników w grach takich jak Warcraft III\footnote{\url{https://warcraft3.blizzard.com}} lub StarCraft II\footnote{\url{https://starcraft2.blizzard.com}}, przede wszystkim w trybie kampanii,
jest odpowiedzialna za kontrolowanie wrogich jednostek w celu zaoferowania graczowi wyzwania. Głównym zadaniem AI jest zasymulowanie
strategicznych decyzji i wydajne zarządzanie zasobami.
"W przypadku gier strategicznych czasu rzeczywistego ruch idzie w parze z odnajdywaniem ścieżki. Jednostki z pewnością potrzebują planu (np. ścieżki), 
w jaki sposób przedostanie się z jednej strony miasta na drugą."\cite{units}
AI podejmuje decyzję na podstawie predefiniowanych zasad i algorytmów. Analizuje sytuację, w której się znajduje, biorąc pod uwagę
siłę swojej własnej armii, siłę armii gracza oraz specjalne zdolności jednostek i środowisko, w którym toczy się gra.
Ta analiza pozwala komputerowi na podejmowanie strategicznych decyzji, takich jak na przykład, kiedy atakować, bronić się, eksplorować, czy rozszerzać swoje terytorium.
W tych grach sztuczna inteligencja może przybrać jeden z kilku wariantów wynikających z poziomu trudności. Wyższe poziomy
dają przeciwnikowi przewagę taką jak wydajniejsze zbieranie zasobów lub szybszą produkcję jednostek.

Początkowo przeciwnik konstruuje i rozbudowuje swoją bazę, w celu zgromadzenia odpowiedniej liczby zasobów, szkolenia jednostek i prowadzenia badań.
AI strategicznie rozmieszcza budynki i struktury obronne, aby ochronić swoją fortecę przed najazdami gracza. 
Misje kampanii często też zawierają oskryptowane wydarzenia lub walki, które dodają głębi rozgrywce. Podczas tych starć wroga sztuczna inteligencja
może zachowywać się w specjalny sposób, kontrolując potężne jednostki, do których gracz normalnie nie ma dostępu lub inicjując działania, które popychają
narrację do przodu.
Zachowanie wroga w kampanii jest zróżnicowane i obejmuje różnorodne cele misji i scenariusze. Gracze mogą napotkać wrogów, którzy preferują agresywne ataki,
skupiają się na strategiach obronnych lub specjalizują się w taktyce hit and run. Sztuczna inteligencja dostosowuje proces podejmowania decyzji do
konkretnych wymagań misji, często wykorzystując ukształtowanie terenu, synergię jednostek i scenariusze wydarzeń, aby rzucić wyzwanie umiejętnościom gracza.
Motywuje to do wykorzystywania myślenia strategicznego, zarządzania zasobami i efektywnego dobierania jednostek, aby przezwyciężyć różnorodne strategie stosowane przez wrogą sztuczną inteligencję.

Ten model jest zaimplementowany między innymi przy użyciu algorytmów takich jak A*, który jest rozszerzoną wersją algorytmu Dijkstry o wykorzystanie heurystyki
optymalizującej wyszukiwanie. Stosowany jest również algorytm drzewa decyzyjnego pozwalający na zdefiniowanie zachowań agentów sztucznej inteligencji z zastosowaniem wydarzeń losowych.
