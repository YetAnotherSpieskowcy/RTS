\section{Mechanizm walki oraz zarządzania ekwipunkiem w Kingdom Come: Deliverance (Bogna Lew)}

Kingdom Come: Deliverance to gra z gatunku RPG osadzoną w realiach Europy Środkowej na początku XV wieku. Chociaż nie
jest to gra czasu rzeczywistego to jest to pozycja warta wymienienia ze względu na dbałość twórców o zachowanie realizmu
epoki oraz staranność wykonania mechanizmów walki oraz zarządzania ekwipunkiem. Jest ona przeznaczona dla jednego gracza,
a całość zaprezentowana jest z perspektywy pierwszoosobowej. W trakcie rozgrywki użytkownik rozwija swoją postać, bierze
udział w starciach, prowadzi rozmowy z niezależnymi postaciami i wiele więcej.

Godny uwagi jest mechanizm walki. Twórcy skupili się na jak najdokładniejszym oddaniu średniowiecznego stylu walki. W tym
celu skrupulatnie przestudiowali w jaki sposób władano mieczem w tamtych czasach, a następnie w pełni oddali to w grze.
Wykorzystali do tego tysiące animacji oraz starannie oddali fizykę pojedynków. W efekcie powstał realistyczny mechanizm
walki, który umożliwia graczowi parowanie, zadawanie ciosów oraz blokowanie.

Kolejnym elementem wartym wymienienia jest rozbudowany system zarządzania ekwipunkiem. Przeznaczony do tego panel jest
podzielony na dwie sekcje - jedna w której wyświetlona jest lista posiadanych rzeczy oraz druga przeznaczona na postać
gracza. Może on dowolnie spersonalizować swoją postać, poprzez możliwość założenia wielu elementów ubioru naraz tworząc
warstwy.  Dodatkowo, każdy przedmiot posiada swoją wagę, a postać swój maksymalny udźwig. W przypadku przekroczenia limitu
gracz zostaje ukarany poprzez spowolnienie ruchów w walce i uniemożliwieniu biegania. Jest to wzorowane na rzeczywistości,
dzięki czemu gra jeszcze lepiej oddaje realia epoki.
