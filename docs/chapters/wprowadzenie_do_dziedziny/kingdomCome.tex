\section{Sterowanie postacią oraz walka (Bogna Lew)}\label{s:walka}
Udostępnianie graczowi jego własnej postaci jest cechą charakterystyczną raczej komputerowych gier fabularnych niż gier
strategii czasu rzeczywistego. Jednakże jest to ciekawe rozwiązanie, które w znaczny sposób urozmaica rozgrywkę oraz
wpływa na stopień zaangażowania użytkownika. Udostępnienie graczowi jego własnej postaci "czyni gracza kreatywnym
elementem działającym wewnątrz dyskursu, który posiada przestrzenny charakter." \cite{olbrzymwcieniu}.

Przykładem gry implementującej łatwy w obsłudze sposób sterowania postacią jest gra \textit{The Elder Scrolls V: Skyrim}. Umożliwia
ona graczowi możliwość poruszania się postacią za pomocą klawiszy \texttt{W}, \texttt{A}, \texttt{S} oraz \texttt{D}. Dodatkowo użytkownik może
zmieniać prędkość swojego bohatera, przytrzymując klawisze \texttt{Alt} bądź \texttt{Ctrl}, które odpowiednio powodują zwolnienie lub
przyspieszenie tempa przemieszczania się. Do rozglądania się wykorzystywana jest mysz, której ruch powoduje obrót postaci
wokół własnej osi. Ataki gracz może wykonać poprzez naciśnięcie prawego bądź lewego przycisku myszy, które powodują
wykonanie akcji odpowiednio lewą lub prawą dłonią. Jest to przykład dbałości o szczegóły, które sprawiają, że gra
jest jeszcze bardziej interesująca i satysfakcjonująca.

Innym tytułem wartym uwagi jest \textit{Kingdom Come: Deliverance}\footnote{\url{https://www.kingdomcomerpg.com/pl}}. Jest to gra osadzona w realiach Europy Środkowej na początku
XV wieku. Godny uwagi jest jej mechanizm walki, ponieważ twórcy skupili się na jak najdokładniejszym oddaniu średniowiecznego
stylu walki. W tym celu skrupulatnie przestudiowali, w jaki sposób władano mieczem w tamtych czasach, a następnie w
pełni oddali to w grze. Wykorzystali do tego tysiące animacji oraz starannie oddali fizykę pojedynków. W efekcie powstał
realistyczny mechanizm walki, który umożliwia graczowi parowanie, zadawanie ciosów oraz blokowanie.

Kolejnym interesującym tytułem jest gra \textit{Mount\&Blade}. Podobnie jak w przypadku \textit{Skyrima}, gracz steruje swoją postacią za
pomocą klawiszy \texttt{W}, \texttt{A}, \texttt{S} oraz \texttt{D}. Co więcej, gracz może wykonywać ataki poprzez naciśnięcie
lewego przycisku myszy, co jest typowym rozwiązaniem w grach,
które w prosty sposób umożliwia graczowi uczestniczenie w potyczkach. Jednakże w przeciwnieństwie do \textit{Skyrima}, w
\textit{Mount\&Blade} ruch myszą nie powoduje obrotu całej postaci,
a jedynie kamery. Zmiana kierunku odbywa się wyłącznie za pomocą klawiszy sterujących. Dzięki temu gracz może zobaczyć, co
się dzieje za jego postacią bez konieczności zmiany kierunku ruchu bądź zatrzymania się. Powoduje to jednak pewne kłopoty z zachowaniem
realizmu, gdyż w prawdziwym świecie nie jest możliwe zobaczenie czegoś bez konieczności zwrócenia się w tę stronę.

Powyższe przykłady obrazują najpopularniejsze rozwiązania umożliwiające sterowanie postacią oraz wykonywanie ataków.
Typowymi narzędziami wykorzystywanymi do poruszania postacią oraz kamerą są myszka i klawisze \texttt{W}, \texttt{A}, \texttt{S} oraz \texttt{D},
natomiast do atakowania - lewy przycisk myszy. Różnice zaczynają się pojawiać w samych mechanikach, gdyż to w jaki
sposób postać gracza będzie się zachowywać zależy od wizji autorów gry.