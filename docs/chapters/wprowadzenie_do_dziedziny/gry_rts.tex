\section{Specyfika gier RTS (Bartosz Strzelecki)}

Strategie czasu rzeczywistego (RTS) odróżnia się od innych gatunków występowaniem zarówno strategicznych, jak i taktycznych
elementów rozgrywki. Podstawowe założenia gry RTS to skomplikowana równowaga pomiędzy zarządzaniem zasobami, budowaniem budynków oraz
dowodzeniem armii. Centralnym elementem rozgrywki RTS jest zarządzanie różnorodnymi jednostkami, z których każda posiada unikalne zdolności i rolę na polu bitwy.
Gry te działają w dynamicznym środowisku czasu rzeczywistego, co odróżnia je od tytułów strategii turowych. Ten model wymusza szybkie podejmowanie decyzji
i wymaga od gracza podzielności uwagi. Ogólnie rzecz biorąc, gatunek RTS oddaje dreszcz emocji związanych z prowadzeniem działań strategicznych, co wymaga połączenia zaradności,
przewidywania i sprytnego podejmowania decyzji na szybkim i stale zmieniającym się polu bitwy.
