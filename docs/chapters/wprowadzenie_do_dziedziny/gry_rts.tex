\section{Specyfika gier RTS (Bartosz Strzelecki)}

Strategie czasu rzeczywistego (RTS) odróżnia się od innych gatunków występowaniem zarówno strategicznych, jak i taktycznych
elementów rozgrywki. Podstawowe założenia gry RTS to skomplikowana równowaga pomiędzy zarządzaniem zasobami, budowaniem budynków oraz
dowodzeniem armii. Centralnym elementem rozgrywki RTS jest zarządzanie różnorodnymi jednostkami, z których każda posiada unikalne zdolności i rolę na polu bitwy.

Gry te działają w dynamicznym środowisku czasu rzeczywistego, co odróżnia je od strategii turowych TBS (ang. turn-based stratrgy). Ten model wymusza szybkie podejmowanie decyzji
i wymaga od gracza podzielności uwagi. Strategie czasu rzeczywistego zwykle też posiadają bardziej złożoną konstrukcję mapy, zawierającą skomplikowane
rozłożenia celów i przeszkód, natomiast mapy w strategiach turowych zazwyczaj wykorzystują ruch oparty o siatkę, co zapewnia dużo bardziej
zorganizowane środowisko rozgrywki.

W grach strategicznych czasu rzeczywistego w trybie kampanii zachowanie przeciwników jest zaprojektowane z myślą o zanurzeniu gracza w fabularnej opowieści, jednocześnie
prezentując wyzwania związane z rozgrywką. Akcje wykonywane przez sztuczną inteligencję są dostosowane do celów danej misji, co pozwala
na dopasowanie do obowiązującej narracji.

Tego typu produkcje pozwalają również na wciągającą rozgrywkę w trybie wieloosobowym, w którym gracze mogą rywalizować między sobą, jak również
współpracować w celu pokonania wspólnego przeciwnika kontrolowanego przez sztuczną inteligencję. Na ten styl rozgrywki jest nakładany istotny nacisk
w wielu grach tego gatunku, poprzez zapewnienie zróżnicowanych grywalnych frakcji zachowując przy tym symetrię balansy rozgrywki.

Ogólnie rzecz biorąc, gatunek RTS oddaje dreszcz emocji związanych z prowadzeniem działań strategicznych, co wymaga połączenia zaradności,
przewidywania i sprytnego podejmowania decyzji na szybkim i stale zmieniającym się polu bitwy.
