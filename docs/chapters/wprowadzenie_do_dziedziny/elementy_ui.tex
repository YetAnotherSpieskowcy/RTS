\section{Interfejs użytkownika (Zofia Sosińska)}\label{c:pasek_war3}

Dopuszczalnym założeniem przy tworzeniu gry komputerowej jest, że postać gracza ma jakąś wiedzę o świecie, w którym się znajduje oraz
poziom inteligencji i umiejętności, które pozwalają jej przeanalizować konkretną sytuacje i wyciągnąć wnioski. Obrazującym to przykładem 
może byc sytuacjca stawiania budynku. Postać wie, że dana budowla kosztuje pewną sumę i ilość surowców oraz że ją na to stać. Może 
wyciągnąć takie wnioski, ponieważ zakładamy, że wciągu jej życia zdażyło jej się taki budynek zakupić oraz pamięta stan swoich funduszy 
i surowców. Skoro więc te informacje są w zasięgu możliwości wnioskowania postaci, gra ma prawo wyświetlić je użytkownikowi. 
Takie graficzne przedstawienie najważniejszych danych z otoczenia gracza nazywamy interfejsem użytkownika UI (ang. \textit{user interface}). Narzędziami odzwierciedlania informacji
w czytelny i przystępny sposób mogą być takie elementy jak obrazki, teksty, czy wskaźniki. Dzięki UI możliwy jest wgląd w aktualny stan wiedzy grywalnej postaci, a tym samym lepsze
zrozumienie otoczenia. Wymagające podkreślenia jest, że “[...] na stopień zaangażowania gracza na poziomie świata gry duży wpływ ma interfejs, którym się posługuje. Mówiąc najprościej:
gracz, który rozpoczyna grę, zanim dozna poczucia integracji ze sterowaną przez siebie postacią, musi - używając stwierdzenia Aarsetha - spełnić oczekiwania interfejsu.”\cite{olbrzymwcieniu}

Jednym z kluczowych problemów przy projektowaniu elementów interfejsu użytkownika jest odpowiednie wybranie miejsca na ekranie. Konieczne jest zwrócenie uwawgi na takie 
problemy, jak od wielkość, rola, priorytet dostarczenia informacji, częstość i długość występowania oraz częstotliwość zmiany zawartości.

Inną ważną cechą dobrze zaprojektowanego interfejsu użytkownika jest zbalansowanie między pokazaniem jak największej liczby kluczowych informacji, 
jednocześnie okrawając je tak, aby nie przytłaczały gracza i nie zaciemniały przekazu. Użytkownik może cały czas widzieć na ekranie migające statystyki
oraz dodtawać informacje o poczynaniach każdej swojej jednostki, ale jest szansa, że te dane go przeciąża i będą przeszkadzać w cieszeniu się rozgrywką. 

Często dane segreguje się według ich tematyki, ale czasem właściwsze może okazać się zgrupowanie ich 
względem tego, jak często gracz będzie ich potrzebował. Jeśli są to informacje bardzo podstawowe do zrozumienia aktualnego 
stanu świata gry, wartościowe może być pokazanie ich obok siebie.

Przykładem zaprojektowania zbalansowanego elementu interfejsu użytkownika jest rozwiązanie gry Warcraft III: Reign of Chaos studia Blizzard Entertainment.
Zgrupowała ona najbardziej podstawowe informacje o świecie gry i skupiła te dane w cienkim pasku na samej górze ekranu. 
Skład elementów tej części interfjsu użytkownika jest niezmienny: pola otwierające zakładki, pora dnia oraz trzy wskaźniki zasobów. Pasek jest widoczny
podczas całej rozgrywki, niezależnie od wykonywanych czynności. W tym statycznie zakotwiczonym na górze ekranu elemencie, dynamicznie
zmieniają się jedynie ciągle aktualizowane informacje. Odpowiednio podmieniana jest tekstura pory dnia, zmieniająca się ze Słońca
na Księżyc oraz stan zasobów, zależnie od wydania, czy pozyskania.

\begin{figure}[htbp]
    \centering
    \includegraphics[width=1.0\textwidth]{images/ui/warcraft3_gorny_pasek_lewy.png}
    \caption{Lewa część paska z informacjami w grze Warcraft 3.}\label{fig:Warcraft3}
\end{figure}

\begin{figure}[htbp]
    \centering
    \includegraphics[width=1.0\textwidth]{images/ui/warcraft3_gorny_pasek_prawy.png}
    \caption{Prawa część paska z informacjami w grze Warcraft 3.}\label{fig:Warcraft3}
\end{figure}