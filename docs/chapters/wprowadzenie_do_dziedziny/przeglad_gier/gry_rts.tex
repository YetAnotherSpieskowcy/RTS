\subsection{Gry RTS (Bartosz Strzelecki)}\label{ss:rts}
Strategie czasu rzeczywistego RTS (ang. \textit{real-time strategy}) odróżnia od innych gatunków występowanie zarówno strategicznych, jak i taktycznych
elementów rozgrywki, w której czas, w tym szybkość reakcji ma niezwykle istotne znaczenie. Podstawowe założenia klasycznych gier RTS to skomplikowana równowaga pomiędzy zarządzaniem zasobami, budowaniem budynków oraz
dowodzeniem armii. Centralnym elementem rozgrywki RTS jest zarządzanie różnorodnymi jednostkami, z których każda posiada unikalne zdolności i rolę na polu bitwy.

Gry te działają w dynamicznym środowisku czasu rzeczywistego, co odróżnia je od strategii turowych TBS (ang. \textit{turn-based strategy}). Ten model wymusza szybkie podejmowanie decyzji
i wymaga od gracza podzielności uwagi. Strategie czasu rzeczywistego zwykle też posiadają bardziej złożoną konstrukcję mapy, zawierającą skomplikowane
rozłożenie celów i przeszkód, natomiast mapy w strategiach turowych zazwyczaj wykorzystują ruch oparty o siatkę, co zapewnia dużo bardziej
zorganizowane środowisko rozgrywki.

W grach strategicznych czasu rzeczywistego w trybie kampanii zachowanie przeciwników jest zaprojektowane z myślą o zanurzeniu gracza w fabularnej opowieści, jednocześnie
prezentując wyzwania związane z rozgrywką. Akcje wykonywane przez sztuczną inteligencję są dostosowane do celów danej misji, co pozwala
na dopasowanie do obowiązującej narracji.

Tego typu produkcje pozwalają również na wciągającą rozgrywkę w trybie wieloosobowym, w którym gracze mogą rywalizować między sobą, jak również
współpracować w celu pokonania wspólnego przeciwnika kontrolowanego przez sztuczną inteligencję. Na ten styl rozgrywki jest nakładany istotny nacisk
w wielu grach tego gatunku, poprzez zapewnienie zróżnicowanych grywalnych frakcji, zachowując przy tym symetrię balansu rozgrywki.

Z uwagi na cechę gier gatunku RTS polegającej na ciągłym przepływie czasu, gracze są zmuszeni
do sprawnego reagowania na ciągle rozwijające się wydarzenia. Mijający czas
wywiera presję na użytkowników, wymagając natychmiastowej reakcji w podejmowanych decyzjach.
Z tego powodu w grach z tego gatunku efektywne zarządzanie czasem jest kluczowym elementem
wymaganym do osiągnięcia oczekiwanych rezultatów. Zazwyczaj czas upływa w sposób jednostajny, co oznacza, że
nie mogą spowolnić lub zatrzymać tempa rozgrywki, aby odetchnąć lub przemyśleć swój plan działania.

Ogólnie rzecz biorąc, gatunek RTS oddaje dreszcz emocji związanych z prowadzeniem działań strategicznych, co wymaga połączenia zaradności,
umiejętności przewidywania oraz sprytnego podejmowania decyzji na szybkim i stale zmieniającym się polu bitwy.
"Wewnątrz tej ogólnej definicji znajdują się pewne podgatunki. [...] Tradycyjne gry RTS składają się z elementów budowania bazy, zarządzania zasobami i drzew technologicznych. Do tej kategorii pasują gry takie jak Command \& Conquer, StarCraft i Age of Empires."\cite{stateoftherts}
Do gatunku gier strategii czasu rzeczywistego należą również takie tytuły jak Company of Heroes 2\footnote{\url{https://www.companyofheroes.com}} kanadyjskiego studia 
Relic Entertainment wyprodukowane w 2013 roku. Producent jest znany również z gry Warhammer 40,000: Dawn of War\footnote{\url{https://www.dawnofwar.com}} powstałego w 2006 roku.
Natomiast najbardziej przełomową grą tego gatunku jest gra Starcraft\footnote{\url{https://starcraft.blizzard.com}} wydana przez Blizzard Entertainment w 1998 roku, jest to jedna z najlepiej sprzedających się gier w historii gatunku\cite{rtslist}.
