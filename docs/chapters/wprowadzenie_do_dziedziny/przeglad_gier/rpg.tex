\subsection{Gry cRPG (Bogna Lew)}\label{ss:rpg}
Komputerowe gry fabularne cRPG (ang. \textit{computer role-playing game}) czerpią inspirację z tradycyjnych gier fabularnych. Komputerowa
gra fabularna opowiada pewną historię, w której gracz wciela się w wybraną postać bądź drużynę i
za ich pomocą eksploruje świat gry, wykonując zadania oraz rozwiązując łamigłówki. Gry tego typu cechują się zwykle
nieliniową, z góry zdefiniowaną fabułą. Czasami jest ona podzielona na rozdziały, które gracz musi kolejno ukończyć, aby
móc kontynuować rozgrywkę.

Istotnym elementem gier fabularnych jest rozwój postaci. W trakcie rozgrywki gracz może kreować swojego bohatera
poprzez podnoszenie jego współczynników, dodawanie mu nowych umiejętności, czy zmienianie ekwipunku. Gra umożliwia
użytkownikowi zmianę statystyk swojej postaci za każdym razem, gdy osiągnie następny poziom doświadczenia. Zwyczajowo
punkty doświadczenia są graczowi przyznawane po wykonaniu kolejnych zadań bądź pokonaniu przeciwników.

Komputerowe gry fabularne zwykle cechują się otwartym światem, czasem zawierającym niewielkie ograniczenia, co umożliwia
graczowi swobodne eksplorowanie świata. Aby ułatwić użytkownikowi nawigację i podróżowanie, komputerowe gry fabularne
udostępniają mu system map, który pokazuje lokalizację głównych elementów świata. W trakcie swojej wędrówki gracz może
wchodzić w interakcję z postaciami niezależnymi, od których może dostać zadania do zrealizowania. Może je wykonywać na
różne sposoby, dzięki czemu może wpływać na fabułę gry.

Ważnym aspektem gier fabularnych są przeciwnicy, z którymi gracz może walczyć. W trakcie potyczki może wykorzystywać
specjalne umiejętności swojej postaci, wykonywać ataki bądź przemieszczać się. W zależności od przyjętej przez twórców
formy, walka może mieć charakter turowy lub przebiegać w czasie rzeczywistym. Pokonanie przeciwników przynosi pewne
korzyści w postaci zdobywania punktów doświadczenia bądź łupu.

Do gatunku komputerowych gier fabularnych należą między innymi takie tytuły jak \textit{Divinity: Original Sin II}\footnote{\url{https://divinity.game/}} wydane w 2017
roku przez Larian Studios, wyprodukowane przez studio CD Project Red w 2015 \textit{Wiedźmin 3: Dziki Gon}\footnote{\url{https://www.thewitcher.com/us/pl/witcher3}}, czy gra \textit{The Elder
Scrolls V: Skyrim}\footnote{\url{https://elderscrolls.bethesda.net/pl}} studia Bethesda Softworks wydane w 2016 roku.
