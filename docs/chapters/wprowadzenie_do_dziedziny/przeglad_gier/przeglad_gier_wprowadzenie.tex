\section{Przegląd wybranych gatunków gier komputerowych}
W niniejszym podrozdziale zostały przedstawione wybrane gatunki gier komputerowych, które są najistotniejsze z punktu widzenia
projektu. Przede wszystkim zostały opisane gry typu strategii czasu rzeczywistego oraz strategiczne gry turowe TBS (ang.
\textit{turn-based strategy}. Ma to na celu zaprezentowanie podstawowych różnic pomiędzy tymi gatunkami.

"Gatunek określa graczowi, w jakiego typu grę będzie grał i jest bardzo przydatnym sposobem klasyfikacji gier"\cite{practical_game_design}.
Podział gier na gatunki pozwala na określenie podstawowych cech gry i daje ogólne pojęcie o jej charakterze. "Dlatego w
grach gatunek niesie więcej informacji niż temat i sceneria"\cite{practical_game_design}. Z tego powodu aby zrozumieć cel projektu ważne jest zauważenie podobieństw i różnic pomiędzy kluczowymi dla niego gatunkami.

\subsection{Specyfika gier RTS (Bartosz Strzelecki)}\label{ss:rts}

Strategie czasu rzeczywistego RTS (ang. \textit{real-time strategy}) odróżnia się od innych gatunków występowaniem zarówno strategicznych, jak i taktycznych
elementów rozgrywki. Podstawowe założenia gry RTS to skomplikowana równowaga pomiędzy zarządzaniem zasobami, budowaniem budynków oraz
dowodzeniem armii. Centralnym elementem rozgrywki RTS jest zarządzanie różnorodnymi jednostkami, z których każda posiada unikalne zdolności i rolę na polu bitwy.

Gry te działają w dynamicznym środowisku czasu rzeczywistego, co odróżnia je od strategii turowych TBS (ang.  \textit{turn-based stratrgy}). Ten model wymusza szybkie podejmowanie decyzji
i wymaga od gracza podzielności uwagi. Strategie czasu rzeczywistego zwykle też posiadają bardziej złożoną konstrukcję mapy, zawierającą skomplikowane
rozłożenia celów i przeszkód, natomiast mapy w strategiach turowych zazwyczaj wykorzystują ruch oparty o siatkę, co zapewnia dużo bardziej
zorganizowane środowisko rozgrywki.

W grach strategicznych czasu rzeczywistego w trybie kampanii zachowanie przeciwników jest zaprojektowane z myślą o zanurzeniu gracza w fabularnej opowieści, jednocześnie
prezentując wyzwania związane z rozgrywką. Akcje wykonywane przez sztuczną inteligencję są dostosowane do celów danej misji, co pozwala
na dopasowanie do obowiązującej narracji.

Tego typu produkcje pozwalają również na wciągającą rozgrywkę w trybie wieloosobowym, w którym gracze mogą rywalizować między sobą, jak również
współpracować w celu pokonania wspólnego przeciwnika kontrolowanego przez sztuczną inteligencję. Na ten styl rozgrywki jest nakładany istotny nacisk
w wielu grach tego gatunku, poprzez zapewnienie zróżnicowanych grywalnych frakcji, zachowując przy tym symetrię balansu rozgrywki.

Ogólnie rzecz biorąc, gatunek RTS oddaje dreszcz emocji związanych z prowadzeniem działań strategicznych, co wymaga połączenia zaradności,
przewidywania i sprytnego podejmowania decyzji na szybkim i stale zmieniającym się polu bitwy.

Do gatunku gier strategii czasu rzeczywistego należą takie tytuły jak:
\begin{itemize}
  \item Company of Heroes 2
  \item Warcraft III
  \item StarCraft II
  \item Warhammer 40000: Dawn of War
\end{itemize}

\subsection{Specyfika komputerowych gier fabularnych (Bogna Lew)}\label{ss:rpg}
Komputerowe gry fabularne czerpią inspirację z tradycyjnych gier fabularnych. Komputerowa
gra fabularna opowiada pewną historię, w której gracz wciela się w wybraną postać bądź drużynę i
za ich pomocą eksploruje świat gry, wykonując zadania oraz rozwiązując łamigłówki. Gry tego typu cechują się zwykle
nieliniową, z góry zdefiniowaną fabułą. Czasami jest ona podzielona na rozdziały, które gracz musi kolejno ukończyć, aby
móc kontynuować rozgrywkę.

Istotnym elementem gier fabularnych jest rozwój postaci. W trakcie rozgrywki gracz może kreować swojego bohatera
poprzez podnoszenie jego współczynników, dodawanie mu nowych umiejętności, czy zmienianie ekwipunku. Gra umożliwia
użytkownikowi zmianę statystyk swojej postaci za każdym razem, gdy osiągnie następny poziom doświadczenia. Zwyczajowo
punkty doświadczenia są graczowi przyznawane po wykonaniu kolejnych zadań bądź pokonaniu przeciwników.

Komputerowe gry fabularne zwykle cechują się otwartym światem, czasem zawierającym niewielkie ograniczenia, co umożliwia
graczowi swobodne eksplorowanie świata. Aby ułatwić użytkownikowi nawigację i podróżowanie, komputerowe gry fabularne
udostępniają mu system map, który pokazuje lokalizację głównych elementów świata. W trakcie swojej wędrówki gracz może
wchodzić w interakcję z postaciami niezależnymi, od których może dostać zadania do zrealizowania. Może je wykonywać na
różne sposoby, dzięki czemu może wpływać na fabułę gry.

Ważnym aspektem gier fabularnych są przeciwnicy, z którymi gracz może walczyć. W trakcie potyczki może wykorzystywać
specjalne umiejętności swojej postaci, wykonywać ataki bądź przemieszczać się. W zależności od przyjętej przez twórców
formy, walka może mieć charakter turowy lub przebiegać w czasie rzeczywistym. Pokonanie przeciwników przynosi pewne
korzyści w postaci zdobywania punktów doświadczenia bądź łupu.

Do gatunku komputerowych gier fabularnych należą między innymi takie tytuły jak Divinity: Original Sin II wydane w 2017
roku przez Larian Studios, wyprodukowane przez studio CD Project Red w 2015 Wiedźmin 3: Dziki Gon, czy gra The Elder
Scrolls V: Skyrim studia Bathesda Softworks wydane w 2016 roku.

\subsection{Gry TBS (Zofia Sosińska)}\label{ss:tbs}
Strategiczne gry turowe TBS (ang. \textit{turn-based strategy}) są jednym z wielu popularnych gatunków gier komputerowych, ale w swej podstawowej mechanice działania są silnie zbliżone do
gier planszowych. Te bardzo często trzymają się określonego schematu: konkretny gracz wykonuje jeden z dostępnych ruchów, a jego współgracze patrzą i czekają 
na swoją kolej. Zabronione jest wtedy przerywanie, czy przeszkadzanie mu i ma pełną autonomię podjęcia decyzji. Gdy dana osoba zakończy swój ruch, następny w kolejce gracz
dostaje swoją szansę. Opisany schemat nosi nazwę "tury" i jest to, inaczej mówiąc, mechanika wymuszająca na graczach skolejkowanie się, przyznając każdemu po kolei
prawo do podjęcia ściśle ustalonych akcji. Strategiczne gry turowe korzystają z tego schematu pełniąc rolę semafora, który zczytuje ruchy tylko jednego gracza.

Oprawą fabularną takich gier często jest jakaś forma bitwy. W wersji jednoosobowej użytkownik może być dowódcą pewnej drużyny, która napotykać będzie przeciwników. Po 
rozpoczęciu walki postacie ustawiane są w kolejkę. Sortowanie może odbywać się względem wylosowanych wartości, lub chociażby według umiejętności, takich jak np. szybkości. Każda postać
 dostaje swoją kolej na wykonanie konkretnej liczby ściśle określonych ruchów. To jak wiele może zostać podjętych jest ewaluowane według przyznanych im punktów 
 trudności, tego ile postać ma energii, albo po prostu jest to z góry ustalone na przykładowo jeden. Pośród tur znajdują się też te przeciwników, których działaniami będzie 
 kierować sztuczna inteligencja. W wersji wieloosobowej tylko jeden wykonuje ruch, a reszta ma zamrożony stan gry, obserwując decyzje współgracza.

Gry tego typu często wspierają rozwój drużyny i umiejętności, dając użytkownikowi dostęp do coraz to nowych mechanik. Może on często wybierać specjalizacje postaci tak, aby 
pokrywały się z jego taktyką. Sedno strategii tych gier leży w odpowiednim przyznaniu umiejętności, a następnie wykorzystaniu ich podczas walki w jak najoptymalniejszy sposób.

Do gatunku strategicznych gier turowych należą m.in. takie tytuły jak Heroes of Might and Magic III HD Edition\footnote{\url{https://www.ubisoft.com/pl-pl/game/heroes-of-might-and-magic-3-hd}} z 2015 wyprodukowana przez DotEmu, 
Civilization VI: Gathering Storm\footnote{\url{https://civilization.com/pl-PL/civilization-6-gathering-storm/}} z 2019 roku studia Firaxis Games oraz Total War: Warhammer\footnote{\url{https://www.totalwar.com/games/warhammer/}} wydana przez Strategic Simulations, Inc. w 2016 roku.

\subsection{Porównanie wybranych gatunków gier. (Bogna Lew, Bartosz Strzelecki)}

\begin{table}[h]
\caption{Porównanie gatunków gier.}
\begin{center}
\begin{tabular}{| m{11em} | m{10em} | m{10em} | m{10em}|} 
 \hline
 Aspekt & RTS & RPG & TBS \\
 \hline \hline
 Punkt widzenia gracza & Widok z lotu ptaka & Widok zza postaci & Widok z lotu ptaka \\
 \hline
 Turowy przebieg rozgrywki & Nie & Nie & Tak \\
 \hline
 Fabuła & Liniowe scenariusze & Zwykle nieliniowa & losowo generowane warunki startowe \\
 \hline
 Świat & ograniczony & głównie otwarty & ograniczony \\
 \hline
 Styl rozgrywki & Głównie wieloosobowy & Jednoosobowy & Wieloosobowy i jednoosobowy \\
 \hline
 Mapa & Gęsta siatka & Brak siatki & Rzadka siatka \\
 \hline
\end{tabular}
\end{center}
\label{fig:teng} 
\end{table}

