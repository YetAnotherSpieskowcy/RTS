\subsection{Gry TBS (Zofia Sosińska)}\label{ss:tbs}
Strategiczne gry turowe TBS (ang. \textit{turn-based strategy}) są jednym z wielu popularnych gatunków gier komputerowych, ale w swej podstawowej mechanice działania są silnie zbliżone do
gier planszowych. Te bardzo często trzymają się określonego schematu: konkretny gracz wykonuje jeden z dostępnych ruchów, a jego współgracze patrzą i czekają 
na swoją kolej. Zabronione jest wtedy przerywanie, czy przeszkadzanie mu i ma pełną autonomię podjęcia decyzji. Gdy dana osoba zakończy swój ruch, następny w kolejce gracz
dostaje swoją szansę. Opisany schemat nosi nazwę "tury" i jest to, inaczej mówiąc, mechanika wymuszająca na graczach skolejkowanie się, przyznając każdemu po kolei
prawo do podjęcia ściśle ustalonych akcji. Strategiczne gry turowe korzystają z tego schematu pełniąc rolę semafora, który zczytuje ruchy tylko jednego gracza.

Oprawą fabularną takich gier często jest jakaś forma bitwy. W wersji jednoosobowej użytkownik może być dowódcą pewnej drużyny, która napotykać będzie przeciwników. Po 
rozpoczęciu walki postacie ustawiane są w kolejkę. Sortowanie może odbywać się względem wylosowanych wartości, lub chociażby według umiejętności, takich jak np. szybkości. Każda postać
 dostaje swoją kolej na wykonanie konkretnej liczby ściśle określonych ruchów. To jak wiele może zostać podjętych jest ewaluowane według przyznanych im punktów 
 trudności, tego ile postać ma energii, albo po prostu jest to z góry ustalone na przykładowo jeden. Pośród tur znajdują się też te przeciwników, których działaniami będzie 
 kierować sztuczna inteligencja. W wersji wieloosobowej tylko jeden wykonuje ruch, a reszta ma zamrożony stan gry, obserwując decyzje współgracza.

Gry tego typu często wspierają rozwój drużyny i umiejętności, dając użytkownikowi dostęp do coraz to nowych mechanik. Może on często wybierać specjalizacje postaci tak, aby 
pokrywały się z jego taktyką. Sedno strategii tych gier leży w odpowiednim przyznaniu umiejętności, a następnie wykorzystaniu ich podczas walki w jak najoptymalniejszy sposób.

Do gatunku strategicznych gier turowych należą m.in. takie tytuły jak \textit{Heroes of Might and Magic III HD Edition}\footnote{\url{https://www.ubisoft.com/pl-pl/game/heroes-of-might-and-magic-3-hd}} z 2015 wyprodukowana przez DotEmu,
\textit{Civilization VI: Gathering Storm}\footnote{\url{https://civilization.com/pl-PL/civilization-6-gathering-storm/}} z 2019 roku studia Firaxis Games oraz \textit{Total War: Warhammer}\footnote{\url{https://www.totalwar.com/games/warhammer/}} wydana przez Strategic Simulations, Inc. w 2016 roku.
