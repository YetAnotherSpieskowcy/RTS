\subsection{Specyfika strategicznych gier turowych (Zofia Sosińska)}
Strategiczne gry turowe są jednym z wielu popularnych gatunków gier komputerowych, ale w swej podstawowej mechanice działania są silnie zbliżone do
gier planszowych. Te bardzo często trzymają się określonego schematu: konkretny gracz wykonuje jeden z dostępych ruchów, a jego współgracze patrzą i czekają 
na swoja kolej. Zabronione jest wtedy przerywanie, czy przeszkadzaniu mu i ma pełną autonomię podjęcia decyzji. Gdy dana osoba zakończy ruch następny w kolejce gracz 
dostaje swoją szansę. Opisany schemat nosi nazwę "tury". Inaczej mówiąc jest to mechanika wymuszająca na graczach skolejkowanie się, przyznając każdemu po koleji  
prawo do podjęcia ściśle ustalonych akcji. Strategiczne gry turowe korzystają z tego schematu pełniąc rolę semafora, który zczytuje ruchy tylko jednego gracza.

Oprawą fabularna takich gier często jest jakaś forma bitwy. W wersji jednoosobowej użytkownik może być dowódzcą pewnej drużyny, która napotykać będzie przeciwników. Po 
rozpoczęciu walki postacie ustawiane są w kolejkę. Sortowanie może obywać się względem wylosowanych wartości, lub chociażby według umiejętności np. szybkości. Każda postać
 dostaje swoją kolej na wykonanie konkretnej liczby ściśle określonych ruchów. To jak wiele może może zostać podjętych jest ewaluowane według przyznanych im punktów 
 trudności, tego ile postać ma energii, albo po prosty jest to z góry ustalone na 

Do gatunku strategicznych gier turowych należą takie tytuły jak:
\begin{itemize}
  \item seria Heroes of Might and Magic;
  \item seria Civilization;
  \item seria Worms;
  \item seria Total War.
\end{itemize}
