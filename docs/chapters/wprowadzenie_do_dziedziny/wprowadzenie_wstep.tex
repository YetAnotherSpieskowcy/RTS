\chapter{Przegląd strategii czasu rzeczywistego}

W niniejszym rozdziale zostały przedstawione wybrane mechaniki wykorzystywane w grach typu RTS. Omówione zostało ich
działanie na podstawie przykładów z wybranych gier. Celem tego jest obrazowe zaprezentowanie ich integralności
w grach oraz wpływu na rozgrywkę.

Na odbiór gry wpływają zawarte w niej mechaniki. Twórcy bardzo często stosują w nich uproszczenia, co ma na celu
ułatwienie graczowi wykonywanie zadań. Ma to jednak ogromny wpływ na zachowanie realizmu w grach, zwłaszcza tych opartych
na wydarzeniach historycznych oraz prawdziwym świecie.

\section{Przegląd wybranych gatunków gier komputerowych}
W niniejszym podrozdziale zostały przedstawione wybrane gatunki gier komputerowych, które są najistotniejsze z punktu widzenia
projektu. Przede wszystkim zostały opisane gry typu strategii czasu rzeczywistego oraz strategiczne gry turowe TBS (ang.
\textit{turn-based strategy}. Ma to na celu zaprezentowanie podstawowych różnic pomiędzy tymi gatunkami.

"Gatunek określa graczowi, w jakiego typu grę będzie grał i jest bardzo przydatnym sposobem klasyfikacji gier"\cite{practical_game_design}.
Podział gier na gatunki pozwala na określenie podstawowych cech gry i daje ogólne pojęcie o jej charakterze. "Dlatego w
grach gatunek niesie więcej informacji niż temat i sceneria"\cite{practical_game_design}. Z tego powodu aby zrozumieć cel projektu ważne jest zauważenie podobieństw i różnic pomiędzy kluczowymi dla niego gatunkami.

\subsection{Specyfika gier RTS (Bartosz Strzelecki)}\label{ss:rts}

Strategie czasu rzeczywistego RTS (ang. \textit{real-time strategy}) odróżnia się od innych gatunków występowaniem zarówno strategicznych, jak i taktycznych
elementów rozgrywki. Podstawowe założenia gry RTS to skomplikowana równowaga pomiędzy zarządzaniem zasobami, budowaniem budynków oraz
dowodzeniem armii. Centralnym elementem rozgrywki RTS jest zarządzanie różnorodnymi jednostkami, z których każda posiada unikalne zdolności i rolę na polu bitwy.

Gry te działają w dynamicznym środowisku czasu rzeczywistego, co odróżnia je od strategii turowych TBS (ang.  \textit{turn-based stratrgy}). Ten model wymusza szybkie podejmowanie decyzji
i wymaga od gracza podzielności uwagi. Strategie czasu rzeczywistego zwykle też posiadają bardziej złożoną konstrukcję mapy, zawierającą skomplikowane
rozłożenia celów i przeszkód, natomiast mapy w strategiach turowych zazwyczaj wykorzystują ruch oparty o siatkę, co zapewnia dużo bardziej
zorganizowane środowisko rozgrywki.

W grach strategicznych czasu rzeczywistego w trybie kampanii zachowanie przeciwników jest zaprojektowane z myślą o zanurzeniu gracza w fabularnej opowieści, jednocześnie
prezentując wyzwania związane z rozgrywką. Akcje wykonywane przez sztuczną inteligencję są dostosowane do celów danej misji, co pozwala
na dopasowanie do obowiązującej narracji.

Tego typu produkcje pozwalają również na wciągającą rozgrywkę w trybie wieloosobowym, w którym gracze mogą rywalizować między sobą, jak również
współpracować w celu pokonania wspólnego przeciwnika kontrolowanego przez sztuczną inteligencję. Na ten styl rozgrywki jest nakładany istotny nacisk
w wielu grach tego gatunku, poprzez zapewnienie zróżnicowanych grywalnych frakcji, zachowując przy tym symetrię balansu rozgrywki.

Ogólnie rzecz biorąc, gatunek RTS oddaje dreszcz emocji związanych z prowadzeniem działań strategicznych, co wymaga połączenia zaradności,
przewidywania i sprytnego podejmowania decyzji na szybkim i stale zmieniającym się polu bitwy.

Do gatunku gier strategii czasu rzeczywistego należą takie tytuły jak:
\begin{itemize}
  \item Company of Heroes 2
  \item Warcraft III
  \item StarCraft II
  \item Warhammer 40000: Dawn of War
\end{itemize}

\subsection{Specyfika komputerowych gier fabularnych (Bogna Lew)}\label{ss:rpg}
Komputerowe gry fabularne czerpią inspirację z tradycyjnych gier fabularnych. Komputerowa
gra fabularna opowiada pewną historię, w której gracz wciela się w wybraną postać bądź drużynę i
za ich pomocą eksploruje świat gry, wykonując zadania oraz rozwiązując łamigłówki. Gry tego typu cechują się zwykle
nieliniową, z góry zdefiniowaną fabułą. Czasami jest ona podzielona na rozdziały, które gracz musi kolejno ukończyć, aby
móc kontynuować rozgrywkę.

Istotnym elementem gier fabularnych jest rozwój postaci. W trakcie rozgrywki gracz może kreować swojego bohatera
poprzez podnoszenie jego współczynników, dodawanie mu nowych umiejętności, czy zmienianie ekwipunku. Gra umożliwia
użytkownikowi zmianę statystyk swojej postaci za każdym razem, gdy osiągnie następny poziom doświadczenia. Zwyczajowo
punkty doświadczenia są graczowi przyznawane po wykonaniu kolejnych zadań bądź pokonaniu przeciwników.

Komputerowe gry fabularne zwykle cechują się otwartym światem, czasem zawierającym niewielkie ograniczenia, co umożliwia
graczowi swobodne eksplorowanie świata. Aby ułatwić użytkownikowi nawigację i podróżowanie, komputerowe gry fabularne
udostępniają mu system map, który pokazuje lokalizację głównych elementów świata. W trakcie swojej wędrówki gracz może
wchodzić w interakcję z postaciami niezależnymi, od których może dostać zadania do zrealizowania. Może je wykonywać na
różne sposoby, dzięki czemu może wpływać na fabułę gry.

Ważnym aspektem gier fabularnych są przeciwnicy, z którymi gracz może walczyć. W trakcie potyczki może wykorzystywać
specjalne umiejętności swojej postaci, wykonywać ataki bądź przemieszczać się. W zależności od przyjętej przez twórców
formy, walka może mieć charakter turowy lub przebiegać w czasie rzeczywistym. Pokonanie przeciwników przynosi pewne
korzyści w postaci zdobywania punktów doświadczenia bądź łupu.

Do gatunku komputerowych gier fabularnych należą między innymi takie tytuły jak Divinity: Original Sin II wydane w 2017
roku przez Larian Studios, wyprodukowane przez studio CD Project Red w 2015 Wiedźmin 3: Dziki Gon, czy gra The Elder
Scrolls V: Skyrim studia Bathesda Softworks wydane w 2016 roku.

\subsection{Gry TBS (Zofia Sosińska)}\label{ss:tbs}
Strategiczne gry turowe TBS (ang. \textit{turn-based strategy}) są jednym z wielu popularnych gatunków gier komputerowych, ale w swej podstawowej mechanice działania są silnie zbliżone do
gier planszowych. Te bardzo często trzymają się określonego schematu: konkretny gracz wykonuje jeden z dostępnych ruchów, a jego współgracze patrzą i czekają 
na swoją kolej. Zabronione jest wtedy przerywanie, czy przeszkadzanie mu i ma pełną autonomię podjęcia decyzji. Gdy dana osoba zakończy swój ruch, następny w kolejce gracz
dostaje swoją szansę. Opisany schemat nosi nazwę "tury" i jest to, inaczej mówiąc, mechanika wymuszająca na graczach skolejkowanie się, przyznając każdemu po kolei
prawo do podjęcia ściśle ustalonych akcji. Strategiczne gry turowe korzystają z tego schematu pełniąc rolę semafora, który zczytuje ruchy tylko jednego gracza.

Oprawą fabularną takich gier często jest jakaś forma bitwy. W wersji jednoosobowej użytkownik może być dowódcą pewnej drużyny, która napotykać będzie przeciwników. Po 
rozpoczęciu walki postacie ustawiane są w kolejkę. Sortowanie może odbywać się względem wylosowanych wartości, lub chociażby według umiejętności, takich jak np. szybkości. Każda postać
 dostaje swoją kolej na wykonanie konkretnej liczby ściśle określonych ruchów. To jak wiele może zostać podjętych jest ewaluowane według przyznanych im punktów 
 trudności, tego ile postać ma energii, albo po prostu jest to z góry ustalone na przykładowo jeden. Pośród tur znajdują się też te przeciwników, których działaniami będzie 
 kierować sztuczna inteligencja. W wersji wieloosobowej tylko jeden wykonuje ruch, a reszta ma zamrożony stan gry, obserwując decyzje współgracza.

Gry tego typu często wspierają rozwój drużyny i umiejętności, dając użytkownikowi dostęp do coraz to nowych mechanik. Może on często wybierać specjalizacje postaci tak, aby 
pokrywały się z jego taktyką. Sedno strategii tych gier leży w odpowiednim przyznaniu umiejętności, a następnie wykorzystaniu ich podczas walki w jak najoptymalniejszy sposób.

Do gatunku strategicznych gier turowych należą m.in. takie tytuły jak Heroes of Might and Magic III HD Edition\footnote{\url{https://www.ubisoft.com/pl-pl/game/heroes-of-might-and-magic-3-hd}} z 2015 wyprodukowana przez DotEmu, 
Civilization VI: Gathering Storm\footnote{\url{https://civilization.com/pl-PL/civilization-6-gathering-storm/}} z 2019 roku studia Firaxis Games oraz Total War: Warhammer\footnote{\url{https://www.totalwar.com/games/warhammer/}} wydana przez Strategic Simulations, Inc. w 2016 roku.

\subsection{Porównanie wybranych gatunków gier. (Bogna Lew, Bartosz Strzelecki)}

\begin{table}[h]
\caption{Porównanie gatunków gier.}
\begin{center}
\begin{tabular}{| m{11em} | m{10em} | m{10em} | m{10em}|} 
 \hline
 Aspekt & RTS & RPG & TBS \\
 \hline \hline
 Punkt widzenia gracza & Widok z lotu ptaka & Widok zza postaci & Widok z lotu ptaka \\
 \hline
 Turowy przebieg rozgrywki & Nie & Nie & Tak \\
 \hline
 Fabuła & Liniowe scenariusze & Zwykle nieliniowa & losowo generowane warunki startowe \\
 \hline
 Świat & ograniczony & głównie otwarty & ograniczony \\
 \hline
 Styl rozgrywki & Głównie wieloosobowy & Jednoosobowy & Wieloosobowy i jednoosobowy \\
 \hline
 Mapa & Gęsta siatka & Brak siatki & Rzadka siatka \\
 \hline
\end{tabular}
\end{center}
\label{fig:teng} 
\end{table}


\section{Model sztucznej inteligencji przeciwników w grach Warcraft III i StarCraft II. Bartosz Strzelecki}
Sztuczna inteligencja przeciwników w grach takich jak Warcraft III lub StarCraft II, przede wszystkim w trybie kampanii,
jest odpowiedzialna za kontrolowanie wrogich jednostek w celu zaoferowania graczowi wyzwania. Głównym zadaniem AI jest zasymulowanie
strategicznych decyzji i wydajne zarządzanie zasobami.
AI podejmuje decyzję na podstawie predefiniowanych zasad i algorytmów. Analizuje sytuację, w której się znajduje, biorąc pod uwagę
siłę swojej własnej armii, siłę armii gracza oraz specjalne zdolności jednostek i środowisko, w którym toczy się gra.
Ta analiza pozwala komputerowi na podejmowanie strategicznych decyzji jak na przykład, kiedy atakować, bronić się, eksplorować oraz rozszerzać swoje terytorium.
W tych grach sztuczna inteligencja może przybrać jedną z kilku wariantów wynikających z poziomu trudności. Wyższe poziomy
dają przeciwnikowi przewagę takie jak wydajniejsze zbieranie zasobów lub szybsza produkcja jednostek.

W grze Warcraft III w trybie kampanii zachowanie przeciwników jest zaprojektowane z myślą o zanurzeniu gracza w fabularnej opowieści, jednocześnie
prezentując wciągające wyzwania związane z rozgrywką. Akcje wykonywane przez sztuczną inteligencję są dostosowane do celów danej misji, co pozwala
na dopasowanie do obowiązującej narracji.
Początkowo przeciwnik konstruuje i rozbudowuje swoją bazę, w celu zgromadzenia odpowiedniej liczby zasobów, szkolenia jednostek i prowadzenia badań.
AI strategicznie rozmieszcza budynki i struktury obronne, aby ochronić swoją fortecę przed najazdami gracza. 
Misje kampanii często też zawierają oskryptowane wydarzenia lub walki, które dodają głębi rozgrywce. Podczas tych starć wroga sztuczna inteligencja
może zachowywać się w specjalny sposób, kontrolując potężne jednostki, do których gracz normalnie nie ma dostępu lub inicjując działania, które popychają
narrację do przodu. Te wyreżyserowane wydarzenia tworzą niezapomniane chwile i jeszcze bardziej wciągają gracza w fabułę kampanii.
Zachowanie wroga w kampanii jest zróżnicowane i obejmuje różnorodne cele misji i scenariusze. Gracze mogą napotkać wrogów, którzy preferują agresywne ataki,
inni skupiają się na strategiach obronnych lub specjalizują się w taktyce hit and run. Sztuczna inteligencja dostosowuje proces podejmowania decyzji do
konkretnych wymagań misji, często wykorzystując ukształtowanie terenu, synergię jednostek i scenariusze wydarzeń, aby rzucić wyzwanie umiejętnościom gracza.
Ogólnie rzecz biorąc, zachowanie wrogów w kampanii Warcraft III ma na celu zapewnienie dynamicznego i wciągającego doświadczenia. Gracze muszą 
wykorzystywać myślenie strategiczne, zarządzanie zasobami i efektywny skład jednostek, aby przezwyciężyć różnorodne strategie stosowane przez wrogą sztuczną inteligencję.

\section{Mechanizm budowania oraz zarządzanie zasobami w grach RTS (Bogna Lew)}\label{s:budowanie}
Jednym z typowych elementów gier strategii czasu rzeczywistego  jest tworzenie baz i budowanie fortyfikacji. Mechanizm
ten stanowi urozmaicenie rozgrywki i wprowadza dodatkowe aspekty możliwe do uwzględnienia w planowaniu strategii. Dla
wielu gier RTS jest wręcz nieodłącznym elementem, który umożliwia graczowi tworzenie i rozwój nowych jednostek,
produkcję zasobów, umacnianie swojej pozycji oraz zwiększanie swojej potęgi.

Mechanizm ten wiąże się z szeregiem ograniczeń, które mają kluczowy wpływ na rozgrywkę. Należą do nich między innymi
ograniczenia związane z ukształtowaniem terenu oraz obecnością innych elementów scenerii. Każde z tych ograniczeń ma
swoje źródło w prawdziwym świecie i mechanizm budowania musi je uwzględniać.

Z tą mechaniką związany jest system zasobów, który jest popularnym aspektem gier z tego gatunku. Wiele gier strategii
czasu rzeczywistego umożliwia graczowi budowanie własnej ekonomii. Uzyskane przez niego zasoby często mogą zostać
wykorzystane przez mechanizm budowania jako koszta budowy obiektów.

Przykładem gry strategii czasu rzeczywistego implementującej tę mechanikę jest \textit{Warhammer 40,000: Dawn of War}\footnote{\url{https://www.dawnofwar.com/}}. Jest to
gra, której realia są osadzone w uniwersum gry bitewnej \textit{Warhammer 40,000}. Udostępnia ona tryb jednoosobowy oraz
wieloosobowy dla maksymalnie sześciu graczy. W pierwszym wariancie gracz wciela się w postać dowódcy
armii Space Marines z Blood Ravens i ma za zadanie zapobiec inwazji Orków. Gra \textit{Warhammer 40,000: Dawn of War} bardzo szybko
zyskała na popularności i oferowała wszystko, co było potrzebne dla tego gatunku. Z tego powodu warto się jej przyjrzeć,
pomimo faktu, że jej realia znacząco odbiegających od tych, w których zostanie osadzona tworzona przez nas gra.

\textit{Warhammer 40,000: Dawn of War} wyróżnia model pozyskiwania surowców. W grze dostępne są dwa rodzaje: Energia, która jest
generowana przez dedykowane do tego budowle oraz Rekwizycja, której szybkość wytwarzania jest uzależniona od kontrolowanych
przez gracza punktów strategicznych. Taka mechanika znacznie lepiej wpasowuje się w realia gry oraz wymusza na użytkowniku
przyjęcie agresywniejszej strategii.

Dodatkowo \textit{Warhammer 40,000: Dawn of War} posiada typowy dla gier RTS mechanizm tworzenia budowli. Gracz ma
do dyspozycji jednostki, którym może zlecić budowę wybranego przez siebie obiektu po poniesieniu kosztów jego utworzenia.
Zanim będzie możliwe rozpoczęcie budowania użytkownik musi wybrać miejsce, w którym budynek powstanie, co robi, przesuwając
jego podgląd po mapie. W tym czasie gra dokonuje walidacji miejsca i informuje gracza czy wybrany obszar jest poprawny,
odpowiednio podświetlając widok budynku. Wybudowanie obiektu nie jest natychmiastowe, co sprawia, że gra lepiej oddaje
realia, w których jest osadzona.

\begin{figure}[h!]
    \centering
    \includegraphics[width=0.9\textwidth]{images/warhammer1.jpg}
    \caption[Budowanie budynku przez dedykowaną do tego jednostkę w grze \textit{Warhammer 40,000: Dawn of War}.]{Budowanie budynku przez dedykowaną do tego jednostkę w grze \textit{Warhammer 40,000: Dawn of War}.\protect\footnotemark}
\end{figure}
\FloatBarrier
\footnotetext{Internet, \url{https://www.youtube.com/watch?v=wNtnGFoVReU}, dostęp: 19.11.2023}
\section{Sterowanie postacią oraz walka (Bogna Lew)}\label{s:walka}
Udostępnianie graczowi jego własnej postaci jest cechą charakterystyczną raczej komputerowych gier fabularnych niż gier
strategii czasu rzeczywistego. Jednakże jest to ciekawe rozwiązanie, które w znaczny sposób urozmaica rozgrywkę oraz
wpływa na stopień zaangażowania użytkownika. Udostępnienie graczowi jego własnej postaci "czyni gracza kreatywnym
elementem działającym wewnątrz dyskursu, który posiada przestrzenny charakter" \cite{olbrzymwcieniu}.

Przykładem gry implementującej łatwy w obsłudze sposób sterowania postacią jest gra \textit{The Elder Scrolls V: Skyrim}. Umożliwia
ona graczowi możliwość poruszania się postacią za pomocą klawiszy \texttt{W}, \texttt{A}, \texttt{S} oraz \texttt{D}. Dodatkowo użytkownik może
zmieniać prędkość swojego bohatera, przytrzymując klawisze \texttt{Alt} bądź \texttt{Ctrl}, które odpowiednio powodują zwolnienie lub
przyspieszenie tempa przemieszczania się. Do rozglądania się wykorzystywana jest mysz, której ruch powoduje obrót postaci
wokół własnej osi. Ataki gracz może wykonać poprzez naciśnięcie prawego bądź lewego przycisku myszy, które powodują
wykonanie akcji odpowiednio lewą lub prawą dłonią. Jest to przykład dbałości o szczegóły, które sprawiają, że gra
jest jeszcze bardziej interesująca i satysfakcjonująca.

Innym tytułem wartym uwagi jest \textit{Kingdom Come: Deliverance}\footnote{\url{https://www.kingdomcomerpg.com/pl}}. Jest to gra osadzona w realiach Europy Środkowej na początku
XV wieku. Godny uwagi jest jej mechanizm walki, ponieważ twórcy skupili się na jak najdokładniejszym oddaniu średniowiecznego
stylu walki. W tym celu skrupulatnie przestudiowali, w jaki sposób władano mieczem w tamtych czasach, a następnie w
pełni oddali to w grze. Wykorzystali do tego tysiące animacji oraz starannie oddali fizykę pojedynków. W efekcie powstał
realistyczny mechanizm walki, który umożliwia graczowi parowanie, zadawanie ciosów oraz blokowanie.

Kolejnym interesującym tytułem jest gra \textit{Mount\&Blade}. Podobnie jak w przypadku \textit{Skyrima}, gracz steruje swoją postacią za
pomocą klawiszy \texttt{W}, \texttt{A}, \texttt{S} oraz \texttt{D}. Co więcej, gracz może wykonywać ataki poprzez naciśnięcie
lewego przycisku myszy, co jest typowym rozwiązaniem w grach,
które w prosty sposób umożliwia graczowi uczestniczenie w potyczkach. Jednakże w przeciwnieństwie do \textit{Skyrima}, w
\textit{Mount\&Blade} ruch myszą nie powoduje obrotu całej postaci,
a jedynie kamery. Zmiana kierunku odbywa się wyłącznie za pomocą klawiszy sterujących. Dzięki temu gracz może zobaczyć, co
się dzieje za jego postacią bez konieczności zmiany kierunku ruchu bądź zatrzymania się. Powoduje to jednak pewne kłopoty z zachowaniem
realizmu, gdyż w prawdziwym świecie nie jest możliwe zobaczenie czegoś bez konieczności zwrócenia się w tę stronę.

Powyższe przykłady obrazują najpopularniejsze rozwiązania umożliwiające sterowanie postacią oraz wykonywanie ataków.
Typowymi narzędziami wykorzystywanymi do poruszania postacią oraz kamerą są myszka i klawisze \texttt{W}, \texttt{A}, \texttt{S} oraz \texttt{D},
natomiast do atakowania - lewy przycisk myszy. Różnice zaczynają się pojawiać w samych mechanikach, gdyż to w jaki
sposób postać gracza będzie się zachowywać zależy od wizji autorów gry.
\section{Pasek najważniejszych informacji w interfejsie użytkownika gry Warcraft3 (Zofia Sosińska)}
Wcześniej wymieniona gra Warcraft III: Reign of Chaos studia Blizzard Entertainment skupia informacje o czasie gry i inwentarzu w cienkim pasku na samej górze ekranu.
Skład elementów tej części jest niezmienny: pola otwierające zakładki, pora dnia oraz trzy wskaźniki zasobów. Pasek jest widoczny
podczas całej rozgrywki, niezależnie od wykonywanych czynności. W tym statycznie zakotwiczonym na górze ekranu elemencie, dynamicznie
zmieniają się jedynie ciągle aktualizujące się informacje. Odpowiednio podmieniana jest tekstura pory dnia, zmieniająca się ze Słońca
na Księżyc oraz stan zasobów, zależnie od wydania, czy pozyskania.


\begin{figure}[htbp]
    \centering
    \includegraphics[width=1.0\textwidth]{images/ui/warcraft3_gorny_pasek.png}
    \caption{Pasek z informacjami w grze Warcraft 3.}\label{fig:Warcraft3}
\end{figure}
\section{System dialogów w grach (Bartosz Strzelecki)}\label{chap:dialogi}
Systemy dialogów w grach wideo kształtują wciągającą historię, umożliwiając graczom dokonywanie wyborów, które wpływają na relacje między postaciami, zadania i narrację gry. 
Odkrywają wiedzę, pogłębiają zaangażowanie i oferują dynamiczną rozgrywkę poprzez różnorodne podejmowanie decyzji.
Dialogi umożliwiają graczowi wpłynięcie na świat, pozwalając mu wybrać, w którą stronę historia będzie podążać.
Gracz w ten sposób rozwiązuje dylematy moralne i może wczuć się w klimat rozgrywki.
"Najpopularniejsze zachodnie gry RPG, takie jak serie Baldur's Gate i Fallout, żyją i umierają dzięki sile dialogów i zdolności gracza do wpływania na postacie niezależne." \cite{dialogue}.

W \textit{Mass Effect 3}\footnote{\url{https://www.ea.com/games/mass-effect/mass-effect-3}} system dialogowy jest integralną częścią rozgrywki i pozwala graczom na prowadzenie rozmów z różnymi postaciami w trakcie gry.
System dialogów w \textit{Mass Effect 3} wykorzystuje interfejs oparty na kole dialogowym (rys. \ref{fig:wheel}), które
przedstawia graczom wiele opcji odpowiedzi podczas rozmów, zwykle podzielonych na kategorie według ich ogólnego tonu lub intencji.
Dostępne opcje często obejmują wybory dyplomatyczne, agresywne bądź konfrontacyjne oraz opcje neutralne lub śledcze.
Podczas niektórych rozmów lub przerywników filmowych gracze mogą przerwać trwającą rozmowę, szybko wybierając określoną opcję dialogową.
Te opcje przerywania pozwalają graczom podjąć natychmiastowe działania lub podjąć decyzje na miejscu, często wpływając na wynik sytuacji lub relacje postaci z innymi.
Ogólnie rzecz biorąc, system dialogowy w \textit{Mass Effect 3} został zaprojektowany tak, aby zapewnić graczom bogate i wciągające doświadczenie w opowiadaniu historii,
pozwalając im kształtować narrację poprzez wybory i interakcje z olbrzymią gamą postaci. System oferuje różnorodne opcje odpowiedzi, dynamiczne rozmowy i konsekwencje,
przyczyniając się do fascynującej i rozgałęzionej narracji gry.

Alternatywnym rozwiązaniem jest to zaprezentowane w grze \textit{Fallout 3}\footnote{\url{https://fallout.bethesda.net/pl}} (rys. \ref{fig:fallout}), które odróżniają przede wszystkim możliwe odpowiedzi gracza.
W tym przypadku użytkownik wybiera z listy gotową odpowiedź, zamiast jedynie tonu jak w grze \textit{Mass Effect}. Pozwala to na większą kontrolę
przez gracza oraz umożliwia uniknięcie sytuacji, w której gracz spodziewał się innej odpowiedzi, wybierając daną opcję dialogową.

\begin{figure}[h]
\centering
\includegraphics[width=0.8\textwidth]{images/me}
\caption{Przykład koła dialogowego w grze \textit{Mass Effect}\protect\footnotemark.}
\label{fig:wheel}
\end{figure}
\footnotetext{Internet \url{https://cdn.vox-cdn.com/thumbor/DP9qp4fQbE88gJMar2WlwAJ1gRg=/0x0:1920x1080/920x0/filters:focal(0x0:1920x1080):format(webp):no_upscale()/cdn.vox-cdn.com/uploads/chorus_asset/file/22515161/5_14_2021_10_51_45_AM_5044r2pc.png} dostęp: 12.09.2023}

\begin{figure}[h]
\centering
\includegraphics[width=0.8\textwidth]{images/fallout3}
\caption{Kadr z gry \textit{Fallout 3} przedstawiający przykładowy dialog\protect\footnotemark.}
\label{fig:fallout}
\end{figure}
\footnotetext{Internet, \url{https://www.gameuidatabase.com/uploads/Fallout-307252021-055357-81413.jpg}, dostęp: 12.09.2023}

\section{Sterowanie jednostkami (Zofia Sosińska)}\label{chap:mb}
Gry z możliwością tworzenia drużyny muszą rozwiązć problem zachowania podwładnych. Program może udostępniać
skomplikowaną sztuczną inteligencję dla wojowników, zawsze rozwiązujac konflikt w optymalny sposób, ale w tym przypadku 
użytkownik staje się obserwatorem, a nie dowodzącym, co odciąży gracza. Innym podejściem może być zaprojektowanie podwładnych jako 
marionetek bez własnej woli, które będą biernie czekać, aż do otrzymania rozkazu. Wtedy jednak gracz musi skupiać się na 
każdym ruchu za równo swojej, jak i przeciwnej drużyny tak, aby w porę móc zareagować na wszystkie zmiany. Takie rozwiązanie 
jest obciżążające dla użytkownika. Przy projektowaniu mechaniki sterowania jednostkami trzeba zachować balans pomiędzy 
dodaniem i odjęciem użytkownikowi zadań, na kórych musi się skupić. Dla każdej gry ta proporcja może być inna, zależnie
od unikalnego charakteru gry.

Przy projektowaniu gry Mount\&Blade studio TaleWorlds Entertainment zdecydowało się zaimplementować mechanikę sterowania jednostkami tak, aby 
dodać do walki element strategii.Gracz bezpośredniokieruje jedynie główną postacią. Podczas walki reszcie może wydawać rozkazy. Poprzez
cyfry 0-4 wybiera grupę, do której się odnosi np. łuczników. Po naciśnięciu konkretnego przycisku, po lewej stronie ekranu pojawia się lista dostępnych komend.
Następnie przez klawisze F1-F11 wydaje konkretny rozkaz np. odwrót. Lista znika, sztuczna inteligencja postaci zajmuje się już samym wykonaniem czynności. 
Gracz nie martwi się, czy jednostki znajdą optymalną drogę, 
będą celować w przeciwników, czy z nimi walczyć. Zachowany jest więc przyjemny balans pomiędzy podejmowaniem kluczowych decyzji,
odciążeniem poprzez wprowadzenie sztucznej inteligencji.

\begin{figure}[h!tbp]
    \centering
    \includegraphics[width=0.9\textwidth]{images/ui/commandsMountBla.png}
    \caption{Wykaz dostępnych rozkazów z gry Mount\&Blade.}\label{fig:MountnBlade}
    \label{fig:mnb}
\end{figure}

\section{Kompas w grze Skyrim}\label{chap:skrm}

The Elder Scrolls V: Skyrim (skrótowo Skyrim) jest to fabularna gra akcji o otwartym świecie, wyprodukowana przez Bethesda Game Studios i wydana przez Bethesda Softworks. Skyrim jest piątym tytułem z serii The Elder Scrolls oraz kontynuacją gry The Elder Scrolls IV: Oblivion. Jest to jednak nowa historią osadzoną w uniwersum The Elder Scrolls, a nie kontynuacją poprzednika. Fabuła opiera się na powrocie smoków do krainy Tamriel. Bohater okazuje się posiadać moc Głosu, dzięki czemu jest w stanie posługiwać się zaklęciami tych starożytnych stworzeń.
	Z punktu tworzonej przez nas gry, szczególnie interesujące jest  bardzo proste i sprytne rozwiązanie, jakim jest pasek przedstawiający pole widzenia gracza. Służy on między innymi jako kompas, ponieważ jedną z jego mechanik jest pokazanie użytkownikowi stron świata, znajdujących się w kierunku, w którym on patrzy. Pasek ułatwia także poruszanie się po świecie sygnalizując położenie wrogów, kompanów i ważnych dla rozgrywki lokalizacji.

    \begin{figure}[htbp]
        \centering
        \includegraphics[width=0.9\textwidth]{images/ui/compassSkyrim.png}
        \caption{Pasek z gry The Elder Scrolls V: Skyrim ukazujący pole widzenia gracza. W tym momencie patrzy on delikatnie w prawo od południa (pokazuje to litera “S”). W jego zasięgu widzenia jest dwoje przeciwników (czerwone kropki) oraz jeden sojusznik (szara ikona po prawej).}\label{fig:Fallout}
    \end{figure}



\section{Przedstawienie dostępnych pułapek do zbudowania w grze Orcs must die!. Zofia Sosińska}\label{chap:omd}

Orcs must die! to strategiczna gra akcji stworzona i wydana przez studio Robot Entertainment. Akcja toczy się w krainie fantasy, w której największym zagrożeniem dla ludzkości są orkowie. W obronie świata przed tymi stworzeniami staje Zakon dowodzony przez Wojennych Magów. Wznieśli oni system fortec odgradzających ojczyznę orków od pozostałych ziem. Zadaniem gracza jest wcielenie się w jednego z  Wojennych Magów i mordowanie nadciągających grup orków za pomocą różnorodnych broni i mechanizmów, które może postawić.
W przejrzysty sposób zostało rozwiązane samo wyświetlenie dostępnych do zbudowania pułapek. Graczowi pokazują się wizerunki mechanizmów, które może postawić. Naciskając odpowiedni numer na klawiaturze, gracz wybiera, co chce zbudować. Po zatwierdzeniu lewym przyciskiem myszki, budynek pojawia się w zaznaczonym miejscu.

\section{Przekazywanie informacji o świecie w grach (Bartosz Strzelecki)}\label{chap:dbd}
Systemy przekazywania graczowi informacji o świecie są nieocenioną pomocą w kształtowaniu dynamiki i dostarczaniu kluczowych informacji.
Jest to podstawowy element rozgrywki, pozwalający na dynamiczne podejmowanie decyzji oraz na koordynację działań w grach wieloosobowych.

Gra \textit{Dead by Daylight}\footnote{\url{https://deadbydaylight.com}} jest asymetryczną grą wieloosobową, w której gracze wcielają się w rolę postaci "ocalałych" starających się uciec
z mapy albo w rolę "zabójcy", którego celem jest niedopuszczenie do tego. Jednym z udostępnianych przez tę grę mechanizmów jest mechanika widzenia przez przeszkody, która stanowi istotny element rozgrywki, zapewniający
dodatkową warstwę strategii. Polega na wyświetlaniu reprezentacji odległych celów, przedmiotów i przeciwników
zakrytych przez przeszkody. Ta zdolność odgrywa kluczową rolę dla obu stron konfliktu.

W przypadku "ocalałych", ta mechanika jest dostępna dzięki atutom i przedmiotom. Mechanika ujawnia lokalizację celów oraz
innych "ocalałych" pozwalając na koordynację i opracowanie strategii wspólnych działań.

I odwrotnie, zdolność "zabójcy" do widzenia aury jest kluczowa dla jego mechaniki rozgrywki.
Aury umożliwiają im śledzenie "ocalałych", zwłaszcza gdy korzystają z ich unikalnych mocy lub specyficznych atutów.
Mechanika ta zwiększa napięcie, ponieważ "ocalali" muszą zachować czujność i strategiczne podejście,
aby uniknąć pola widzenia "zabójcy" lub zakłócić ich zdolność czytania aury, aby uciec i osiągnąć cele.

\begin{figure}[h]
\centering
\includegraphics[width=0.4\textwidth]{images/dbd}
\caption{Przykładowy efekt aury w grze \textit{Dead by Daylight}.}
\end{figure}
\FloatBarrier
% \begin{figure}[h]
% \centering
% \includegraphics[width=0.6\textwidth]{images/aura}
% \caption{Przykładowe elementy widzenia przez horyzont w grze \textit{Dead by Daylight}.}
% \end{figure}

Ogólnie rzecz biorąc, mechanika aury w \textit{Dead by Daylight} służy jako podstawowy element rozgrywki,
który równoważy wymianę informacji między "ocalałymi" a "zabójcami", znacząco przyczyniając się do atmosfery napięcia i strategii w grze.


\section{Modele celowania w grach (Bartosz Strzelecki)}
Celowanie w grach taktycznych jest podstawowym elementem rozgrywki, który ma
olbrzymi wpływ na przebieg rozgrywki. Ten mechanizm jest przede wszystkim wykorzystywany
do wzbogacenia gry o kolejną warstwę decyzji taktycznych i zażądania podejmowanym ryzykiem.
Gracz, przed oddaniem strzału, podejmuje decyzję czy ryzykuje podjęcie strzału o niskiej szansy
na trafienie, czy zdecyduję się na niezawodny atak, lecz wtedy może być wystawiony na ataki przeciwników.

W Phoenix Point modelowanie celności to wieloaspektowy system, który w zawiły sposób definiuje wynik interakcji bojowych. 
Gra wykorzystuje dynamiczny system celowania, który uwzględnia różne elementy, takie jak postawa żołnierza, biegłość w posługiwaniu się bronią, zasięg, 
osłona i warunki środowiskowe, aby określić precyzję strzału. Każdy z tych elementów odgrywa znaczącą rolę w ogólnym obliczeniu trafienia w cel.

W przeciwieństwie do podobnej gry XCOM, gdzie celność jest zamodelowana za pomocą prostej szansy na trafienie, w grze Phoenix Point
trajektoria każdego pocisku obliczana jest osobno. Podczas celowania widoczne są dwa okręgi: wewnętrzny, który reprezentuje miejsce,
w którym znajdzie się 50\% pocisków oraz zewnętrzny, który reprezentuje maksymalny rozrzut broni. W tym przypadku im celniejsza broń tym
okręgi będą mniejsze.

\begin{figure}[h]
\centering
\includegraphics[width=0.6\textwidth]{images/point}
\caption{System celowania występujący w grze Phoenix Point.}
\label{fig:acc}
\end{figure}

Ostatecznie system zaimplementowany w grze Phoenix Point okazuje się dużo bardziej realistyczny i pozwala na utrzymywanie ciągłego napięcia
pomiędzy trafieniem i chybieniem celu. Umożliwiając w ten sposób na strategiczne wyzwania, leżących u podstaw gier tego typu. 

\section{System Nemesis w grach Middle-earth: Shadow of War oraz Middle-earth: Shadow of Mordor (Bogna Lew)}
System Nemesis jest mechaniką generowania przeciwników zaimplementowaną w grze Middle-earth: Shadow of Mordor i
rozwiniętą w Middle-earth: Shadow of War. Jest to mechanizm, który nadaje przeciwnikom ich unikalny charakter oraz
wygląd. W ciągu gry system Nemesis dynamicznie wpływa na postacie, rozwijając je i odpowiednio zmieniając ich wygląd.
Efektem tego jest płynnie zmieniająca się fabuła, która dla każdego gracza będzie inna.

Do podstawowych elementów tej mechaniki należy rozbudowany system relacji pomiędzy postaciami w grze, w tym postaci
gracza. Dodatkowo buduje hierarchię wśród Orków, dynamicznie ją aktualizując w trakcie gry w zależności od śmierci
poszczególnych bohaterów oraz gracza.

\begin{figure}[h!]
    \centering
    \includegraphics[width=0.9\textwidth]{images/system_nemesis.png}
    \caption{Przykładowa postać przeciwnika utworzona przez system Nemesis.}
\end{figure}

System Nemesis jest interesującą mechaniką budującą fabułę w grze. Pozwala na zbudowanie unikatowych przeciwników,
którzy mają bezpośredni wpływ na rozgrywkę i kształtowanie jej przebiegu.
