% Wczytanie szablonu
\documentclass[nostrict]{Szablon}

% Definicja dokumentu
\usepackage[unicode=true]{hyperref}
\usepackage{listings}
\newcommand\PDFtitle{Tytuł pracy}
\newcommand\PDFauthors{Imie Nazwisko}
\hypersetup{
  pdftitle={\PDFtitle},
  pdfauthor={\PDFauthors},
}

% Zmiana czcionki dla symulacji maszynopisu (verbatim)
\makeatletter
\renewcommand{\verbatim@font}{\ttfamily\small}
\makeatother

% Część właściwa pracy
\begin{document}
\chapter*{Streszczenie}
\addcontentsline{toc}{chapter}{Streszczenie}
Tematem pracy inżynierskiej jest realizacja gry z gatunku strategii czasu rzeczywistego osadzonej w realiach historycznych
wybranej epoki. Niniejsza praca zawiera opis przykładowych rozwiązań zastosowanych w niektórych grach dostępnych na rynku
oraz projekt i implementację wytwarzanego programu. Praca skupia się na przedstawieniu zaimplementowanych mechanik oraz
sposobie oddania realiów, w których osadzona jest fabuła opracowanej gry.

Do przygotowania prototypu gry wykorzystano silnik Unity oraz udostępniane przez niego narzędzia. Za ich pomocą
przygotowano podstawowe mechanizmy, typowe dla gier strategii czasu rzeczywistego oraz przykładowe zadania tworzące
fabułę gry. Utworzony prototyp pozwala na dalsze rozwijanie świata gry poprzez dodawanie nowych zadań oraz elementów
rozgrywki. Zagadnienia implementacyjne zostały opisane w rozdziale Implementacja, a efekt finalny w Przebiegu rozgrywki.

Podstawowym celem projektu jest przygotowanie gry, która w jak najdokładniejszy sposób oddawałaby realia wybranej epoki.
Zdecydowaliśmy się na osadzenie fabuły we wczesnym średniowieczu. Główną inspiracją stali się Celtowie, którzy w
tamtych czasach zamieszkiwali tereny współczesnej Irlandii. W ramach projekty przede wszystkim skupiono się na
opracowaniu sposobu nawigacji w grze oraz zachowania przeciwników, stosujących broń oraz słownictwo charakterystyczne
dla realiów historycznych. W rozdziale Projekt została przedstawiona wizja autorów na poszczególne elementy gry.

W celu przygotowania się do projektu i implementacji gry, przeprowadzono przegląd wybranych mechanik w grach dostępnych
na rynku. Niniejsza praca zawiera opis ich działania na podstawie przykładów z istniejących gier oraz sposobu, w jaki
wpływają na ich rozgrywkę. Przedstawione rozwiązania stanowią inspirację dla wytwarzanej gry.

Praca została przygotowana przez trzech współautorów. Bartosz Strzelecki był odpowiedzialny za rozdziały ref{ss:rts}, ...
Jest również współatorem punktów ref{ss:comp} oraz ... Bogna Lew odpowiadała za rozdziały ref{ss:rpg}, ref{s:budowanie},
ref{s:walka}, ... oraz współtworzyła ref{ss:comp}, {chap:introduction}, ... Zofia Sosińska była odpowiedzialna za
pracę nad rozdziałami ref{ss:tbs}. Współtworzyła również rozdział {chap:introduction}.

\chapter*{Abstract}
\addcontentsline{toc}{chapter}{Abstract}  
The topic of the enginnering thesis is a prototype of real-time strategy game set in historical reality of selected era.
This work contains a description of sample solutions used in selected games available on the market
and the design and a overview of the implementation of the produced program. The work focuses on the presentation 
of the implemented mechanics as well as
the way of rendering the realities in which the plot of the developed game is set.

The Unity engine and the tools it provides were used to prepare a prototype of the game.
Employing these tools, basic mechanics typical of real-time strategy games, were prepared,
as well as sample tasks forming the games's plot. The devised prototype allows for 
further development of the game environment by adding new activities and gameplay elements.
The implementation matters are described in the Implementation chapter, and the results in the Gameplay Course.

The fundamental goal of the project is to develop a game that would reflect the reality of the selected era as accurately as possible.
The main inspiration came from the Celts, who at that time inhabited the territory of modern day Ireland. 
The project primarily focused on
developing the means of navigation in the game and the behavior of opponents, using weaponry and vocabulary characteristic to the
historical realities. The Project chapter presents the authors' vision for the various elements of the game.

In preparation for the design and implementation of the game, a review of selected mechanics in games available
on the market were conducted. This paper describes how they work using examples from existing games and how
they affect their gameplay. The solutions presented provide inspiration for the developed game.

\chapter*{Spis treści}
\addcontentsline{toc}{chapter}{Spis treści}

\tableofcontents

\chapter*{Wykaz ważniejszych oznaczeń i skrótów}
\addcontentsline{toc}{chapter}{Wykaz ważniejszych oznaczeń i skrótów}
\begin{description}[style=multiline,leftmargin=3cm]
\item[RTS] Gra strategii czasu rzeczywistego (ang. \textit{real-time strategy}). Jest to gatunek gier, w którym gracz nie jest
ograniczany przez turowość i kolejność ruchów. Wymusza szybsze podejmowanie decyzji, a ich skutki są natychmiastowe.
\item[cRPG] Kompyterowa gra fabularna (ang. \textit{computer role-playing game}). Gatunek gier, w którym gracz wciela
się w postać lub drużynę, przemieszczając się w świecie stworzonym przez autorów gry.
\item[TBS] Strategiczna gra turowa (ang. \textit{turn-based strategy}). Jest to podgatunek gier strategicznych, w którym
gracze wykonują swoje akcje w turach.
\item[AAA (Triple-A)] Termin stosowany w przemyśle gier komputerowych. Służy do określenia wysokobudżetowych gier, od
których oczekuje się wysokiej jakości.
\item[Wielkie odkrycia geograficzne] Termin odnoszący się do odkryć geograficznych, które miały miejsce na przełomie XV
i XVI wieku.
\item[AI] Sztuczna inteligencja (ang. \textit{artificial intelligence}) jest wykorzystywana do imitowania inteligentnego
zachowania postaci niezależnych.
\item[NPC] Termin określa postacie niezależne (ang. \textit{non-playable character}), czyli postacie, które nie są kontrolowane bezpośrednio
przez gracza.
\item[Input Manager] Komponent silnika Unity, który umożliwia definiowanie wirtualnych osi oraz przypisywanie do nich
odpowiednich klawiszy. Poprzez odczyt wartości zwracanej przez oś możliwe jest wyznaczenie odpowiedniej akcji.
\item[UI] Interfejs użytkownika (ang. \textit{user interface}), czyli oprogramowanie umożliwiające użytkownikowi
interakcję z systemem.
\end{description}

\section{Organizacja (Bartosz Strzelecki)}\label{s:org}
W tym podrozdziale przedstawiono harmonogram prac, wraz z ich przewidywanym terminem realizacji.
Ponadto zaprezentowano skład zespołu projektowego, kompetencje ich członków oraz podział zadań.
\subsection{Główne etapy projektu}
\begin{center}
  \begin{tabular}{| m{30em} | m{12em}|} 
  \hline
  Etap & Termin realizacji \\
  \hline\hline
  Wybór i analiza konkretnego kontekstu historycznego. & Kwiecień 2023 \\
  \hline
  Syntetyczny opis modelu postrzegania przestrzeni na podstawie dzieł pisanych, architektury i sztuki. & Kwiecień — Maj 2023 \\
  \hline
  Przegląd rozwiązań stosowanych w grach strategicznych z wybranego okresu oraz dodatkowo mechanizmów z innych gier, które mogłyby być zaadoptowane na potrzeby projektu. & 2, 3 kwartał 2023 \\
  \hline
  Opracowanie fabuły, selekcja postaci i wydarzeń, a także określenie zakresu autonomii świata gry oraz możliwości modyfikowania go przez gracza. & Czerwiec 2023 \\
  \hline
  Opracowanie szczegółowej koncepcji i projektu gry, w tym projekt mechanizmów zawartych w prototypie. & Lipiec 2023 \\
  \hline
  Implementacja poszczególnych funkcjonalności gry. & 4 kwartał 2023 \\ 
  \hline
  Testowanie, weryfikacja założeń i walidacja. & Listopad — Grudzień 2023 \\
  \hline
  Stworzenie dokumentacji przeprowadzonych prac. & 3, 4 kwartał 2023 \\
  \hline
\end{tabular}
\end{center}
Przewidywany termin zakończenia prac nad projektem to grudzień 2023 roku.
\begin{figure}[htbp]
    \centering
    \includegraphics[width=1\textwidth]{uml/Harmonogram}
    \caption{Harmonogram przedstawiony w postaci diagramu gantt.}
\end{figure}
\section{Skład zespołu projektowego}
\begin{center}
  \begin{tabular}{ m{10em} m{10em} m{10em} m{10em} }
    Imię i nazwisko & Bogna Lew & Zofia Sosińska & Bartosz Strzelecki \\
    Numer indeksu & 184757 & 184896 & 184529 \\
    %%Kompentencje & Posiada & Posiada & Posiada \\
    Zadania & System budowania, sterowanie postacią & Interfejs użytkownika & Sztuczna inteligencja postaci\\
  \end{tabular}
\end{center}

\chapter{Wstęp i cel pracy}\label{chap:introduction}

\chapter{Przegląd rozwiązań stosowanych w grach}
\section{System dialogów w grach (Bartosz Strzelecki)}\label{chap:dialogi}
Systemy dialogów w grach wideo kształtują wciągającą historię, umożliwiając graczom dokonywanie wyborów, które wpływają na relacje między postaciami, zadania i narrację gry. 
Odkrywają wiedzę, pogłębiają zaangażowanie i oferują dynamiczną rozgrywkę poprzez różnorodne podejmowanie decyzji.
Dialogi umożliwiają graczowi wpłynięcie na świat, pozwalając mu wybrać, w którą stronę historia będzie podążać.
Gracz w ten sposób rozwiązuje dylematy moralne i może wczuć się w klimat rozgrywki.
"Najpopularniejsze zachodnie gry RPG, takie jak serie Baldur's Gate i Fallout, żyją i umierają dzięki sile dialogów i zdolności gracza do wpływania na postacie niezależne." \cite{dialogue}.

W \textit{Mass Effect 3}\footnote{\url{https://www.ea.com/games/mass-effect/mass-effect-3}} system dialogowy jest integralną częścią rozgrywki i pozwala graczom na prowadzenie rozmów z różnymi postaciami w trakcie gry.
System dialogów w \textit{Mass Effect 3} wykorzystuje interfejs oparty na kole dialogowym (rys. \ref{fig:wheel}), które
przedstawia graczom wiele opcji odpowiedzi podczas rozmów, zwykle podzielonych na kategorie według ich ogólnego tonu lub intencji.
Dostępne opcje często obejmują wybory dyplomatyczne, agresywne bądź konfrontacyjne oraz opcje neutralne lub śledcze.
Podczas niektórych rozmów lub przerywników filmowych gracze mogą przerwać trwającą rozmowę, szybko wybierając określoną opcję dialogową.
Te opcje przerywania pozwalają graczom podjąć natychmiastowe działania lub podjąć decyzje na miejscu, często wpływając na wynik sytuacji lub relacje postaci z innymi.
Ogólnie rzecz biorąc, system dialogowy w \textit{Mass Effect 3} został zaprojektowany tak, aby zapewnić graczom bogate i wciągające doświadczenie w opowiadaniu historii,
pozwalając im kształtować narrację poprzez wybory i interakcje z olbrzymią gamą postaci. System oferuje różnorodne opcje odpowiedzi, dynamiczne rozmowy i konsekwencje,
przyczyniając się do fascynującej i rozgałęzionej narracji gry.

Alternatywnym rozwiązaniem jest to zaprezentowane w grze \textit{Fallout 3}\footnote{\url{https://fallout.bethesda.net/pl}} (rys. \ref{fig:fallout}), które odróżniają przede wszystkim możliwe odpowiedzi gracza.
W tym przypadku użytkownik wybiera z listy gotową odpowiedź, zamiast jedynie tonu jak w grze \textit{Mass Effect}. Pozwala to na większą kontrolę
przez gracza oraz umożliwia uniknięcie sytuacji, w której gracz spodziewał się innej odpowiedzi, wybierając daną opcję dialogową.

\begin{figure}[h]
\centering
\includegraphics[width=0.8\textwidth]{images/me}
\caption{Przykład koła dialogowego w grze \textit{Mass Effect}\protect\footnotemark.}
\label{fig:wheel}
\end{figure}
\footnotetext{Internet \url{https://cdn.vox-cdn.com/thumbor/DP9qp4fQbE88gJMar2WlwAJ1gRg=/0x0:1920x1080/920x0/filters:focal(0x0:1920x1080):format(webp):no_upscale()/cdn.vox-cdn.com/uploads/chorus_asset/file/22515161/5_14_2021_10_51_45_AM_5044r2pc.png} dostęp: 12.09.2023}

\begin{figure}[h]
\centering
\includegraphics[width=0.8\textwidth]{images/fallout3}
\caption{Kadr z gry \textit{Fallout 3} przedstawiający przykładowy dialog\protect\footnotemark.}
\label{fig:fallout}
\end{figure}
\footnotetext{Internet, \url{https://www.gameuidatabase.com/uploads/Fallout-307252021-055357-81413.jpg}, dostęp: 12.09.2023}

\section{Model sztucznej inteligencji przeciwników w grach Warcraft III i StarCraft II. Bartosz Strzelecki}
Sztuczna inteligencja przeciwników w grach takich jak Warcraft III lub StarCraft II, przede wszystkim w trybie kampanii,
jest odpowiedzialna za kontrolowanie wrogich jednostek w celu zaoferowania graczowi wyzwania. Głównym zadaniem AI jest zasymulowanie
strategicznych decyzji i wydajne zarządzanie zasobami.
AI podejmuje decyzję na podstawie predefiniowanych zasad i algorytmów. Analizuje sytuację, w której się znajduje, biorąc pod uwagę
siłę swojej własnej armii, siłę armii gracza oraz specjalne zdolności jednostek i środowisko, w którym toczy się gra.
Ta analiza pozwala komputerowi na podejmowanie strategicznych decyzji jak na przykład, kiedy atakować, bronić się, eksplorować oraz rozszerzać swoje terytorium.
W tych grach sztuczna inteligencja może przybrać jedną z kilku wariantów wynikających z poziomu trudności. Wyższe poziomy
dają przeciwnikowi przewagę takie jak wydajniejsze zbieranie zasobów lub szybsza produkcja jednostek.

W grze Warcraft III w trybie kampanii zachowanie przeciwników jest zaprojektowane z myślą o zanurzeniu gracza w fabularnej opowieści, jednocześnie
prezentując wciągające wyzwania związane z rozgrywką. Akcje wykonywane przez sztuczną inteligencję są dostosowane do celów danej misji, co pozwala
na dopasowanie do obowiązującej narracji.
Początkowo przeciwnik konstruuje i rozbudowuje swoją bazę, w celu zgromadzenia odpowiedniej liczby zasobów, szkolenia jednostek i prowadzenia badań.
AI strategicznie rozmieszcza budynki i struktury obronne, aby ochronić swoją fortecę przed najazdami gracza. 
Misje kampanii często też zawierają oskryptowane wydarzenia lub walki, które dodają głębi rozgrywce. Podczas tych starć wroga sztuczna inteligencja
może zachowywać się w specjalny sposób, kontrolując potężne jednostki, do których gracz normalnie nie ma dostępu lub inicjując działania, które popychają
narrację do przodu. Te wyreżyserowane wydarzenia tworzą niezapomniane chwile i jeszcze bardziej wciągają gracza w fabułę kampanii.
Zachowanie wroga w kampanii jest zróżnicowane i obejmuje różnorodne cele misji i scenariusze. Gracze mogą napotkać wrogów, którzy preferują agresywne ataki,
inni skupiają się na strategiach obronnych lub specjalizują się w taktyce hit and run. Sztuczna inteligencja dostosowuje proces podejmowania decyzji do
konkretnych wymagań misji, często wykorzystując ukształtowanie terenu, synergię jednostek i scenariusze wydarzeń, aby rzucić wyzwanie umiejętnościom gracza.
Ogólnie rzecz biorąc, zachowanie wrogów w kampanii Warcraft III ma na celu zapewnienie dynamicznego i wciągającego doświadczenia. Gracze muszą 
wykorzystywać myślenie strategiczne, zarządzanie zasobami i efektywny skład jednostek, aby przezwyciężyć różnorodne strategie stosowane przez wrogą sztuczną inteligencję.

\section{System dialogów (Bartosz Strzelecki)}

System dialogów jest podstawową metodą, którą gracz będzie wykorzystywał, aby pozyskać informacje  o świecie oraz celach misji.
Gracz może inicjować konwersacje z postaciami niezależnymi, po czym zostaną mu zaproponowane opcje sposobu prowadzenia rozmowy.
W zależności od wybranych opcji dialogowych gracz może się spodziewać różnych konsekwencji.
"Postacie niezależne (NPC) w grach są jednym z najbardziej złożonych i uniwersalnych sposobów pośredniego prowadzenia graczy, który może przybierać różne formy"\cite{projektowanie_gier}


\href{https://assetstore.unity.com/packages/tools/utilities/dialogue-editor-168329}{Dialogue Editor} autorstwa Grasshop Dev jest prostym narzędziem pozwalającym na szybkie dodawanie i modyfikację dialogów.
Zawiera zestaw elementów ułatwiających wdrożenie systemu do projektu oraz udostępnia struktury danych wykorzystywanych do tworzenia interfejsu użytkownika.
Podczas rozmowy z postaciami niezależnymi gracz będzie mógł pozyskać informację o geografii świata, możliwych zagrożeniach oraz zadaniach do wykonania. 
Podobne systemy występują w grach takich jak Pillars of Eternity oraz w grach z serii Mass Effect. (por. \ref{chap:dialogi})


\section{Sztuczna inteligencja (Bartosz Strzelecki)}
W przypadku implementacji mechanizmów sztucznej inteligencji przeciwników będziemy się inspirować trybami kampanii w grach Warcraft III oraz Starcraft II (por. \ref{s:ai_wpr}). 
Na mapie będą rozsiane punkty, w których będą pojawiać się przeciwnicy. 
Tak długo, jak drużyna gracza jest poza zasięgiem, wrogowie pozostają nieaktywni. 
Aktywni przeciwnicy zachowują się zgodnie z ich archetypem (klasą postaci) oraz pozostają aktywni tak długo, jak drużyna gracza jest w zasięgu. 
Zdezaktywowani przeciwnicy wracają do swojego oryginalnego stanu. Gracz będzie napotykał tego typu obozowiska przede wszystkim w trakcie eksploracji świata. 
Innym planowanym przykładem implementacji sztucznej inteligencji przeciwników jest model, w którym jednostki wroga poruszają się z punktu początkowego w stronę bazy gracza. 
Jeśli podczas swojej podróży napotkają drużynę gracza, wtedy niezwłocznie zmieniają swój cel ataku.
Będziemy wyróżniać trzy archetypy jednostek w zależności od sposobu walki (bliski zasięg, średni zasięg, daleki zasięg). 
Postacie walczące na bliski zasięg mają na celu podejście w stronę najbliższego przeciwnika i wykonać atak. 
Jednostki średnio zasięgowe w momencie, w którym najbliższy przeciwnik jest odpowiednio daleko, wykonują atak dystansowy, 
w przeciwnym wypadku zachowują się tak jak jednostki walczące w zwarciu. 
Postacie  dalekodystansowe dokonują ataków dystansowych w kierunku najbliższego przeciwnika, natomiast uciekają, gdy ten podejdzie zbyt blisko.



\section{Widzenie przez ściany (Bartosz Strzelecki)}\label{s:wid_proj}

Do umiejętności wykorzystywanych przez gracza będzie należeć zdolność widzenia przeciwników oraz innych istotnych obiektów przez przeszkody.
Gracz po wciśnięciu przycisku przez krótki okres będzie w stanie zobaczyć sylwetki przeciwników znajdującymi się w jego polu widzenia.
Rozwiązanie jest inspirowane wcześniej wspomnianą grą \textit{Dead by Daylight} (por. \ref{chap:dbd}). Po pojawieniu się,
markery nie będą się poruszać za celem, lecz pozostać w tym samym miejscu przez czas trwania animacji.



\chapter{Wprowadzenie do dziedziny}\label{chap:field}




\section{Podrozdział}
Lorem ipsum dolor sit amet, consectetur adipiscing elit. In semper, sem id aliquam consectetur, nulla enim ornare erat, vitae lacinia odio enim sagittis leo. Nunc vestibulum lorem sem, a pharetra orci volutpat a. Nullam quis fermentum dolor. Sed quam nisl, imperdiet quis leo quis, sollicitudin dictum nisi. Vivamus orci mauris, convallis eget blandit eget, convallis sed est. Nullam a ex in ex ultricies suscipit nec a purus. Morbi tincidunt libero et magna mollis, in posuere quam maximus. Donec commodo nunc orci, ut convallis nulla congue ac. Pellentesque pulvinar semper aliquet. Mauris ac libero vel ante ullamcorper vulputate sed id diam. Vivamus lobortis orci non nunc lobortis, ac varius est scelerisque. Integer ut nibh est. Phasellus nec dui luctus, porttitor nulla ac, commodo lectus. Vestibulum vehicula aliquam sem, quis accumsan nibh vulputate sed. Obrazek kawusi to rys.\ref{fig:coffee}~\cite{dx12_2}.

\begin{figure}[htbp]
    \centering
    \includegraphics[width=0.9\textwidth]{images/kawunia.png}
    \caption{Smacznej kawuni}\label{fig:coffee}
\end{figure}

Nunc venenatis in felis lobortis vehicula. Praesent consequat aliquam mollis. Cras sed interdum lectus, eget maximus ipsum. Praesent sollicitudin nisi eros, et pellentesque nisi fermentum non. Orci varius natoque penatibus et magnis dis parturient montes, nascetur ridiculus mus. Maecenas in volutpat sapien, eget mattis orci. Nulla tellus ligula, blandit nec lorem nec, eleifend condimentum ante.

Cras ultricies leo ipsum, ut sodales augue mattis ut. Nam aliquam blandit felis, posuere blandit felis rutrum in. Aenean feugiat sodales leo ut pulvinar. Maecenas vel scelerisque ex, vitae feugiat ante. Donec porta tincidunt dapibus. Sed sed pharetra tellus, a gravida quam. Donec euismod risus vitae turpis commodo, porta eleifend diam porta. Nullam sagittis gravida dolor, in blandit velit ornare in. Proin quis accumsan mi, sed consequat purus. Integer vitae imperdiet diam. Donec nec congue urna, sit amet tempus justo. Etiam pharetra pellentesque libero in tincidunt. Integer tristique tempor enim et malesuada. Donec accumsan lacus ut ligula mollis imperdiet. Maecenas malesuada dui eu egestas fermentum. Fusce rhoncus faucibus elit sit amet fermentum.

Nunc vehicula vel mi vel rhoncus. Curabitur euismod, arcu id faucibus rutrum, arcu leo consectetur sapien, ac ultricies mi erat sed velit. Nunc vitae quam eget eros facilisis hendrerit vitae non nibh. Duis dui orci, bibendum cursus libero id, tempor consectetur purus. Vivamus sit amet viverra lorem. Vivamus id tincidunt lectus. Sed varius feugiat enim, non aliquet urna fermentum eu. Integer vestibulum velit sit amet metus consectetur dapibus. Vestibulum sagittis eu magna a porttitor. Cras dapibus ipsum at urna condimentum, quis ornare sem vulputate. Suspendisse efficitur sit amet odio nec rhoncus. Donec venenatis urna velit, et aliquam neque molestie sit amet. Nulla vulputate accumsan metus. Donec vel tortor mauris. Fusce sit amet leo ut tortor commodo accumsan at et turpis. Ut hendrerit elit in nisl posuere viverra.

In hac habitasse platea dictumst. Curabitur id nibh ut nisl vehicula interdum. Fusce diam eros, scelerisque at fermentum nec, eleifend non erat. In pharetra mauris purus, ac malesuada neque suscipit id. Vestibulum placerat orci in dui lacinia, in porta urna consectetur. Etiam feugiat vehicula nulla, eget scelerisque velit malesuada in. Donec nec rhoncus ex. Orci varius natoque penatibus et magnis dis parturient montes, nascetur ridiculus mus. Donec tincidunt ex sed urna maximus, vel dapibus arcu venenatis. Cras posuere nisl erat. Vestibulum ante ipsum primis in faucibus orci luctus et ultrices posuere cubilia curae; Sed posuere neque et pretium congue. Cras at nisl sit amet dui varius feugiat. Nulla nec leo ut lacus pharetra placerat quis quis velit. Ut id hendrerit magna, ut sollicitudin diam.

Aenean varius bibendum odio, eu dapibus ipsum vulputate iaculis. Aliquam nec eleifend dui. Nunc vel eros vitae risus tincidunt fermentum in at ex. Etiam leo leo, ultricies et ultricies nec, tristique accumsan odio. Quisque nibh quam, sollicitudin quis aliquam eget, laoreet sit amet enim. Pellentesque eget sollicitudin justo, sed lobortis nisl. Orci varius natoque penatibus et magnis dis parturient montes, nascetur ridiculus mus. Cras consectetur turpis ut ipsum elementum, vel feugiat dui dapibus. Maecenas est metus, accumsan vel dolor rutrum, rutrum porttitor lectus. Curabitur nec finibus sapien. Nam non rutrum magna, at tempor massa. Morbi nec congue sem. Curabitur rutrum dolor eu diam mattis, ut ultrices ipsum auctor.

\chapter{Technologie, algorytmy i narzędzia}\label{chap:algs}

\chapter{Projekt systemu}
\label{chap:project}
\chapter{Badania}
\label{chap:research}
\chapter{Podsumowanie}
\label{chap:summary}

% Bibliografia, ignorujemy overfull box, bo są długie URL
\hfuzz=50pt
\printbibliography[title=\bibliographyname]
\addcontentsline{toc}{chapter}{\bibliographyname}
\hfuzz=0pt

% Wykaz rysunków
\listoffigures
\addcontentsline{toc}{chapter}{\listfigurename}

% Wykaz tabel
\listoftables
\addcontentsline{toc}{chapter}{\listtablename}

\end{document}
