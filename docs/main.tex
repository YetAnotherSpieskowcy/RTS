% Wczytanie szablonu
\documentclass[nostrict]{Szablon}

% Definicja dokumentu
\usepackage[unicode=true]{hyperref}
\usepackage{listings}
\newcommand\PDFtitle{Tytuł pracy}
\newcommand\PDFauthors{Imie Nazwisko}
\hypersetup{
  pdftitle={\PDFtitle},
  pdfauthor={\PDFauthors},
}

\addbibresource{bib.bib}

% Zmiana czcionki dla symulacji maszynopisu (verbatim)
\makeatletter
\renewcommand{\verbatim@font}{\ttfamily\small}
\makeatother

% Część właściwa pracy
\begin{document}
\chapter*{Streszczenie}
\addcontentsline{toc}{chapter}{Streszczenie}
Tematem pracy inżynierskiej jest realizacja gry z gatunku strategii czasu rzeczywistego osadzonej w realiach historycznych
wybranej epoki. Niniejsza praca zawiera opis przykładowych rozwiązań zastosowanych w niektórych grach dostępnych na rynku
oraz projekt i implementację wytwarzanego programu. Praca skupia się na przedstawieniu zaimplementowanych mechanik oraz
sposobie oddania realiów, w których osadzona jest fabuła opracowanej gry.

Do przygotowania prototypu gry wykorzystano silnik Unity oraz udostępniane przez niego narzędzia. Za ich pomocą
przygotowano podstawowe mechanizmy, typowe dla gier strategii czasu rzeczywistego oraz przykładowe zadania tworzące
fabułę gry. Utworzony prototyp pozwala na dalsze rozwijanie świata gry poprzez dodawanie nowych zadań oraz elementów
rozgrywki. Zagadnienia implementacyjne zostały opisane w rozdziale Implementacja, a efekt finalny w Przebiegu rozgrywki.

Podstawowym celem projektu jest przygotowanie gry, która w jak najdokładniejszy sposób oddawałaby realia wybranej epoki.
Zdecydowaliśmy się na osadzenie fabuły we wczesnym średniowieczu. Główną inspiracją stali się Celtowie, którzy w
tamtych czasach zamieszkiwali tereny współczesnej Irlandii. W ramach projekty przede wszystkim skupiono się na
opracowaniu sposobu nawigacji w grze oraz zachowania przeciwników, stosujących broń oraz słownictwo charakterystyczne
dla realiów historycznych. W rozdziale Projekt została przedstawiona wizja autorów na poszczególne elementy gry.

W celu przygotowania się do projektu i implementacji gry, przeprowadzono przegląd wybranych mechanik w grach dostępnych
na rynku. Niniejsza praca zawiera opis ich działania na podstawie przykładów z istniejących gier oraz sposobu, w jaki
wpływają na ich rozgrywkę. Przedstawione rozwiązania stanowią inspirację dla wytwarzanej gry.

Praca została przygotowana przez trzech współautorów. Bartosz Strzelecki był odpowiedzialny za rozdziały ref{ss:rts}, ...
Jest również współatorem punktów ref{ss:comp} oraz ... Bogna Lew odpowiadała za rozdziały ref{ss:rpg}, ref{s:budowanie},
ref{s:walka}, ... oraz współtworzyła ref{ss:comp}, {chap:introduction}, ... Zofia Sosińska była odpowiedzialna za
pracę nad rozdziałami ref{ss:tbs}. Współtworzyła również rozdział {chap:introduction}.

\chapter*{Abstract}
\addcontentsline{toc}{chapter}{Abstract}  
The topic of the enginnering thesis is a prototype of real-time strategy game set in historical reality of selected era.
This work contains a description of sample solutions used in selected games available on the market
and the design and a overview of the implementation of the produced program. The work focuses on the presentation 
of the implemented mechanics as well as
the way of rendering the realities in which the plot of the developed game is set.

The Unity engine and the tools it provides were used to prepare a prototype of the game.
Employing these tools, basic mechanics typical of real-time strategy games, were prepared,
as well as sample tasks forming the games's plot. The devised prototype allows for 
further development of the game environment by adding new activities and gameplay elements.
The implementation matters are described in the Implementation chapter, and the results in the Gameplay Course.

The fundamental goal of the project is to develop a game that would reflect the reality of the selected era as accurately as possible.
The main inspiration came from the Celts, who at that time inhabited the territory of modern day Ireland. 
The project primarily focused on
developing the means of navigation in the game and the behavior of opponents, using weaponry and vocabulary characteristic to the
historical realities. The Project chapter presents the authors' vision for the various elements of the game.

In preparation for the design and implementation of the game, a review of selected mechanics in games available
on the market were conducted. This paper describes how they work using examples from existing games and how
they affect their gameplay. The solutions presented provide inspiration for the developed game.

\chapter*{Spis treści}
\addcontentsline{toc}{chapter}{Spis treści}

\tableofcontents

\chapter*{Wykaz ważniejszych oznaczeń i skrótów}
\addcontentsline{toc}{chapter}{Wykaz ważniejszych oznaczeń i skrótów}
\begin{description}[style=multiline,leftmargin=3cm]
\item[RTS] Gra strategii czasu rzeczywistego (ang. \textit{real-time strategy}). Jest to gatunek gier, w którym gracz nie jest
ograniczany przez turowość i kolejność ruchów. Wymusza szybsze podejmowanie decyzji, a ich skutki są natychmiastowe.
\item[cRPG] Kompyterowa gra fabularna (ang. \textit{computer role-playing game}). Gatunek gier, w którym gracz wciela
się w postać lub drużynę, przemieszczając się w świecie stworzonym przez autorów gry.
\item[TBS] Strategiczna gra turowa (ang. \textit{turn-based strategy}). Jest to podgatunek gier strategicznych, w którym
gracze wykonują swoje akcje w turach.
\item[AAA (Triple-A)] Termin stosowany w przemyśle gier komputerowych. Służy do określenia wysokobudżetowych gier, od
których oczekuje się wysokiej jakości.
\item[Wielkie odkrycia geograficzne] Termin odnoszący się do odkryć geograficznych, które miały miejsce na przełomie XV
i XVI wieku.
\item[AI] Sztuczna inteligencja (ang. \textit{artificial intelligence}) jest wykorzystywana do imitowania inteligentnego
zachowania postaci niezależnych.
\item[NPC] Termin określa postacie niezależne (ang. \textit{non-playable character}), czyli postacie, które nie są kontrolowane bezpośrednio
przez gracza.
\item[Input Manager] Komponent silnika Unity, który umożliwia definiowanie wirtualnych osi oraz przypisywanie do nich
odpowiednich klawiszy. Poprzez odczyt wartości zwracanej przez oś możliwe jest wyznaczenie odpowiedniej akcji.
\item[UI] Interfejs użytkownika (ang. \textit{user interface}), czyli oprogramowanie umożliwiające użytkownikowi
interakcję z systemem.
\end{description}

\section{Organizacja (Bartosz Strzelecki)}\label{s:org}
W tym podrozdziale przedstawiono harmonogram prac, wraz z ich przewidywanym terminem realizacji.
Ponadto zaprezentowano skład zespołu projektowego, kompetencje ich członków oraz podział zadań.
\subsection{Główne etapy projektu}
\begin{center}
  \begin{tabular}{| m{30em} | m{12em}|} 
  \hline
  Etap & Termin realizacji \\
  \hline\hline
  Wybór i analiza konkretnego kontekstu historycznego. & Kwiecień 2023 \\
  \hline
  Syntetyczny opis modelu postrzegania przestrzeni na podstawie dzieł pisanych, architektury i sztuki. & Kwiecień — Maj 2023 \\
  \hline
  Przegląd rozwiązań stosowanych w grach strategicznych z wybranego okresu oraz dodatkowo mechanizmów z innych gier, które mogłyby być zaadoptowane na potrzeby projektu. & 2, 3 kwartał 2023 \\
  \hline
  Opracowanie fabuły, selekcja postaci i wydarzeń, a także określenie zakresu autonomii świata gry oraz możliwości modyfikowania go przez gracza. & Czerwiec 2023 \\
  \hline
  Opracowanie szczegółowej koncepcji i projektu gry, w tym projekt mechanizmów zawartych w prototypie. & Lipiec 2023 \\
  \hline
  Implementacja poszczególnych funkcjonalności gry. & 4 kwartał 2023 \\ 
  \hline
  Testowanie, weryfikacja założeń i walidacja. & Listopad — Grudzień 2023 \\
  \hline
  Stworzenie dokumentacji przeprowadzonych prac. & 3, 4 kwartał 2023 \\
  \hline
\end{tabular}
\end{center}
Przewidywany termin zakończenia prac nad projektem to grudzień 2023 roku.
\begin{figure}[htbp]
    \centering
    \includegraphics[width=1\textwidth]{uml/Harmonogram}
    \caption{Harmonogram przedstawiony w postaci diagramu gantt.}
\end{figure}
\section{Skład zespołu projektowego}
\begin{center}
  \begin{tabular}{ m{10em} m{10em} m{10em} m{10em} }
    Imię i nazwisko & Bogna Lew & Zofia Sosińska & Bartosz Strzelecki \\
    Numer indeksu & 184757 & 184896 & 184529 \\
    %%Kompentencje & Posiada & Posiada & Posiada \\
    Zadania & System budowania, sterowanie postacią & Interfejs użytkownika & Sztuczna inteligencja postaci\\
  \end{tabular}
\end{center}

\chapter{Wstęp i cel pracy}\label{chap:introduction}


\chapter{Wprowadzenie do dziedziny}\label{chap:intr}

“Starożytni świat widzieli inaczej, mniej płasko”\cite{gbobrektvgry}. Dobrze obrazującym ówczesne postrzeganie przestrzeni przykładem jest mapa Imperium Rzymskiego,
pokazana na \ref{fig:mapaIR}. Czytanie jej dosłownie mija się z celem. Nie są na niej zachowane ani proporcje, ani strony świata. Mimo tego, że
basen Morza Śródziemnego został ówcześnie dosyć dokładnie oddany, “nie wydaje się, aby Rzymianom współczesna kartograficzna wierność była potrzebna”\cite{gbobrektvgry}.
“Dowódcy opierali się na swojej wiedzy, wiedzy wynajętych przewodników oraz informacjach zwiadowców i tubylców”\cite{gbobrektvgry}.
\begin{figure}[htbp]
    \centering
    \includegraphics[width=0.5\textwidth]{images/mapaIR.png}
    \caption{Mapa basenu Morza Śródziemnego z czasów Imperium Rzymskiego}\label{fig:mapaIR}
\end{figure}

\section{System dialogów w grach (Bartosz Strzelecki)}\label{chap:dialogi}
Systemy dialogów w grach wideo kształtują wciągającą historię, umożliwiając graczom dokonywanie wyborów, które wpływają na relacje między postaciami, zadania i narrację gry. 
Odkrywają wiedzę, pogłębiają zaangażowanie i oferują dynamiczną rozgrywkę poprzez różnorodne podejmowanie decyzji.
Dialogi umożliwiają graczowi wpłynięcie na świat, pozwalając mu wybrać, w którą stronę historia będzie podążać.
Gracz w ten sposób rozwiązuje dylematy moralne i może wczuć się w klimat rozgrywki.
"Najpopularniejsze zachodnie gry RPG, takie jak serie Baldur's Gate i Fallout, żyją i umierają dzięki sile dialogów i zdolności gracza do wpływania na postacie niezależne." \cite{dialogue}.

W \textit{Mass Effect 3}\footnote{\url{https://www.ea.com/games/mass-effect/mass-effect-3}} system dialogowy jest integralną częścią rozgrywki i pozwala graczom na prowadzenie rozmów z różnymi postaciami w trakcie gry.
System dialogów w \textit{Mass Effect 3} wykorzystuje interfejs oparty na kole dialogowym (rys. \ref{fig:wheel}), które
przedstawia graczom wiele opcji odpowiedzi podczas rozmów, zwykle podzielonych na kategorie według ich ogólnego tonu lub intencji.
Dostępne opcje często obejmują wybory dyplomatyczne, agresywne bądź konfrontacyjne oraz opcje neutralne lub śledcze.
Podczas niektórych rozmów lub przerywników filmowych gracze mogą przerwać trwającą rozmowę, szybko wybierając określoną opcję dialogową.
Te opcje przerywania pozwalają graczom podjąć natychmiastowe działania lub podjąć decyzje na miejscu, często wpływając na wynik sytuacji lub relacje postaci z innymi.
Ogólnie rzecz biorąc, system dialogowy w \textit{Mass Effect 3} został zaprojektowany tak, aby zapewnić graczom bogate i wciągające doświadczenie w opowiadaniu historii,
pozwalając im kształtować narrację poprzez wybory i interakcje z olbrzymią gamą postaci. System oferuje różnorodne opcje odpowiedzi, dynamiczne rozmowy i konsekwencje,
przyczyniając się do fascynującej i rozgałęzionej narracji gry.

Alternatywnym rozwiązaniem jest to zaprezentowane w grze \textit{Fallout 3}\footnote{\url{https://fallout.bethesda.net/pl}} (rys. \ref{fig:fallout}), które odróżniają przede wszystkim możliwe odpowiedzi gracza.
W tym przypadku użytkownik wybiera z listy gotową odpowiedź, zamiast jedynie tonu jak w grze \textit{Mass Effect}. Pozwala to na większą kontrolę
przez gracza oraz umożliwia uniknięcie sytuacji, w której gracz spodziewał się innej odpowiedzi, wybierając daną opcję dialogową.

\begin{figure}[h]
\centering
\includegraphics[width=0.8\textwidth]{images/me}
\caption{Przykład koła dialogowego w grze \textit{Mass Effect}\protect\footnotemark.}
\label{fig:wheel}
\end{figure}
\footnotetext{Internet \url{https://cdn.vox-cdn.com/thumbor/DP9qp4fQbE88gJMar2WlwAJ1gRg=/0x0:1920x1080/920x0/filters:focal(0x0:1920x1080):format(webp):no_upscale()/cdn.vox-cdn.com/uploads/chorus_asset/file/22515161/5_14_2021_10_51_45_AM_5044r2pc.png} dostęp: 12.09.2023}

\begin{figure}[h]
\centering
\includegraphics[width=0.8\textwidth]{images/fallout3}
\caption{Kadr z gry \textit{Fallout 3} przedstawiający przykładowy dialog\protect\footnotemark.}
\label{fig:fallout}
\end{figure}
\footnotetext{Internet, \url{https://www.gameuidatabase.com/uploads/Fallout-307252021-055357-81413.jpg}, dostęp: 12.09.2023}

\section{Model sztucznej inteligencji przeciwników w grach Warcraft III i StarCraft II. Bartosz Strzelecki}
Sztuczna inteligencja przeciwników w grach takich jak Warcraft III lub StarCraft II, przede wszystkim w trybie kampanii,
jest odpowiedzialna za kontrolowanie wrogich jednostek w celu zaoferowania graczowi wyzwania. Głównym zadaniem AI jest zasymulowanie
strategicznych decyzji i wydajne zarządzanie zasobami.
AI podejmuje decyzję na podstawie predefiniowanych zasad i algorytmów. Analizuje sytuację, w której się znajduje, biorąc pod uwagę
siłę swojej własnej armii, siłę armii gracza oraz specjalne zdolności jednostek i środowisko, w którym toczy się gra.
Ta analiza pozwala komputerowi na podejmowanie strategicznych decyzji jak na przykład, kiedy atakować, bronić się, eksplorować oraz rozszerzać swoje terytorium.
W tych grach sztuczna inteligencja może przybrać jedną z kilku wariantów wynikających z poziomu trudności. Wyższe poziomy
dają przeciwnikowi przewagę takie jak wydajniejsze zbieranie zasobów lub szybsza produkcja jednostek.

W grze Warcraft III w trybie kampanii zachowanie przeciwników jest zaprojektowane z myślą o zanurzeniu gracza w fabularnej opowieści, jednocześnie
prezentując wciągające wyzwania związane z rozgrywką. Akcje wykonywane przez sztuczną inteligencję są dostosowane do celów danej misji, co pozwala
na dopasowanie do obowiązującej narracji.
Początkowo przeciwnik konstruuje i rozbudowuje swoją bazę, w celu zgromadzenia odpowiedniej liczby zasobów, szkolenia jednostek i prowadzenia badań.
AI strategicznie rozmieszcza budynki i struktury obronne, aby ochronić swoją fortecę przed najazdami gracza. 
Misje kampanii często też zawierają oskryptowane wydarzenia lub walki, które dodają głębi rozgrywce. Podczas tych starć wroga sztuczna inteligencja
może zachowywać się w specjalny sposób, kontrolując potężne jednostki, do których gracz normalnie nie ma dostępu lub inicjując działania, które popychają
narrację do przodu. Te wyreżyserowane wydarzenia tworzą niezapomniane chwile i jeszcze bardziej wciągają gracza w fabułę kampanii.
Zachowanie wroga w kampanii jest zróżnicowane i obejmuje różnorodne cele misji i scenariusze. Gracze mogą napotkać wrogów, którzy preferują agresywne ataki,
inni skupiają się na strategiach obronnych lub specjalizują się w taktyce hit and run. Sztuczna inteligencja dostosowuje proces podejmowania decyzji do
konkretnych wymagań misji, często wykorzystując ukształtowanie terenu, synergię jednostek i scenariusze wydarzeń, aby rzucić wyzwanie umiejętnościom gracza.
Ogólnie rzecz biorąc, zachowanie wrogów w kampanii Warcraft III ma na celu zapewnienie dynamicznego i wciągającego doświadczenia. Gracze muszą 
wykorzystywać myślenie strategiczne, zarządzanie zasobami i efektywny skład jednostek, aby przezwyciężyć różnorodne strategie stosowane przez wrogą sztuczną inteligencję.

\section{Mechanizm budowania oraz zarządzanie zasobami w grach RTS (Bogna Lew)}\label{s:budowanie}
Jednym z typowych elementów gier strategii czasu rzeczywistego  jest tworzenie baz i budowanie fortyfikacji. Mechanizm
ten stanowi urozmaicenie rozgrywki i wprowadza dodatkowe aspekty możliwe do uwzględnienia w planowaniu strategii. Dla
wielu gier RTS jest wręcz nieodłącznym elementem, który umożliwia graczowi tworzenie i rozwój nowych jednostek,
produkcję zasobów, umacnianie swojej pozycji oraz zwiększanie swojej potęgi.

Mechanizm ten wiąże się z szeregiem ograniczeń, które mają kluczowy wpływ na rozgrywkę. Należą do nich między innymi
ograniczenia związane z ukształtowaniem terenu oraz obecnością innych elementów scenerii. Każde z tych ograniczeń ma
swoje źródło w prawdziwym świecie i mechanizm budowania musi je uwzględniać.

Z tą mechaniką związany jest system zasobów, który jest popularnym aspektem gier z tego gatunku. Wiele gier strategii
czasu rzeczywistego umożliwia graczowi budowanie własnej ekonomii. Uzyskane przez niego zasoby często mogą zostać
wykorzystane przez mechanizm budowania jako koszta budowy obiektów.

Przykładem gry strategii czasu rzeczywistego implementującej tę mechanikę jest \textit{Warhammer 40,000: Dawn of War}\footnote{\url{https://www.dawnofwar.com/}}. Jest to
gra, której realia są osadzone w uniwersum gry bitewnej \textit{Warhammer 40,000}. Udostępnia ona tryb jednoosobowy oraz
wieloosobowy dla maksymalnie sześciu graczy. W pierwszym wariancie gracz wciela się w postać dowódcy
armii Space Marines z Blood Ravens i ma za zadanie zapobiec inwazji Orków. Gra \textit{Warhammer 40,000: Dawn of War} bardzo szybko
zyskała na popularności i oferowała wszystko, co było potrzebne dla tego gatunku. Z tego powodu warto się jej przyjrzeć,
pomimo faktu, że jej realia znacząco odbiegających od tych, w których zostanie osadzona tworzona przez nas gra.

\textit{Warhammer 40,000: Dawn of War} wyróżnia model pozyskiwania surowców. W grze dostępne są dwa rodzaje: Energia, która jest
generowana przez dedykowane do tego budowle oraz Rekwizycja, której szybkość wytwarzania jest uzależniona od kontrolowanych
przez gracza punktów strategicznych. Taka mechanika znacznie lepiej wpasowuje się w realia gry oraz wymusza na użytkowniku
przyjęcie agresywniejszej strategii.

Dodatkowo \textit{Warhammer 40,000: Dawn of War} posiada typowy dla gier RTS mechanizm tworzenia budowli. Gracz ma
do dyspozycji jednostki, którym może zlecić budowę wybranego przez siebie obiektu po poniesieniu kosztów jego utworzenia.
Zanim będzie możliwe rozpoczęcie budowania użytkownik musi wybrać miejsce, w którym budynek powstanie, co robi, przesuwając
jego podgląd po mapie. W tym czasie gra dokonuje walidacji miejsca i informuje gracza czy wybrany obszar jest poprawny,
odpowiednio podświetlając widok budynku. Wybudowanie obiektu nie jest natychmiastowe, co sprawia, że gra lepiej oddaje
realia, w których jest osadzona.

\begin{figure}[h!]
    \centering
    \includegraphics[width=0.9\textwidth]{images/warhammer1.jpg}
    \caption[Budowanie budynku przez dedykowaną do tego jednostkę w grze \textit{Warhammer 40,000: Dawn of War}.]{Budowanie budynku przez dedykowaną do tego jednostkę w grze \textit{Warhammer 40,000: Dawn of War}.\protect\footnotemark}
\end{figure}
\FloatBarrier
\footnotetext{Internet, \url{https://www.youtube.com/watch?v=wNtnGFoVReU}, dostęp: 19.11.2023}
\section{Sterowanie postacią oraz walka (Bogna Lew)}\label{s:walka}
Udostępnianie graczowi jego własnej postaci jest cechą charakterystyczną raczej komputerowych gier fabularnych niż gier
strategii czasu rzeczywistego. Jednakże jest to ciekawe rozwiązanie, które w znaczny sposób urozmaica rozgrywkę oraz
wpływa na stopień zaangażowania użytkownika. Udostępnienie graczowi jego własnej postaci "czyni gracza kreatywnym
elementem działającym wewnątrz dyskursu, który posiada przestrzenny charakter" \cite{olbrzymwcieniu}.

Przykładem gry implementującej łatwy w obsłudze sposób sterowania postacią jest gra \textit{The Elder Scrolls V: Skyrim}. Umożliwia
ona graczowi możliwość poruszania się postacią za pomocą klawiszy \texttt{W}, \texttt{A}, \texttt{S} oraz \texttt{D}. Dodatkowo użytkownik może
zmieniać prędkość swojego bohatera, przytrzymując klawisze \texttt{Alt} bądź \texttt{Ctrl}, które odpowiednio powodują zwolnienie lub
przyspieszenie tempa przemieszczania się. Do rozglądania się wykorzystywana jest mysz, której ruch powoduje obrót postaci
wokół własnej osi. Ataki gracz może wykonać poprzez naciśnięcie prawego bądź lewego przycisku myszy, które powodują
wykonanie akcji odpowiednio lewą lub prawą dłonią. Jest to przykład dbałości o szczegóły, które sprawiają, że gra
jest jeszcze bardziej interesująca i satysfakcjonująca.

Innym tytułem wartym uwagi jest \textit{Kingdom Come: Deliverance}\footnote{\url{https://www.kingdomcomerpg.com/pl}}. Jest to gra osadzona w realiach Europy Środkowej na początku
XV wieku. Godny uwagi jest jej mechanizm walki, ponieważ twórcy skupili się na jak najdokładniejszym oddaniu średniowiecznego
stylu walki. W tym celu skrupulatnie przestudiowali, w jaki sposób władano mieczem w tamtych czasach, a następnie w
pełni oddali to w grze. Wykorzystali do tego tysiące animacji oraz starannie oddali fizykę pojedynków. W efekcie powstał
realistyczny mechanizm walki, który umożliwia graczowi parowanie, zadawanie ciosów oraz blokowanie.

Kolejnym interesującym tytułem jest gra \textit{Mount\&Blade}. Podobnie jak w przypadku \textit{Skyrima}, gracz steruje swoją postacią za
pomocą klawiszy \texttt{W}, \texttt{A}, \texttt{S} oraz \texttt{D}. Co więcej, gracz może wykonywać ataki poprzez naciśnięcie
lewego przycisku myszy, co jest typowym rozwiązaniem w grach,
które w prosty sposób umożliwia graczowi uczestniczenie w potyczkach. Jednakże w przeciwnieństwie do \textit{Skyrima}, w
\textit{Mount\&Blade} ruch myszą nie powoduje obrotu całej postaci,
a jedynie kamery. Zmiana kierunku odbywa się wyłącznie za pomocą klawiszy sterujących. Dzięki temu gracz może zobaczyć, co
się dzieje za jego postacią bez konieczności zmiany kierunku ruchu bądź zatrzymania się. Powoduje to jednak pewne kłopoty z zachowaniem
realizmu, gdyż w prawdziwym świecie nie jest możliwe zobaczenie czegoś bez konieczności zwrócenia się w tę stronę.

Powyższe przykłady obrazują najpopularniejsze rozwiązania umożliwiające sterowanie postacią oraz wykonywanie ataków.
Typowymi narzędziami wykorzystywanymi do poruszania postacią oraz kamerą są myszka i klawisze \texttt{W}, \texttt{A}, \texttt{S} oraz \texttt{D},
natomiast do atakowania - lewy przycisk myszy. Różnice zaczynają się pojawiać w samych mechanikach, gdyż to w jaki
sposób postać gracza będzie się zachowywać zależy od wizji autorów gry.
\section{Sterowanie jednostkami (Zofia Sosińska)}\label{chap:mb}
Gry z możliwością tworzenia drużyny muszą rozwiązć problem zachowania podwładnych. Program może udostępniać
skomplikowaną sztuczną inteligencję dla wojowników, zawsze rozwiązujac konflikt w optymalny sposób, ale w tym przypadku 
użytkownik staje się obserwatorem, a nie dowodzącym, co odciąży gracza. Innym podejściem może być zaprojektowanie podwładnych jako 
marionetek bez własnej woli, które będą biernie czekać, aż do otrzymania rozkazu. Wtedy jednak gracz musi skupiać się na 
każdym ruchu za równo swojej, jak i przeciwnej drużyny tak, aby w porę móc zareagować na wszystkie zmiany. Takie rozwiązanie 
jest obciżążające dla użytkownika. Przy projektowaniu mechaniki sterowania jednostkami trzeba zachować balans pomiędzy 
dodaniem i odjęciem użytkownikowi zadań, na kórych musi się skupić. Dla każdej gry ta proporcja może być inna, zależnie
od unikalnego charakteru gry.

Przy projektowaniu gry Mount\&Blade studio TaleWorlds Entertainment zdecydowało się zaimplementować mechanikę sterowania jednostkami tak, aby 
dodać do walki element strategii.Gracz bezpośredniokieruje jedynie główną postacią. Podczas walki reszcie może wydawać rozkazy. Poprzez
cyfry 0-4 wybiera grupę, do której się odnosi np. łuczników. Po naciśnięciu konkretnego przycisku, po lewej stronie ekranu pojawia się lista dostępnych komend.
Następnie przez klawisze F1-F11 wydaje konkretny rozkaz np. odwrót. Lista znika, sztuczna inteligencja postaci zajmuje się już samym wykonaniem czynności. 
Gracz nie martwi się, czy jednostki znajdą optymalną drogę, 
będą celować w przeciwników, czy z nimi walczyć. Zachowany jest więc przyjemny balans pomiędzy podejmowaniem kluczowych decyzji,
odciążeniem poprzez wprowadzenie sztucznej inteligencji.

\begin{figure}[h!tbp]
    \centering
    \includegraphics[width=0.9\textwidth]{images/ui/commandsMountBla.png}
    \caption{Wykaz dostępnych rozkazów z gry Mount\&Blade.}\label{fig:MountnBlade}
    \label{fig:mnb}
\end{figure}

\section{Kompas w grze Skyrim}\label{chap:skrm}

The Elder Scrolls V: Skyrim (skrótowo Skyrim) jest to fabularna gra akcji o otwartym świecie, wyprodukowana przez Bethesda Game Studios i wydana przez Bethesda Softworks. Skyrim jest piątym tytułem z serii The Elder Scrolls oraz kontynuacją gry The Elder Scrolls IV: Oblivion. Jest to jednak nowa historią osadzoną w uniwersum The Elder Scrolls, a nie kontynuacją poprzednika. Fabuła opiera się na powrocie smoków do krainy Tamriel. Bohater okazuje się posiadać moc Głosu, dzięki czemu jest w stanie posługiwać się zaklęciami tych starożytnych stworzeń.
	Z punktu tworzonej przez nas gry, szczególnie interesujące jest  bardzo proste i sprytne rozwiązanie, jakim jest pasek przedstawiający pole widzenia gracza. Służy on między innymi jako kompas, ponieważ jedną z jego mechanik jest pokazanie użytkownikowi stron świata, znajdujących się w kierunku, w którym on patrzy. Pasek ułatwia także poruszanie się po świecie sygnalizując położenie wrogów, kompanów i ważnych dla rozgrywki lokalizacji.

    \begin{figure}[htbp]
        \centering
        \includegraphics[width=0.9\textwidth]{images/ui/compassSkyrim.png}
        \caption{Pasek z gry The Elder Scrolls V: Skyrim ukazujący pole widzenia gracza. W tym momencie patrzy on delikatnie w prawo od południa (pokazuje to litera “S”). W jego zasięgu widzenia jest dwoje przeciwników (czerwone kropki) oraz jeden sojusznik (szara ikona po prawej).}\label{fig:Fallout}
    \end{figure}



\section{Przedstawienie dostępnych pułapek do zbudowania w grze Orcs must die!. Zofia Sosińska}\label{chap:omd}

Orcs must die! to strategiczna gra akcji stworzona i wydana przez studio Robot Entertainment. Akcja toczy się w krainie fantasy, w której największym zagrożeniem dla ludzkości są orkowie. W obronie świata przed tymi stworzeniami staje Zakon dowodzony przez Wojennych Magów. Wznieśli oni system fortec odgradzających ojczyznę orków od pozostałych ziem. Zadaniem gracza jest wcielenie się w jednego z  Wojennych Magów i mordowanie nadciągających grup orków za pomocą różnorodnych broni i mechanizmów, które może postawić.
W przejrzysty sposób zostało rozwiązane samo wyświetlenie dostępnych do zbudowania pułapek. Graczowi pokazują się wizerunki mechanizmów, które może postawić. Naciskając odpowiedni numer na klawiaturze, gracz wybiera, co chce zbudować. Po zatwierdzeniu lewym przyciskiem myszki, budynek pojawia się w zaznaczonym miejscu.


\input{chapters/technologie_narzedzia/technologie-wstep}
\section{Przegląd silników gier (Bogna Lew)}\label{s:silniki}
Silnik gier komputerowych jest oprogramowaniem służącym głównie do wytwarzania gier komputerowych. "Silnik zawiera
narzędzia i komponenty, które obsługują różne podstawowe elementy gry, takie jak rysowanie grafiki, kontrolowanie
dźwięku i przesuwanie obiektów, i pozwalają programiście skupić się na generowaniu danych dołączanych do silnika, takich
jak modele/obrazy, tekst, efekty dźwiękowe i tak dalej"\cite{design_essent}. Wybranie odpowiedniego silnika przed
rozpoczęciem pracy ma kluczowy wpływ na proces wytwórczy i efekt końcowy.

Obecnie dostępnych jest wiele silników, a każdy z nich ma różne możliwości. W celu uproszczenia wyboru zdecydowaliśmy
się zawęzić listę do trzech najpopularniejszych obecnie darmowych silników, czyli Godot, Unity oraz Unreal Engine.

Pierwszy z nich jest w pełni darmowym silnikiem open source. Posiada prosty i intuicyjny interfejs, a w Internecie
tworzone jest przez społeczność wiele samouczków. Nie posiada on jednak oficjalnej dokumentacji oraz jest zdecydowanie
mniej popularny od pozostałych dwóch.

Kolejny silnik, Unity jest określany jako przyjazny dla początkujących. Posiada bogatą dokumentację oraz jest dostępnych
dużo samouczków stworzonych przez jego społeczność. Unity świetnie się nadaje do tworzenia gier 3D. Silnik ten jest
dostępny w wersji bezpłatnej oraz oferującej więcej możliwości wersji płatnej. Co więcej, ma możliwość rozszerzenia
o dodatkowe narzędzia dostępne w Asset Store.

Ostatni z silników jest najbardziej kojarzony z grami AAA. Cechuje go zaawansowana grafika, która umożliwia wytwarzanie
fotorealistycznych gier. Korzystanie z niego jest darmowe, a opłata w wysokości 5\% jest naliczana jedynie, gdy gra
zarobi ponad milion USD.

Tabela \ref{fig:teng} przedstawia porównanie wymienionych silników w istotnych, z punktu widzenia projektu, aspektach.

\begin{table}[h]
\caption{Porównanie silników.}
\begin{center}
\begin{tabular}{ |c||c|c|c| }
 \hline
 Silnik & Unity & Unreal Engine & Godot \\
 \hline \hline
 Popularność & duża & duża & mała \\
 \hline
 3D & Tak & Tak & Tak \\
 \hline
 Język & C\# & C++ & C\#, C++, GDScript \\
 \hline
 Baza wiedzy & dokumentacja, samouczki & dokumentacja, samouczki & samouczki, fora \\
 \hline
 Open source & Nie & Nie & Tak \\
 \hline
\end{tabular}
\end{center}
\label{fig:teng} 
\end{table}

Finalnie zdecydowaliśmy się na implementację gry w Unity, ponieważ jest to silnik, który najlepiej odpowiada wymaganiom projektu.


\input{chapters/projekt/projekt-wstep}
\section{Analiza i specyfikacja wymagań (Bogna Lew, Zofia Sosińska)}\label{s:wymagania}
W niniejszej sekcji przedstawiono specyfikę wymagań funkcjonalnych, pozafunkcjonalnych oraz tych, wynikających z
głównych założeń projektu. Dodatkowo zawiera ona diagramy przypadków użycia, maszyny stanów oraz klas prototypowej gry.

\subsection{Specyfika wymagań wynikających z założeń projektu}
Z punktu widzenia projektu kluczowe jest jak najdokładniejsze oddanie realiów historycznych przy jednoczesnym
uwzględnieniu jakości rozgrywki gracza oraz cech charakterystycznych dla gier typu RTS. Z założeń wynika, że fabuła
gry powinna zostać osadzona w czasach sprzed wielkich odkryć geograficznych. Na tej podstawie zostały zdefiniowane
dodatkowe wymagania, które powinien spełniać prototyp.

Gra powinna zawierać:
\begin{itemize}
  \item sposób nawigacji jak najdokładniej odpowiadający temu stosowanemu w wybranej epoce,
  \item postacie:
  \begin{itemize}
    \item stylistycznie pasujące do realiów historycznych,
    \item wykorzystujące słownictwo adekwatne do czasów, w których osadzona jest gra,
    \item jak najlepiej oddawające światopogląd w danych czasach,
    \item stosujące oręż typowy dla wybranej epoki.
  \end{itemize}
  \item budowle stylistycznie odpowiadające wybranej epoce,
  \item sposób komunikacji z postaciami imitujący ten stosowany w danych czasach.
\end{itemize}

Dodatkowo po konsultacjach z pomysłodawcą projektu wyniknęło, że prototyp gry nie ma być typową grą z gatunku strategii
czasu rzeczywistego, a jego hybrydą z gatunkiem komputerowych gier fabularnych.

\subsection{Wymagania funkcjonalne}\label{ss:fun}
Niniejsza sekcja skupia się na określeniu wymagań funkcjonalnych, które powinien spełniać prototyp gry.

Gra powinna oferować możliwość:
\begin{itemize}\label{list:fun}
  \item uruchomienia nowej gry,
  \item sterowania postacią gracza,
  \item nawigacji w świecie gry,
  \item wchodzenia w interakcję z postaciami niezależnymi,
  \item przyjmowania zleceń od postaci niezależnych,
  \item najmowania postaci wojowników,
  \item wydawania komend wynajętym postaciom,
  \item zlecania budowy,
  \item zdobywania zasobów,
  \item odczytu wybranego stanu gry z komputera użytkownika,
  \item zapisu aktualnego stanu gry lokalnie na komputerze użytkownika.
\end{itemize}

\subsection{Wymagania pozafunkcjonalne}\label{ss:nonfun}
W tej sekcji zostały przedstawione wymagania pozafunkcjonalne projektu.

Gra powinna umożliwiać:
\begin{itemize}\label{list:nonfun}
  \item rozgrywkę w trybie offline,
  \item działanie na urządzeniach z systemem Windows lub Linux,
  \item dostosowywanie rozmiaru do wielkości ekranu komputera użytkownika,
  \item obsługę klawiatury oraz myszy,
  \item działanie w czasie rzeczywistym.
\end{itemize}

\subsection{Diagram przypadków użycia}\label{ss:usecase}
Niniejsza sekcja przedstawia diagram przypadków użycia dla głównych funkcjonalności, które będzie zawierać prototypowa gra.
Opisuje on przewidywane usługi oferowane przez poszczególne mechaniki programu.

Jedną z głównych akcji, które gra udostępni będzie wydanie rozkazów przyjaznym jednostkom. Polegać będzie ona na poinformowaniu
wojowników przez gracza jaką czynność powinni w danym momencie wykonać. Kolejną możliwością będzie zlecenie budowy, czyli
zlecenie budowniczemu wybudowania wybranego obiektu w określonym przez gracza miejscu. Ponadto użytkownik
będzie mógł przeprowadzać rozmowy z postaciami niezależnymi. Oznacza to, że będzie mógł zainicjować z nimi dialog i
następnie kształtować jego przebieg poprzez wybieranie swojej odpowiedzi z opcji proponowanych przez grę.

\begin{figure}[!htbp]
    \centering
    \includegraphics[width=1.0\textwidth]{images/diagrams/usecase.jpg}
    \caption{Diagram przypadków głównych mechanik gry.}\label{fig:usecases_d}
\end{figure}
\FloatBarrier

\subsection{Diagram stanów}\label{ss:state}
W tym podpunkcie został przedstawiony diagram stanów prototypowej gry, który ukazuje jej przewidywany sposób działania.
Prezentuje on podstawowe stany, w których może się znaleźć system gry.

Do podstawowych stanów należą "Menu główne" oraz "Rozgrywka". Pierwszy z nich oznacza, że program został uruchomiony, a
gracz wyświetla panel główny. Z tego stanu możliwe jest przejście do stanów "Odczytanie stanu gry" bądź "Utworzenie
nowej gry", które to powodują rozpoczęcie gry z zapisu lub od początku.

Stan "Rozgrywka" jest stanem złożonym i określa, że gra została rozpoczęta. W jego skład wchodzą przede wszystkim takie
stany, jak "Interakcja z postacią niezależną", w którym program się znajdzie, gdy gracz rozpocznie dialog z postacią w grze,
czy "Zarządzanie jednostkami", który to oznacza, że użytkownik wydaje rozkazy swoim wojownikom.

Diagram stanów został przedstawiony na rysunku \ref{fig:states_d}.

\begin{figure}[!htbp]
    \centering
    \includegraphics[width=1.0\textwidth]{images/diagrams/state.jpg}
    \caption{Diagram stanów gry.}\label{fig:states_d}
\end{figure}
\FloatBarrier

\subsection{Diagram klas}\label{ss:class}
W tej sekcji został pokazany uproszczony diagram klas (rys. \ref{fig:classes_d}), przedstawiający główne elementy gry.
Obrazuje podstawową strukturę tworzonego systemu oraz zależności pomiędzy poszczególnymi komponentami.

Do najważniejszych klas należą "Interfejs użytkownika", "Mechanizm interakcji z postaciami", "Mechanizm zarządzania
jednostkami" oraz "Mechanizm budowania". Obrazują one podstawowe komponenty gry, których głównym zadaniem jest zarządzanie
poszczególnymi mechanikami. "Interfejs użytkownika" jest odpowiedzialny za interakcję z graczem oraz pomaganie mu w
trakcie rozgrywki. Pozostałe trzy kolejno pozwalają graczowi na prowadzenie dialogów z postaciami
niezależnymi, wydawanie komend jego zaprzyjaźnionym jednostkom oraz budowanie obiektów.
\begin{figure}[!htbp]
    \centering
    \includegraphics[width=1.0\textwidth]{images/diagrams/class.jpg}
    \caption{Diagram klas gry.}\label{fig:classes_d}
\end{figure}

\section{Modelowanie terenu. Bogna Lew}
Do utworzenia terenu do gry wykorzystaliśmy zasób 3D World Building udostępniany przez Unity. Umożliwia on automatyczne generowanie terenu na podstawie heightmapy - monochromatycznego obrazu reprezentującego model wysokościowy. Kolor czarny reprezentuje najniższe punkty, natomiast kolor biały -  najwyższe.

\begin{figure}[h!]
    \centering
    \includegraphics[width=0.9\textwidth]{images/modelowanie_terenu/przykladowe_heightmapy.jpg}
    \caption{Przykładowe heightmapy}\label{fig:przykladowe_heightmapy}
\end{figure}

Wygenerowanie terenu umożliwia narzędzie Terrain Toolbox, które można uruchomić wybierając z menu Window -> Terrain -> Terrain Toolbox. Pozwala ono na ustawienie podstawowych parametrów takich jak długość, szerokość oraz wysokość terenu. Należy również zaznaczyć checkbox Import Heightmap oraz załączyć obraz z modelem wysokościowym. Na koniec trzeba nacisnąć przycisk Create, co spowoduje dodanie do sceny wygenerowanego terenu.

\begin{figure}[h!]
    \centering
    \includegraphics[width=0.6\textwidth]{images/modelowanie_terenu/generowanie.jpg}
    \caption{Widok na panel narzędzia Terrain Toolbox z zaznaczonymi wymienionymi sekcjami.}\label{fig:generowanie_terenu}
\end{figure}

Tak utworzony teren, chociaż już jest grywalny, posiada ostre i postrzępione krawędzie, które nie wyglądają zbyt estetycznie. Wygładzenie ich poprawi wygląd terenu i sprawi, że będzie on bardziej realistyczny. Do tego służy narzędzie Smooth Height dostępne w inspektorze terenu. Powoduje ono uśrednienie pobliskich płaszczyzn, co pozwala na usunięcie nagłych zmian terenu i w rezultacie wygładzenie go.

\begin{figure}[h!]
    \centering
    \includegraphics[width=0.8\textwidth]{images/modelowanie_terenu/rzezbienie.jpg}
    \caption{Widok panelu inspektora oraz terenu przed (górny) i po (dolny) zastosowaniu narzędzia Smooth Terrain.}\label{fig:rzezbienie_terenu}
\end{figure}

Kolejnym krokiem jest nałożenie tekstur. Służy do tego narzędzie Paint Texture. Umożliwia ono dodanie warstw, którymi będzie można pokolorować teren. Warstwa znajdująca się najwyżej jest uznawana za domyślną i jej tekstura zostanie nałożona na cały teren. Pozostałe warstwy natomiast stanowią swego rodzaju paletę kolorów, którymi można pomalować teren za pomocą pędzla, który można wybrać w zakładce Brushes. W zakładce Stroke można ustawić podstawowe parametry pędzla takie jak Brush Size oraz Brush Strength. Pierwszy parametr odnosi się do rozmiaru pędzla, a co za tym idzie obszaru na który dana tekstura zostanie nałożona, natomiast drugi pozwala na określenie w jakim stopniu nakładany materiał zakryje już nałożony.

\begin{figure}[h!]
    \centering
    \includegraphics[width=0.8\textwidth]{images/modelowanie_terenu/tekstury.jpg}
    \caption{Widok panelu inspektora z wybranym narzędziem Paint Texture oraz efektu przed nałożeniem 2 tekstury (górny zrzut), po nałożeniu tekstury gdy Brush Strength wynosi 0.26 (środkowe zdjęcie) oraz gdy wynosi 0.67 (dolne).}\label{fig:malowanie_tekstur}
\end{figure}

Kolejną opcją udostępnianą przez Unity jest możliwość automatycznego ustawienia obiektów na mapie w losowy sposób, dzięki narzędziu Paint Trees. Do udostępnianych przez nie parametrów należą między innymi Brush Size, działający analogicznie jak poprzednio, oraz Tree Density, które definiuje średnią liczbę drzew umieszczanych na zdefiniowany obszar. Obok przedstawiono przykładowy rezultat. Wykorzystano do tego paczkę LowPoly Trees and Rocks dostępną w Unity Assets Store.

\begin{figure}[h!]
    \centering
    \includegraphics[width=0.5\textwidth]{images/modelowanie_terenu/drzewa.jpg}
    \caption{Widok na teren z drzewami.}\label{fig:pomalowane_drzewa}
\end{figure}

\begin{figure}[h!]
    \centering
    \includegraphics[width=0.6\textwidth]{images/modelowanie_terenu/malowanie_drzew.jpg}
    \caption{Widok na inspektor z włączonym narzędziem Paint Trees.}\label{fig:malowanie_drzew}
\end{figure}



Do stworzenia interfejsu użytkownika (UI) za wzorce użyliśmy gier “Mount&Blade”, “Mount&Blade2 Bannerlord”, “Warcraft3”, “The Elder Scrolls V: Skyrim”, “Orcs must die”aby umożliwić graczowi jak najprzyjemniejszą rozrywkę oraz wszystkie potrzebne funkcje.

Pierwszy ekran
Wzorowanie: “Mount&Blade”
Jako pierwszy ekran, który widzi gracz przewidujemy grafikę z możliwością wybrania jednej z opcji z menu głównego.

Poruszanie się

Przewidujemy:
górny pasek z najważniejszymi informacjami:surowce, fundusze, czas, kompas
ikony postaci, na które gracz może się przełączyć;
obszar dla komentarzy
pasek życia.

Inspiracja dla górnego paska z najważniejszymi informacjami:surowce, fundusze, czas oraz ikon postaci, na które gracz może się przełączyć:

źródło: Warcraft 3

Inspiracja dla kompasu:

źródło: The Elder Scrolls V: Skyrim
Inspiracja dla obszaru dla komentarzy oraz pasek życia:

źródło: Mount&Blade2 Bannerlord

Rozmowa 

źródło: Mount&Blade

Walka
Przy walce dostępne materiały zamienią się na pasek pokazujący sumaryczne życie naszej drużyny i drużyny przeciwnej oraz możliwe komendy do wydania.


Inspiracja dla możliwych komend:
źródło: Mount&Blade

Budowa budynków
W tym trybie pokażą nam się dostępne do zbudowania budynki, a po wybraniu pojawią się przed nami. Po zatwierdzeniu budynek zostanie wybudowany.


Inspiracja dla trybu budowania:

źródło: Orcs must die

\section{Nawigacja. Zofia Sosińska}\label{chap:naw}

    Kluczowym dla gry założeniem jest ułatwienie graczowi wczucia się w realia świata, w którym się znajduje. Jako jeden z głównych warunków pogłębienia immersji uwypuklono brak implementacji mapy, na której gracz widziałby świat. W ten sposób nie upraszczamy mu poruszania się i odnajdywania lokacji tak, jak i człowiek w realnym świecie w czasach średniowiecznych nie kierował się zapisanymi na kartce kartograficznymi obrazami, ale własną i zdobytą od innych wiedza o otaczającym go terenie. 
    Jedyną pomocą, jaką otrzyma gracz, będzie pasek obrazujący pole widzenia granej postaci. Pierwszą rolą narzędzia będzie pokazanie kierunku świata, który znajduje się w polu widzenia gracza. Zakładamy, że grana postać potrafi sama taką, informację odczytać chociażby z położenia Słońca. Nie jest to więc sztucznie upraszczająca rozgrywkę mechanika, a jedynie odciążenie użytkownika kompasem. 
    Kolejną informacją na omawianym polu będzie miejsce w którym znajduje się przeciwnik. Dotyczy to antagonistów widocznych w polu widzenia, jak i ukrytych za ścianą. Druga część będzie logicznie możliwa dzięki specjalnej umiejętności widzenia wrogów za przeszkodą zaimplementowanej dla postaci druida. Po przyłączeniu takiej osoby do drużyny użytkownik może poprosić przyjaciela o użyczenie mu swej mocy.

\begin{figure}[htbp]
    \centering
    \includegraphics[width=0.9\textwidth]{images/ui/compass.png}
    \caption{Wizualizacja przypadku, w którym gracz patrzy centralnie na północ. Delikatnie na jego prawo znajduje się dwóch przeciwników za ścianą, podświetlonych specjalną umiejętnością druida.
    }\label{fig:compass}
\end{figure}


\section{Zasady działania kompasu. Zofia Sosińska}\label{chap:naw}

Projekt kompasu jest na tyle prosty, aby nie przytłoczyć gracza nadmierną liczbą bodźców. Składa się z horyzontalnego, jednolitego paska, na którym wyświetlane będą najważniejsze informacje o otoczeniu, symbol ośmiokąta, wskazujący na przestrzeń znajdującą się centralnie przed bohaterem oraz z bocznych pasków, wyróżniających końce narzędzia.

\begin{figure}[htbp]
    \centering
    \includegraphics[width=0.9\textwidth]{images/ui/opis_ekementow_kompasu.png}
    \caption{Rozpiska elementów: a. główny pasek, b. symbol środka, c1., c2. paski końców kompasu.}\label{fig:compass_design}
\end{figure}

Ikony wyświetlane na kompasie będą przedstawiać najważniejsze informacje w polu widzenia gracza, czyli stronę świata, w kierunku której jest on zwrócony, oraz przeciwników.
Poniżej przedstawiono kod odpowiedzialny za wyznaczenie pozycji symbolu na omawianym narzędziu. Wejściem jest komponent RectTransform symbolu przypisanemu danemu obiektowi oraz jego położenie. Po obliczeniu wektora prowadzącego do uzyskania np. pozycji wroga, wyznaczany jest kąt, o jaki musi się obrócić. To przeliczamy na pozycję na kompasie i jeśli obiekt jest w polu widzenia, wyświetlamy symbol.

\begin{lstlisting}[caption=Fragment kodu odpowiedzialny za ustawienie symbolu na pasku kompasu]
void SetMarkerPosition(RectTransform markerTransform, Vector3 worldPosition)
{
    Vector3 dirToTarget = worldPosition - CameraTransform.position;
    float angle = Vector2.SignedAngle(
        new Vector2(dirToTarget.x, dirToTarget.z), 
        new Vector2(CameraTransform.transform.forward.x, 
        CameraTransform.transform.forward.z));
    float compassPositionX = Mathf.Clamp(
        2 * angle / Camera.main.fieldOfView, -1, 1);
    if (compassPositionX == 1 || compassPositionX == (-1))
    {
        markerTransform.anchoredPosition = new Vector2(0, 100);
    }
    else
    {
        markerTransform.anchoredPosition = new Vector2(
            compassBarTransform.rect.width / 2 * compassPositionX, 0);
    }
}
\end{lstlisting}

Problem lokalizacji stron świata pojawił się, gdy nieprawidłowe położenie na kompasie nieprawidłowo wyświetlały się symbole po użyciu najprostszego rozwiązania. Było to pobranie wartości od Vector3.forward, co jest skrótowym zapisem Vector3(0, 0, 1). Lewy dolny róg mapy jest położony w punkcie (0, 0, 0), co oznacza, że wymagane jest przesunięcie, aby prawidłowo zasymulować strony świata. Płnoc i Południe przesunęliśmy do połowy szerokości, a Wschód i Zachód - długości mapy. Każde z nich odaliliśmy o 60000 jednostek, co pozwala na wiarygodą symulację stron świata na kompasie.
\begin{figure}[htbp]
    \centering
    \includegraphics[width=0.9\textwidth]{images/ui/strony_swiata.png}
    \caption{Rozmieszczenie zasymulowanych stron świata.}\label{fig:world_sides}
\end{figure}

Wykrycie wrogich jednostek znajduje się w funkcji Start. Każdemu obiektowi jest przypisywany symbol czerwonego miecza i w zależności od położenia wroga, obrazek wyświetlany jest odpowiednio na kompasie.
\begin{lstlisting}[caption=Fragment kodu odpowiedzialny za połączenie wrogich obiektów na mapie z symbolami wyświetlonymi na kompasie]
    void SetPositionOfEnemies()
    {
        foreach (
            var e in enemiesOnMap.Zip(
                enemiesOnUI, (x, y) => new { 
                    enemyOnMap = x, enemyOnUI = y }))
        {
            SetMarkerPosition(
                e.enemyOnUI.GetComponent<RectTransform>(),
                e.enemyOnMap.transform.position);
        }
    }
\end{lstlisting}
\section{Poruszanie postacią (Bogna Lew)}
Podstawową mechaniką, którą będzie oferować implementowana gra jest sterowanie postacią przez gracza. Użytkownik będzie
mieć do dyspozycji jedną postać, którą będzie bezpośrednio zarządzać. Umożliwi mu ona przede wszystkim eksploraację
świata oraz walkę z przeciwnikami.

Poruszanie postacią będzie analogiczne jak w grze The Elder Scrolls V: Skyrim. Gracz będzie mógł przemieszczać się
do przodu, do tyłu, na boki oraz na ukos. Sterowanie będzie możliwe za pomocą klawiszy  “w”, “a”, “s” oraz “d”.
Dodatkowo użytkownik będzie mieć możliwość obracania postaci, a tym samym zmiany kierunku w którą jest zwrócona poprzez
przemieszczanie myszy. Ponadto, gra udostępni możliwość przesunięcia kamery po łuku do góry bądź do dołu.

Kolejnym aspektem jest walka. Użytkownik będzie mógł wykonać atak porzez naciśnięcie prawego przyciku myszy. Postać
gracza będzie mogła wykonywać wyłącznie ataki bronią białą, która zostanie dobyta w momencie gdy znajdzie się on w walce.
\section{Mechanizm budowania (Bogna Lew)}\label{s:build_proj}
Inspiracją do implementacji tego mechanizmu jest gra \textit{Warhammer 40,000: Dawn of War} (por. \ref{s:budowanie}). Do pożądanych
efektów, które ten tytuł zapewnia, należą walidacja terenu oraz wymuszanie poniesienia kosztów budowy. Dodatkowo
wytwarzana gra będzie implementować proces budowania analogicznie jak w \textit{Warhammer 40,000: Dawn of War}.

Najważniejszym aspektem implementowanego mechanizmu będzie walidacja terenu. Zostanie ona uzyskana poprzez wyświetlenie
podglądu budowli w trakcie umiejscawiania konstrukcji na mapie. Jeśli miejsce, w którym gracz chce postawić budynek jest
poprawne tzn. nie nachodzi na inne obiekty oraz teren jest odpowiedni, to pokazywany widok jest podświetlany na zielono,
w przeciwnym razie - na czerwono. Jest to efekt, który wytwarzana przez zespół gra będzie zawierać.

Kolejnym pożądanym efektem jest konieczność poniesienia przez użytkownika kosztów
wybudowania obiektu. W tym celu gra będzie monitorowała, czy gracz posiada wystarczającą ilość wymaganych zasobów, a w
przypadku niespełnienia tych warunków - blokowała możliwość budowy wybranego obiektu.

Dodatkowo budowa obiektu nie może być natychmiastowa, gdyż nie jest to zgodnie z rzeczywistością. Dlatego podobnie jak w
grze \textit{Warhammer 40,000: Dawn of War} każdy budynek będzie musiał zostać wybudowany przez dedykowaną do tego postać. Będzie
ona imitowała proces budowania i dopiero gdy ukończy to zadanie, dana budowla zacznie przynosić graczowi korzyści.

\section{Sterowanie jednostkami, podążanie za główną postacią (Zofia Sosińska)}\label{chap:sjpzgp}
Na samym początku gry postać gracza pojawia się sama i jest jedynym obiektem, którym gracz może sterować.
Kontroluje, gdzie postać idzie, jak walczy oraz z kim rozmawia. Z biegiem czasu gra będzie jednak naciskać na formowanie drużyny,
ponieważ pokonywanie wielu przeciwników w pojedynkę będzie się stawało zbyt trudne. Pojawia się w takim momencie potrzeba 
zaimplementowania funkcjonalności zarządzania wieloma postaciami.

Stworzenie mechaniki poruszania się i oddziaływania na otaczający świat dla jednej postaci wydaje się proste i intuicyjne, ale kierowanie wieloma osobami już nie. 
Bez wprowadzenia zmian, używając jednego sposobu dyrygowania wszystkimi jednostkami tak samo, gra będzie prędko męczyć gracza.
Dla czynności małoznaczących, takich jak przemieszczenie drużyny w konkretne miejsce, wprowadzi monotonię i czasochłonność.
Każdą postać należy wybrać i przemieścić ją w konkretne miejsce. Kilka kliknięć przy jednej postaci jest akceptowalne, ale przy kilku wprowadzi to ogromne opóźnienia.

Jeszcze gorsze skutki pokazałyby się podczas walki. Szybkie przeskakiwanie pomiędzy postaciami podniosłoby zauważalnie trudność gry.
Poruszanie się jedną postacią i zabijanie przeciwników nie ma sensu, gdy reszta drużyny jest bita i nie może się obronić, ponieważ gracz musi się przełączyć na inną postać,
aby ta wykonała ruch. 

Z tego powodu potrzebny jest algorytm odpowiadający za właściwe poruszanie się pobocznych postaci.
Tworzona gra będzie dopuszczała małe, kilkunastoosobowe drużyny z przywódcą - postacią grywalną przez użytkownika - na czele.
Podczas walki graczowi pokażą się możliwe do wydania polecenia oraz specyfikacja, jakiej grupy mają one dotyczyć.
Za pomocą określonych klawiszy klawiatury będzie on mógł kontrolować zachowanie kompanów.

Po rozwiązaniu problemu mechaniki sterowania jednostkami w walce nie można przeoczyć samego poruszania się oraz interakcji ze światem.
Najprostszym rozwiązaniem będzie implementacja mechanizmu, według którego drużyna, po wykryciu znacznego przemieszczenia się przywódcy, sama będzie za nim podążać.
Kompani nie będą też mieli opcji samodzielnej interakcji ze światem, co sprawi, że poza walką zostaną jedynie biernymi obserwatorami.
\section{System dialogów (Bartosz Strzelecki)}

System dialogów jest podstawową metodą, którą gracz będzie wykorzystywał, aby pozyskać informacje  o świecie oraz celach misji.
Gracz może inicjować konwersacje z postaciami niezależnymi, po czym zostaną mu zaproponowane opcje sposobu prowadzenia rozmowy.
W zależności od wybranych opcji dialogowych gracz może się spodziewać różnych konsekwencji.
"Postacie niezależne (NPC) w grach są jednym z najbardziej złożonych i uniwersalnych sposobów pośredniego prowadzenia graczy, który może przybierać różne formy"\cite{projektowanie_gier}


\href{https://assetstore.unity.com/packages/tools/utilities/dialogue-editor-168329}{Dialogue Editor} autorstwa Grasshop Dev jest prostym narzędziem pozwalającym na szybkie dodawanie i modyfikację dialogów.
Zawiera zestaw elementów ułatwiających wdrożenie systemu do projektu oraz udostępnia struktury danych wykorzystywanych do tworzenia interfejsu użytkownika.
Podczas rozmowy z postaciami niezależnymi gracz będzie mógł pozyskać informację o geografii świata, możliwych zagrożeniach oraz zadaniach do wykonania. 
Podobne systemy występują w grach takich jak Pillars of Eternity oraz w grach z serii Mass Effect. (por. \ref{chap:dialogi})


\section{Sztuczna inteligencja (Bartosz Strzelecki)}
W przypadku implementacji mechanizmów sztucznej inteligencji przeciwników będziemy się inspirować trybami kampanii w grach Warcraft III oraz Starcraft II (por. \ref{s:ai_wpr}). 
Na mapie będą rozsiane punkty, w których będą pojawiać się przeciwnicy. 
Tak długo, jak drużyna gracza jest poza zasięgiem, wrogowie pozostają nieaktywni. 
Aktywni przeciwnicy zachowują się zgodnie z ich archetypem (klasą postaci) oraz pozostają aktywni tak długo, jak drużyna gracza jest w zasięgu. 
Zdezaktywowani przeciwnicy wracają do swojego oryginalnego stanu. Gracz będzie napotykał tego typu obozowiska przede wszystkim w trakcie eksploracji świata. 
Innym planowanym przykładem implementacji sztucznej inteligencji przeciwników jest model, w którym jednostki wroga poruszają się z punktu początkowego w stronę bazy gracza. 
Jeśli podczas swojej podróży napotkają drużynę gracza, wtedy niezwłocznie zmieniają swój cel ataku.
Będziemy wyróżniać trzy archetypy jednostek w zależności od sposobu walki (bliski zasięg, średni zasięg, daleki zasięg). 
Postacie walczące na bliski zasięg mają na celu podejście w stronę najbliższego przeciwnika i wykonać atak. 
Jednostki średnio zasięgowe w momencie, w którym najbliższy przeciwnik jest odpowiednio daleko, wykonują atak dystansowy, 
w przeciwnym wypadku zachowują się tak jak jednostki walczące w zwarciu. 
Postacie  dalekodystansowe dokonują ataków dystansowych w kierunku najbliższego przeciwnika, natomiast uciekają, gdy ten podejdzie zbyt blisko.



\subsection{Mechanika zachowań klas jednostek}
\subsubsection{Jednostki walczące w zwarciu}
Zachowanie jednostek bliskiego zasięgu polega na wybraniu przeciwnika będącego w czerwonym obszarze (rys. \ref{fig:regions}), który ma najwyższą
wartość priorytetu obliczonego między innymi na podstawie odległości, ilości innych atakujących przeciwników oraz klasy celu.
Jednostka posiadająca cel ataku zmierza w jego kierunku, obierając najkrótszą drogę. Po dotarciu do celu podróży jednostki biorące udział w walce
losują naprzemiennie liczby, od których zależy wynik walki, jak i aktualnie wykorzystywana animacja.

\subsubsection{Jednostki walczące na dystans}
Takie jednostki wystrzeliwują w stronę przeciwników pociski, których celność jest reprezentowana przez dwie strefy (rys. \ref{fig:acc2}). Jedną oznaczającą 50\% celności,
czyli rozrzut, który będzie posiadała połowa pocisków, oraz drugą oznaczającą maksymalny rozrzut. Ten mechanizm pozwala na łatwe zamodelowanie
celności ze względu na odległość, wielkość celu oraz na osłony, za którymi może stać wroga jednostka.
\begin{figure}[h]
\centering
\includegraphics[width=0.6\textwidth]{images/acc}
\caption{Reprezentacja graficzna celowania widziana z perspektywy jednostki dalekozasięgowej. Czerwony okrąg reprezentuje obszar, w którym znajdzie się 50\% pocisków. Żółty obszar pokazuje maksymalny rozrzut.}
\label{fig:acc2}
\end{figure}
\subsubsection{Jednostki hybrydowe}
Obejmują zachowanie dwóch powyższych klas jednostek w zależności od odległości od przeciwnika.
W sytuacji, w której wroga jednostka znajduje się w pobliskim otoczeniu, wykorzystywany jest kod przeznaczony dla jednostek bliskiego zasięgu,
w przeciwnym przypadku wybierana jest logika jednostek zasięgowych.

\subsection{Sztuczna inteligencja przyjaznych jednostek.}

Kontrola jednostek przez gracza odbywa się za pomocą wyboru jednej z opcji w dwóch fazach: 
\begin{itemize}
\item Faza wybrania jednostek:
  \begin{itemize}
    \item wszyscy,
    \item jednostki bliskozasięgowe,
    \item jednostki hybrydowe,
    \item jednostki dalekozasięgowe,
    \item opcja anulowania wyboru.
  \end{itemize}

\item Faza wydania rozkazu:
  \begin{itemize}
    \item podążanie za graczem,
    \item zatrzymanie,
    \item atak,
    \item podejście do wskazanego przez gracza miejsca,
    \item ucieczka,
    \item opcja anulowania rozkazu.
  \end{itemize}
\end{itemize}

Wydanie rozkazu polega na kliknięciu przycisku odpowiedzialnego za wejście w tryb wydawania poleceń, a następnie
wybraniu numeru opcji reprezentującej grupę jednostek, której komenda ma dotyczyć. Ostatecznie należy podać numer rozkazu, który zostanie wydany.
Przykładowa reprezentacja graficzna systemu jest widoczna na rysunku \ref{fig:mnb}.
Wykorzystanie tego modelu pozwala na łatwe rozwinięcie systemu o dodanie nowych metod wyboru grup jednostek oraz możliwych instrukcji dla przyjaznych agentów.

\begin{figure}[h]
\centering
\includegraphics[width=1\textwidth]{uml/commands}
\caption{Wizualizacja przepływu sterowania podczas wydawania poleceń jednostkom.}
\end{figure}

\section{Widzenie przez ściany (Bartosz Strzelecki)}\label{s:wid_proj}

Do umiejętności wykorzystywanych przez gracza będzie należeć zdolność widzenia przeciwników oraz innych istotnych obiektów przez przeszkody.
Gracz po wciśnięciu przycisku przez krótki okres będzie w stanie zobaczyć sylwetki przeciwników znajdującymi się w jego polu widzenia.
Rozwiązanie jest inspirowane wcześniej wspomnianą grą \textit{Dead by Daylight} (por. \ref{chap:dbd}). Po pojawieniu się,
markery nie będą się poruszać za celem, lecz pozostać w tym samym miejscu przez czas trwania animacji.



\chapter{Badania}
\label{chap:research}
\chapter{Podsumowanie}
\label{chap:summary}

% Bibliografia, ignorujemy overfull box, bo są długie URL
\hfuzz=50pt
\printbibliography[title=\bibliographyname]
\addcontentsline{toc}{chapter}{\bibliographyname}
\hfuzz=0pt

% Wykaz rysunków
\listoffigures
\addcontentsline{toc}{chapter}{\listfigurename}

% Wykaz tabel
\listoftables
\addcontentsline{toc}{chapter}{\listtablename}

\end{document}
